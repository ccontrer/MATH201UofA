% MATH 201 Lab notes (c) by Carlos Contreras And Philippe Gaudreau
% MATH 201 Lab notes is licensed under a 
% Creative Commons Attribution 4.0 International license.
% CC BY 4.0

% You should have received a copy of the license along with this
% work. If not, see <http://creativecommons.org/licenses/by/4.0/>.

\documentclass[11pt]{article}
% MATH 201 Lab notes (c) by Carlos Contreras And Philippe Gaudreau
% MATH 201 Lab notes is licensed under a 
% Creative Commons Attribution 4.0 International license.
% CC BY 4.0

% You should have received a copy of the license along with this
% work. If not, see <http://creativecommons.org/licenses/by/4.0/>.

%% libraries
\usepackage[utf8x]{inputenc}
\usepackage{xcolor}
\usepackage[left=1.5cm,right=1.5cm,top=2.0cm,bottom=1.5cm,headheight=110pt]{geometry}
\usepackage{amsmath}
\usepackage{amssymb}
\usepackage{graphicx}
\usepackage{xifthen}
\usepackage{sverb}
\usepackage{fancyhdr}
\usepackage{mdframed}
\usepackage{textcomp}

%%%%%%%%%%%%%%%%%%%%%%%%%%%%%%%%%%%%%%%%%%%%%%%%%%%%%%%%%%%%%%%%%%%%%%%
% PDF compiling
\usepackage{ifpdf}
\ifpdf %
        \DeclareGraphicsExtensions{.pdf}%
\else %
        \DeclareGraphicsExtensions{.eps,.ps}%
\fi

%%%%%%%%%%%%%%%%%%%%%%%%%%%%%%%%%%%%%%%%%%%%%%%%%%%%%%%%%%%%%%%%%%%%%%%
% Figures path
\graphicspath{{figures/}}

%%%%%%%%%%%%%%%%%%%%%%%%%%%%%%%%%%%%%%%%%%%%%%%%%%%%%%%%%%%%%%%%%%%%%%%
% Problem counter
\newcounter{Problem}
\setcounter{Problem}{0}

%%%%%%%%%%%%%%%%%%%%%%%%%%%%%%%%%%%%%%%%%%%%%%%%%%%%%%%%%%%%%%%%%%%%%%%
% Definitions
\def\LabSolutions{\clearpage \newpage \begin{center} {\Large \it Solutions} \end{center} \setcounter{Problem}{0}}
\def\QuizSolutions{\newpage \begin{center} {\Large \it Solutions} \end{center} \setcounter{Problem}{0}}
\def\degree{\textdegree}
\def\grade#1{\begin{flushright} {\small [#1]}\\ \end{flushright} \vspace{-10pt}}
\def\codecolor{red!50!black}
\def\code#1{\textcolor{\codecolor}{\tt #1}}
\def\examname#1{%
                \ifnum\value{page}>1%
                    \newpage%
                \else%
                    \vspace*{5pt}%
                \fi%
                \large \textbf{#1} \setcounter{Problem}{0}\vspace{10pt}}
\def\topic#1{\par\needspace{2\baselineskip} \noindent \textsl{\footnotesize #1}}

 
%%%%%%%%%%%%%%%%%%%%%%%%%%%%%%%%%%%%%%%%%%%%%%%%%%%%%%%%%%%%%%%%%%%%%%%
% Environments
\newenvironment{problem}%
     {\stepcounter{Problem}%
      \begin{list}{\textbf{\arabic{Problem}}.~}{}%
      \item}%
     {\end{list}\vspace*{5pt}}

\newenvironment{solution}%
     {\indent \textit{Solution} \newline}%
     {\begin{flushright}$\blacksquare$\end{flushright}}

\newenvironment{preamble}%
     {\vspace*{1em}\begin{mdframed}[leftmargin=1cm,rightmargin=1cm]}%
     {\end{mdframed}\vspace*{1em}}

\newenvironment{multchoice}%
     {\begin{enumerate} \addtolength{\leftskip}{2em} \renewcommand{\labelenumi}{(\alph{enumi})}}
     {\end{enumerate}}

\newenvironment{formulaitem}%
     {\setlength{\leftmargini}{1.5em}\begin{itemize}%
      \setlength\itemindent{-\itemindent}%
      \renewcommand{\labelitemi}{$\rightarrow$}}%
     {\end{itemize}}


%%%%%%%%%%%%%%%%%%%%%%%%%%%%%%%%%%%%%%%%%%%%%%%%%%%%%%%%%%%%%%%%%%%%%%%
% New theorems
\newtheorem{theorem}{Theorem}


\makeatletter

%%%%%%%%%%%%%%%%%%%%%%%%%%%%%%%%%%%%%%%%%%%%%%%%%%%%%%%%%%%%%%%%%%%%%%%
%% New commands
\newcommand*{\course}[1]{\gdef\@course{#1}}
\newcommand*{\coursecode}[1]{\gdef\@coursecode{#1}}
\newcommand*{\term}[1]{\gdef\@term{#1}}
\newcommand*{\instructor}[1]{\gdef\@instructor{#1}}
\newcommand*{\lqnumber}[1]{\gdef\@lqnumber{#1}}
\newcommand*{\labtitle}[1]{\gdef\@labtitle{#1}}
\newcommand*{\quizversion}[1]{\gdef\@quizversion{#1}}
\newcommand*{\probleminfo}[1]{\noindent \textsl{\footnotesize #1}}

% Title header for labs
\newcommand\makelabtitle{%
  \begin{flushleft}%
  {\scshape \@coursecode~\@course~-- University of Alberta}\\%
  {\scshape \@term~-- Labs -- \@instructor}\\%
  {\scshape Authors: Carlos Contreras and Philippe Gaudreau}%
  \end{flushleft}%
  \begin{center}%
  {\Large \bf \@lqnumber:~\@labtitle}%
  \end{center}%
  \thispagestyle{empty}%
  \global\let\@course\@empty%
  \global\let\@labtitle\@empty%
}

% Title header for quizzes
\newcommand\makequiztitle{%
  \begin{flushleft}%
  {\scshape \@coursecode~\@course~-- University of Alberta}\\%
  {\scshape \@term~-- Labs -- \@instructor}\\%
  \end{flushleft}%
  \begin{center}%
  {\Large \bf \@lqnumber} \marginpar{\tiny\tt [\@quizversion]}%
  \end{center}%
  \thispagestyle{empty}%
  \global\let\@course\@empty%
  \global\let\@quizversion\@empty%
}

%%%%%%%%%%%%%%%%%%%%%%%%%%%%%%%%%%%%%%%%%%%%%%%%%%%%%%%%%%%%%%%%%%%%%%%
% Fancy header package
\fancyhead[L]{\small {\scshape \@coursecode~-- \@lqnumber~-- \@term~-- \@instructor}}
\pagestyle{fancy}

\makeatother


\usepackage{hyperref}
\usepackage{cancel}


\begin{document}

\course{Differential Equations}
\coursecode{MATH 201}
\term{Winter 2018}
\instructor{Carlos Contreras}
\lqnumber{Lab 12}
\labtitle{Heat and wave equation}
\makelabtitle


\topic{Heat equation with zero BC's.}
\begin{problem}
Solve the heat flow problem,
\begin{equation*} 
\begin{split}
& \dfrac{\partial u }{\partial t } (x,t) = 3 \dfrac{\partial^2 u}{\partial x^2}(x,t), \qquad 0<x<\pi, \quad t>0, \\
& u(0,t) = u(\pi,t)=0, \qquad t>0, \\
& u(x,0) = \sin(x)-6 \sin(4x), \qquad 0<x<\pi.
\end{split}
\end{equation*}
\end{problem}




\topic{Heat equation with zero derivative BC's.}
\begin{problem}
Solve the heat flow problem
\begin{equation*} \begin{split}
& \dfrac{\partial u }{\partial t } (x,t) = \alpha \dfrac{\partial^2 u}{\partial x^2}(x,t), \qquad 0<x<1, \quad t>0, \\
& \dfrac{\partial u }{\partial x } (0,t) = \dfrac{\partial u }{\partial x } (1,t)=0, \qquad t>0, \\
& u(x,0) = x(1-x), \qquad 0<x<1.
\end{split}\end{equation*}
\end{problem}


\topic{Heat equation with non-zero BC's.}
\begin{problem}
Find a formal solution to the initial value problem,
\begin{equation*} \begin{split}
& \dfrac{\partial u }{\partial t } (x,t) = \dfrac{\partial^2 u}{\partial x^2}(x,t), \qquad 0<x<\pi, \quad t>0, \\
& u(0,t) = 0, \quad u(\pi,t)= 3\pi, \qquad t>0, \\
& u(x,0) = 0, \qquad 0<x<\pi.
\end{split}\end{equation*}
\end{problem}


\topic{Heat equation transient and steady state solution.}
\begin{problem}
Find a formal solution to the initial value problem,
\begin{equation*} \begin{split}
& \dfrac{\partial u }{\partial t } (x,t) = \dfrac{\partial^2 u}{\partial x^2}(x,t) + 6 x -2, \qquad 0<x<1, \quad t>0, \\
& u(0,t) = 0, \quad u(1,t)= -1, \qquad t>0, \\
& u(x,0) = -x^{3}, \qquad 0<x<1.
\end{split}\end{equation*}
\end{problem}


\topic{Heat equation special case.}
\begin{problem}
Solve the heat flow problem,
\begin{equation*} 
\begin{split}
& \dfrac{\partial u }{\partial t } (x,t) = \dfrac{\partial^2 u}{\partial x^2}(x,t) -3u, \qquad 0<x<\pi, \quad t>0, \\
& \dfrac{\partial u }{\partial x } (0,t) = \dfrac{\partial u }{\partial x } (\pi,t)=0, \qquad t>0, \\
& u(x,0) = 2 + \cos x - 5 \cos 4x, \qquad 0<x<\pi.
\end{split}
\end{equation*}
\end{problem}


\topic{Wave equation}
\begin{problem}
Find a formal solution to the vibrating string problem governed by the given initial-boundary value problem
\begin{equation*} \begin{array}{lr}
\dfrac{ \partial^2 u }{\partial t^2} =  4 \dfrac{ \partial^2 u }{\partial x^2} , & 0<x<\pi, \quad t>0 \\
 u(0,t)  =  u(\pi,t) =0,  & t>0 \\
 u(x,0)  =  x^2(\pi-x), & 0<x<\pi \\
 \frac{ \partial u }{\partial t}(x,0) =  \dfrac{(x-\pi)^3}{3}, & 0<x<\pi 
\end{array}\end{equation*}
\end{problem}








%%%%%%%%%%%%%%%%%%%%%%%%%%%%%%%%%%%%%%%%%%%%%%%%%%%%%%%%%%%%%%%%%%%%%%%%%%%%%%%%%%%%%%%%%%%%%%%%%%%



\LabSolutions


Theory and problems from: Nagel, Saff \& Sneider, \textit{Fundamentals of Differential Equations}, Eighth Edition, Adisson--Wesley.

\begin{preamble}
\begin{formulaitem}

\item The eigenvalue problem
\begin{equation*}
X^{\prime \prime} - \lambda X =0\,; \qquad  0<x<L \,, \qquad X(0)=0  \,,\qquad X(L) =0,
\end{equation*}
has eigenvalues and eigenfunctions
\[\lambda_{n}=-\left( \frac{n\pi}{L}\right)^{2}, \quad X_{n}(x)=B_{n}\sin\left( \frac{n\pi x}{L} \right), \quad  n\geq 1.\]


\item The eigenvalue problem
\begin{equation*}
X^{\prime \prime} - \lambda X =0\,; \qquad  0<x<L \,, \qquad X'(0)=0  \,,\qquad X'(L) =0,
\end{equation*}
has eigenvalues and eigenfunctions
\[\lambda_{n}=-\left( \frac{n\pi}{L}\right)^{2}, \quad X_{n}(x)=B_{n}\cos\left( \frac{n\pi x}{L} \right), \quad  n\geq 0.\]

\item For the heat equation with \textbf{non-zero boundary conditions and external force}
\begin{equation*} \begin{split}
& \dfrac{\partial u }{\partial t } (x,t) = \alpha\dfrac{\partial^2 u}{\partial x^2}(x,t) + g(x), \qquad 0<x<L, \quad t>0, \\
& u(0,t) = U_{1}, \quad u(L,t)= U_{2}, \qquad t>0, \\
& u(x,0) = f(x), \qquad 0<x<L,
\end{split}\end{equation*}
we apply the change of variable
\[u(x,t)=v(x)+w(x,t), \,\, \Rightarrow w(x,t)=u(x, t)-v(x),\]
to arrive the zero BC's problem
\begin{equation*} \begin{split}
& \dfrac{\partial w }{\partial t } (x,t) = \alpha\dfrac{\partial^2 w}{\partial x^2}(x,t), \qquad 0<x<L, \quad t>0, \\
& w(0,t) = w(L,t)= 0, \qquad t>0, \\
& w(x,0) = f(x) - v(x), \qquad 0<x<L,
\end{split}\end{equation*}
and the second order problem
\begin{equation*} \begin{split}
& v''(x) = -\tfrac{1}{\alpha}g(x) \qquad 0<x<L, \\
& v(0) = U_{1}, \quad v(L)= U_{2}.
\end{split}\end{equation*}
We first solve for $v(x)$, then solve $w(x,t)$, and finally write the solution in terms of $u(x,t)$.

If $g(x)=0$, then, clearly
\[v(x)=(U_{2}-U_{1})\frac{x}{L}+U_{1}.\]

The functions $w(x, t)$ and $v(x)$ are called \textbf{transient} and \textbf{steady state} solutions of $u(x,t)$.

\end{formulaitem}
\end{preamble}



\begin{problem}
Solve the heat flow problem,
\begin{equation*} \begin{split}
& \dfrac{\partial u }{\partial t } (x,t) = 3 \dfrac{\partial^2 u}{\partial x^2}(x,t), \qquad 0<x<\pi, \quad t>0, \\
& u(0,t) = u(\pi,t)=0, \qquad t>0, \\
& u(x,0) = \sin(x)-6 \sin(4x), \qquad 0<x<\pi.
\end{split}\end{equation*}
\end{problem}

\begin{solution}
We suppose that the solution $u(x,t)$ can be written in the following way:
\begin{eqnarray*}
u(x,t) & = & X(x)T(t)\\
\dfrac{\partial u }{\partial t }(x,t) & = & X(x)T^{\prime}(t) \\
\dfrac{\partial^2 u}{\partial x^2}(x,t) & = & X^{\prime \prime}(x)T(t)
\end{eqnarray*}
Inserting this assumption into our PDE, we obtain:
\begin{eqnarray*}
X(x)T^{\prime}(t) & = & 3 X^{\prime \prime}(x)T(t) \\
\Rightarrow \quad \dfrac{T^{\prime}(t)}{3T(t)} & = &  \dfrac{X^{\prime \prime}(x)}{X(x)}
\end{eqnarray*}
The only way this can be true is if they are both equal to some constant $\lambda$.
Hence, we obtain two equations:
\begin{eqnarray*}
T^{\prime}(t) - 3\lambda T(t) & = & 0\\
X^{\prime \prime}(x) - \lambda X(x) & = & 0
\end{eqnarray*}
Using our initial conditions $u(0,t)  =  u(\pi,t)=0$, we can see that for the $X(x)$ equation, we have: $X(0)=X(\pi)=0$.

Solving the heat flow problem reduces now to solving the three following problems.

\textsl{1.} We now know from previous labs that the eigenvalue problem
\begin{eqnarray*}
X^{\prime \prime}(x) - \lambda X(x) =0; & \quad & 0<x<\pi \\
X(0)=0 & \quad & X(\pi) =0,
\end{eqnarray*}
has eigenvalues $$\lambda_{n} = -n^2, \quad n = 1,2 , \dots$$ with eigenfunctions $$X_{n}(x) = b_{n} \sin(nx),$$ for some arbitrary constants $b_{n}$.

\textsl{2. } We can now find a Fourier series representation for the function $f(x)=\sin(x)-6 \sin(4x)$ in terms of these eigenfunctions $X_{n}(x)$
\begin{eqnarray*}
\sin(x)-6 \sin(4x) = \sum_{n=1}^{\infty} b_{n} \sin(nx).
\end{eqnarray*}
From this representation, it is obvious that we have the following values for the coefficients.
\begin{equation*}
b_{n} = \begin{cases}
1 & n=1 \\
-6  & n=4 \\
0  & {\rm otherwise} \\
\end{cases}
\end{equation*}
\textsl{3. } We can now solve the $T(t)$ equation by inserting the eigenvalue $ \lambda = -n^2$
\begin{eqnarray*}
T_{n}^{\prime}(t) + 3n^2 T_{n}(t) & = & 0.
\end{eqnarray*}
Solving this equation,we have:
\begin{eqnarray*}
T_{n}(t) & = & Ae^{-3n^2 t},
\end{eqnarray*}
for some constant $A$. However, we have to set $T_{n}(0)=1$ in order for our previous values of $b_{n}$ to still be valid. Clearly $A=1$.

Using the three previous results, the solution to the heat flow problem is then given by
\begin{eqnarray*}
u(x,t) & = & \sum_{n=1}^{\infty}X_{n}(x)T_{n}(t) \\
& = &\sum_{n=1}^{\infty}b_{n}\sin(nx)e^{-3n^2 t} \\
& = & \sin(x)e^{-3t} - 6\sin(4x)e^{-48 t}
\end{eqnarray*}
In summary, the solution to the following heat flow problem:
\begin{eqnarray*}
\dfrac{\partial u }{\partial t } (x,t)  & = & 3 \dfrac{\partial^2 u}{\partial x^2}(x,t), \quad 0<x<\pi, \quad t>0, \\
u(0,t) & = & u(\pi,t)=0, \quad t>0 \\
u(x,0) & = & \sin(x)-6 \sin(4x)
\end{eqnarray*}
is given by
\begin{eqnarray*}
\boxed{u(x,t) = \sin(x)e^{-3t} - 6\sin(4x)e^{-48 t}}.
\end{eqnarray*}
\end{solution}



\begin{problem}
Solve the heat flow problem
\begin{equation*} \begin{split}
& \dfrac{\partial u }{\partial t } (x,t) = \alpha \dfrac{\partial^2 u}{\partial x^2}(x,t), \qquad 0<x<1, \quad t>0, \\
& \dfrac{\partial u }{\partial x } (0,t) = \dfrac{\partial u }{\partial x } (1,t)=0, \qquad t>0, \\
& u(x,0) = x(1-x), \qquad 0<x<1.
\end{split}\end{equation*}
\end{problem}

\begin{solution}
Using the separation of variables method
\begin{eqnarray*}
u(x,t) = X(x)T(t), \qquad
\dfrac{\partial u }{\partial t }(x,t) = X(x)T^{\prime}(t), \qquad
\dfrac{\partial^2 u}{\partial x^2}(x,t) = X^{\prime \prime}(x)T(t),
\end{eqnarray*}
in the partial differential equation
\begin{eqnarray*}
X(x)T^{\prime}(t) = \alpha X^{\prime \prime}(x)T(t) \quad 
\Rightarrow \quad \dfrac{T^{\prime}(t)}{\alpha T(t)} = \dfrac{X^{\prime \prime}(x)}{X(x)} = \lambda.
\end{eqnarray*}

Then, solving the heat flow problem reduces now to solving the following three problems.

\textsl{1.} The eigenvalue problem
\begin{eqnarray*}
X^{\prime \prime}(x) - \lambda X(x) =0, \quad 0<x<1; \qquad
X'(0)=0, \quad & X'(1) =0,
\end{eqnarray*}
has eigenvalues 
$$\lambda_{n} = -n^2\pi^{2}, \quad n = 0,1,2 , \dots$$ 
with eigenfunctions 
$$\boxed{X_{n}(x) = B_{n}\cos(n\pi x)},\quad n = 0,1,2 , \dots$$


\textsl{2. } Using the eigenvalue $\lambda = -n^2\pi^{2}$ found in \textsl{1}, is easy to see that 
\begin{eqnarray*}
T_{n}^{\prime}(t) = - \alpha n^{2}\pi^{2} T_{n}(t),
\end{eqnarray*}
has solution
\begin{eqnarray*}
\boxed{T_{n}(t) = A_{n}e^{-\alpha n^{2}\pi^{2} t}},\quad n = 0,1,2 , \dots
\end{eqnarray*}

\textbf{Note.} The eigenfunction $X_{n}=\cos(n\pi x)$ implies that we need the Fourier \textsl{cosine} series.

\textsl{3. } The Fourier cosine series representation for the function $f(x)=x(1-x)$ 
\begin{eqnarray*}
x(1-x) = \frac{a_{0}}{2}+\sum_{n=1}^{\infty} a_{n} \cos(n\pi x),
\end{eqnarray*}
has coefficients (see a previous lab notes)
\begin{eqnarray*}
\boxed{a_{0} = \frac{1}{3}, \quad a_{n} = \frac{2((-1)^{n+1}-1)}{n^{2}\pi^{2}}}, \quad n = 0,1,2,\dots
\end{eqnarray*}

Putting the three previous results together, the solution to the heat flow problem is then given by
\begin{eqnarray*}
\boxed{u(x,t) = \frac{1}{6} + \sum_{n=1}^{\infty}\frac{2((-1)^{n+1}-1)}{n^{2}\pi^{2}}e^{-\alpha n^{2}\pi^{2}t}\cos (n\pi x)}.
\end{eqnarray*}


\end{solution}





\begin{problem}
Find a formal solution to the initial value problem,
\begin{equation*} \begin{split}
& \dfrac{\partial u }{\partial t } (x,t) = \dfrac{\partial^2 u}{\partial x^2}(x,t), \qquad 0<x<\pi, \quad t>0, \\
& u(0,t) = 0, \quad u(\pi,t)= 3\pi, \qquad t>0, \\
& u(x,0) = 0, \qquad 0<x<\pi.
\end{split}\end{equation*}
\end{problem}

\begin{solution}
Note that we have non-zero boundary conditions. So, we apply the appropriate change of variable
\[u(x,t)=v(x)+w(x, t)=3x + w(x,t),\]
since
\[v(x)=(3\pi-0)\frac{x}{\pi}+0=3x.\]

Then, the new problem is to solve the alternate Heat equation \textsl{with zero boundary conditions}
\begin{equation*} \begin{split}
& \dfrac{\partial w }{\partial t } (x,t) = \dfrac{\partial^2 u}{\partial x^2}(x,t), \qquad 0<x<\pi, \quad t>0, \\
& w(0,t) = 0, \quad w(\pi,t)= 0, \qquad t>0, \\
& w(x,0) = -3x, \qquad 0<x<\pi.
\end{split}\end{equation*}
Now, we use separation of variables
\[w(x,t)=X(x)T(t),\]
\[\Rightarrow\, \frac{\partial w}{\partial t}(x,t)=X(x)T'(t),\quad \frac{\partial^{2} w}{\partial x^{2}}(x,t)=X''(x)T(t).\]
Then 
\[X(x)T'(t)=X''(x)T(t) \,\,\Rightarrow\,\, \frac{T'(t)}{T(t)}=\frac{X''(x)}{X(x)}=\lambda.\]
Which becomes
\[\begin{split}T'(t)-\lambda T(t)&=0 \\ X''(x)-\lambda X(x)&=0 \end{split}.\]

\textsl{1. } First, we solve the second order eigenvalue problem for $X(x)$.
\[X''(x)-\lambda X(x)=0, \qquad 0<x<\pi, \qquad  X(x)=X(\pi)=0.\]
The boundary values come from setting zero boundary conditions in the Heat equation problem (we forced those boundary conditions). This eigenvalue problem has solution 
\[\begin{split}
  \lambda_{n}=-n^{2}, \,\, n=1, 2, \dots, \\
  X_{n}(x)=b_{n}\sin(nx), \,\, n=1, 2, \dots.
  \end{split}
\]

Note: the eigenvalues will be used in the other differential equation (eigenvalue problem) for $T(t)$, while the coefficients $b_{n}$ will be determined by the initial condition $w(x,0)=-3x$. 

\textsl{2. } Now, for the first order eigenvalue problem
\[T'(t)-\lambda T(t)=0, \quad T(0)=1,\]
we use the $\lambda_{n}$ previously found (we can assume $T(0)=1$, but why?\footnote{$T_{n}(0)=1$ for convenience, since we had to determine the coefficients $b_{n}$, i.e, $w(x,0)=-3x=\sum T_{n}(0)X_{n}(x)= \sum \cancelto{1}{T_{n}(0)} b_{n}\sin(nx)$}). This leads to the first order differential equations
\[T_{n}'(t) +n^{2} T_{n}(t)=0, \quad T_{n}(0)=1.\]
Which has solution ($A=1$ from initial condition)
\[T_{n}(t)=Ae^{-n^{2}t}=e^{-n^{2}t}.\]

\textsl{3.} Since the eigen function are sine, we require
\[-3x=\sum_{i=1}^{\infty}b_{n}\sin(nx),\]
on $(0,\pi)$, which is the Fourier sine series of $-3x$. Thus, the coefficients are
\begin{align*}
b_{n}=\frac{2}{\pi}\int_{0}^{\pi}(-3x)\sin(nx)dx=-\frac{6}{\pi}\left[ -\frac{1}{n}x\cos(nx)|^{\pi}_{0} + \frac{1}{n^{2}}\sin(nx)|^{\pi}_{0} \right] = 6\frac{(-1)^{n}}{n}.
\end{align*}


\textsl{4. } The next step is to combine $X_{n}(x)$ and $T_{n}(t)$ as 
\[w(x,t)=\sum_{i=1}^{\infty}X_{n}(x)T_{n}(t).\]
Why $\textstyle \sum$? \footnote{Superposition principle!}. That is 
\[w(x, t) = 6 \sum_{i=1}^{\infty}\frac{(-1)^{n}}{n}\sin(nx)e^{-n^{2}t}.\]
Finally, changing back to $u(x,t)$
\[\boxed{u(x, t) = 3x + 6 \sum_{i=1}^{\infty}\frac{(-1)^{n}}{n}\sin(nx)e^{-n^{2}t}}.\]
\end{solution}


\begin{problem}
Find a formal solution to the initial value problem,
\begin{equation*} \begin{split}
& \dfrac{\partial u }{\partial t } (x,t) = \dfrac{\partial^2 u}{\partial x^2}(x,t) + 6 x -2, \qquad 0<x<1, \quad t>0, \\
& u(0,t) = 0, \quad u(1,t)= -1, \qquad t>0, \\
& u(x,0) = -x^{3}, \qquad 0<x<1.
\end{split}\end{equation*}
\end{problem}
\begin{solution}
Note that the non-zero boundary conditions and the external force $6x-2$ imply we nee to use the change of variable
\[u(x,t)=w(x,t)+v(x).\]
Then,
\[u_{t}(x,t)=w_{t}(x,t), \qquad u_{xx}(x,t)=w_{xx}(x,t)+''v(x).\]
We now use the equation
\[\dfrac{\partial w }{\partial t } (x,t) = \dfrac{\partial^2 w}{\partial x^2}(x,t) + v''(x) + 6 x -2,\]
the boundary conditions
\[u(0,t) = w(0,t) + v(0) = 0, \quad u(1,t)= w(1,t) + v(1) = -1,\]
and the initial condition
\[u(x,0) = w(x,0) + v(x) = -x^{3}.\]
The idea is to force $v(x)$ to absorbe the external force and the non-zero boundary conditions so the heat equation in for $w$ has a familiar form. In other words, we need to solve
\begin{equation*} \begin{split}
& \dfrac{\partial w }{\partial t } (x,t) = \dfrac{\partial^2 w}{\partial x^2}(x,t), \qquad 0<x<1, \quad t>0, \\
& w(0,t) = w(1,t)= 0, \qquad t>0, \\
& w(x,0) = -x^{3} - v(x), \qquad 0<x<1,
\end{split}\end{equation*}
given that $v(x)$ is the solution to
\begin{equation}
v''(x)+6x-2=0,\quad v(0)=0, \quad v(1)=-1.
\end{equation}
The second order equation is easy to solve
\[v(x)=-x^{3}+x^{2}-x.\]
The heat equation with zero BC's and initial condition $w(x,0)=x^{3}-x^{3}+x^{2}-x=x(1-x)$ have solution (exercise)
\begin{eqnarray*}
w(x,t) = \sum_{n=1}^{\infty}\frac{4(1-(-1)^{n})}{n^{3}\pi^{3}}e^{-\alpha n^{2}\pi^{2}t}\sin (n\pi x).
\end{eqnarray*}
Thus, the solution to the original heat problem is $w(x,t)+v(x)$
\begin{eqnarray*}
\boxed{u(x,t) = -x(x^{2}-x+1)+\sum_{n=1}^{\infty}\frac{4(1-(-1)^{n})}{n^{3}\pi^{3}}e^{-\alpha n^{2}\pi^{2}t}\sin (n\pi x)}.
\end{eqnarray*}
\end{solution}




\begin{problem}
Solve the heat flow problem,
\begin{equation*} 
\begin{split}
& \dfrac{\partial u }{\partial t } (x,t) = \dfrac{\partial^2 u}{\partial x^2}(x,t) -3u, \qquad 0<x<\pi, \quad t>0, \\
& \dfrac{\partial u }{\partial x } (0,t) = \dfrac{\partial u }{\partial x } (\pi,t)=0, \qquad t>0, \\
& u(x,0) = 2 + \cos x - 5 \cos 4x, \qquad 0<x<\pi.
\end{split}
\end{equation*}
\end{problem}

\begin{solution}
Using the separation of variables method
\begin{eqnarray*}
u(x,t) = X(x)T(t), \qquad
\dfrac{\partial u }{\partial t }(x,t) = X(x)T^{\prime}(t), \qquad
\dfrac{\partial^2 u}{\partial x^2}(x,t) = X^{\prime \prime}(x)T(t),
\end{eqnarray*}
in the partial differential equation
\begin{eqnarray*}
X(x)T^{\prime}(t) = X^{\prime \prime}(x)T(t) -3X(x)T(t) \quad 
\Rightarrow \quad \dfrac{T^{\prime}(t)}{T(t)} + 3 = \dfrac{X^{\prime \prime}(x)}{X(x)} = \lambda.
\end{eqnarray*}

Note that the $3$ term is intencionally placed in the $T(t)$ side, so the eigenvalue problem is known and simple.

Then, solving the heat flow problem reduces now to solving the following three problems.

\textsl{1.} The eigenvalue problem
\begin{eqnarray*}
X^{\prime \prime}(x) - \lambda X(x)=0, \quad 0<x<\pi; \qquad
X'(0)=0, \quad & X'(\pi) =0,
\end{eqnarray*}
has eigenvalues 
$$\lambda_{n} = -n^2, \quad n = 0,1,2 , \dots$$ 
with eigenfunctions 
$$\boxed{X_{n}(x) = B_{n}\cos(n x)},\quad n = 0,1,2 , \dots$$


\textsl{2. } Using the eigenvalue $\lambda = -n^2$ found in \textsl{1}, is easy to see that 
\begin{eqnarray*}
T_{n}^{\prime}(t) = - (3+ n^{2})T_{n}(t),
\end{eqnarray*}
has solution
\begin{eqnarray*}
\boxed{T_{n}(t) = A_{n}e^{-(3+n^{2}) t}},\quad n = 0,1,2 , \dots
\end{eqnarray*}

\textbf{Note.} The eigenfunction $X_{n}=\cos(n\pi x)$ implies that we need the Fourier \textsl{cosine} series.

\textsl{3. } The Fourier cosine series representation for the function $f(x)=2+\cos(x)-5\cos(4x)$ has coefficients
\begin{eqnarray*}
\boxed{\frac{a_{0}}{2} = 2, \quad a_{1} = 1, \quad a_{4} = -5}.
\end{eqnarray*}

Putting the three previous results together, the solution to the heat flow problem is then given by
\[u(x,t) = \frac{a_{0}}{2}e^{-3t}+\sum_{n=1}^{\infty}a_{n}e^{-(3+ n^{2})t}\cos (n x).\]
\[\Rightarrow
\boxed{u(x,t) = 2e^{-3t}+e^{-4t}\cos(x)-5e^{-19t}\cos(4x)}.
\]


\end{solution}







\begin{problem}
Find a formal solution to the vibrating string problem governed by the given initial-boundary value problem
\begin{equation*} \begin{array}{lr}
\dfrac{ \partial^2 u }{\partial t^2} =  4 \dfrac{ \partial^2 u }{\partial x^2} , & 0<x<\pi, \quad t>0 \\
 u(0,t)  =  u(\pi,t) =0,  & t>0 \\
 u(x,0)  =  x^2(\pi-x), & 0<x<\pi \\
 \frac{ \partial u }{\partial t}(x,0) =  \dfrac{(x-\pi)^3}{3}, & 0<x<\pi 
\end{array}\end{equation*}
\end{problem}

\begin{solution}
 Using the method of variation of parameters, the general solution of any wave equation of the form:

\begin{eqnarray*}
\dfrac{ \partial^2 u }{\partial t^2} & = & \alpha^2 \dfrac{ \partial^2 u }{\partial x^2} , \quad \quad  \quad 0<x<L, \quad t>0 \\
u(0,t) & = & u(L,t) =0,  \quad \quad t>0 \\
u(x,0) & = & f(x), \quad \quad \quad \quad  0<x<L \\
\frac{ \partial u }{\partial t}(x,0) & = & g(x), \quad \quad \quad \quad \quad 0<x<L
\end{eqnarray*}

is given by:

\begin{eqnarray*}
u(x,t) = \sum_{n=1}^{\infty} \left[ a_{n} \cos \left( \dfrac{n\pi \alpha t}{L} \right) + b_{n} \sin \left( \dfrac{n\pi \alpha t}{L} \right) \right] \sin \left( \dfrac{n\pi x}{L} \right)
\end{eqnarray*}

where the coefficients $a_{n}$ and $b_{n}$ are determined from the Fourier Sine Series

\begin{eqnarray}
f(x) & = & \sum_{n=1}^{\infty} a_{n} \sin \left( \dfrac{n\pi x}{L} \right) \nonumber \\
g(x) & = & \sum_{n=1}^{\infty} b_{n} \left( \dfrac{n\pi \alpha}{L} \right) \sin \left( \dfrac{n\pi x}{L} \right) \label{formula: b_n coefficients}
\end{eqnarray}

In our case, we have: $\alpha =2, L = \pi, f(x) = x^2(\pi-x)$ and $g(x) = \dfrac{(x-\pi)^3}{3}$.

We need to to find the coefficients $a_{n}$ and $b_{n}$. To do this we have to find the Fourier Sine Series of $f(x)= x^2(\pi-x)$  and $g(x)= \dfrac{(x-\pi)^3}{3}$ on the interval given above, namely $0<x<\pi$.

The coefficients $a_{n}$ are given by the following integral:

\begin{eqnarray*}
a_{n} & = & \dfrac{2}{L} \int_{0}^{L} f(x) \sin \left( \dfrac{n \pi x}{L} \right) {\rm d} x \\
& = & \dfrac{2}{\pi} \int_{0}^{\pi} x^2(\pi-x) \sin \left( n x \right) {\rm d} x \\
& = & \dfrac{2}{\pi} \int_{0}^{\pi} (\pi x^2-x^3) \sin \left( n x \right) {\rm d} x \\
& = & \dfrac{2}{\pi} \left[ (\pi x^2-x^3) \left( - \dfrac{\cos \left( n x \right)}{n} \right) \right]_{0}^{\pi}-  \dfrac{2}{\pi} \int_{0}^{\pi} (2\pi x-3x^2) \left( - \dfrac{\cos \left( n x \right)}{n} \right) {\rm d} x \\
& = & 0 +  \dfrac{2}{n\pi} \int_{0}^{\pi} (2\pi x-3x^2)\cos \left( n x \right){\rm d} x \\
& = &  \dfrac{2}{n\pi} \left[ (2\pi x-3x^2)\left(  \dfrac{\sin \left( n x \right)}{n} \right) \right]_{0}^{\pi} -  \dfrac{2}{n\pi}  \int_{0}^{\pi} (2\pi -6x)\left(  \dfrac{\sin \left( n x \right)}{n} \right){\rm d} x \\
& = &  0 -  \dfrac{4}{n^2\pi}  \int_{0}^{\pi} (\pi -3x)\sin \left( n x \right){\rm d} x \\
& = &  0 -  \dfrac{4}{n^2\pi} \left[(\pi -3x)\left( - \dfrac{\cos \left( n x \right)}{n} \right) \right]_{0}^{\pi}  + \dfrac{4}{n^2\pi} \int_{0}^{\pi} (-3)\left( - \dfrac{\cos \left( n x \right)}{n} \right){\rm d} x \\
& = &  \dfrac{4}{n^3\pi} \left[(\pi -3x)\cos \left( n x \right) \right]_{0}^{\pi}  + \dfrac{12}{n^3\pi} \int_{0}^{\pi} \cos \left( n x \right) {\rm d} x \\
& = &  \dfrac{4}{n^3\pi} \left[(-2\pi)\cos \left( n \pi \right) - \pi \right]  + \dfrac{12}{n^3\pi} \left[ \left(  \dfrac{\sin \left( n x \right)}{n} \right) \right]_{0}^{\pi} \\
& = &  \dfrac{4}{n^3} \left(-2(-1)^n - 1 \right)  + 0 \\
& = &  \dfrac{4( 2(-1)^{n+1} - 1)  }{n^3} 
\end{eqnarray*}

If we inspect equation \eqref{formula: b_n coefficients} more carefully, we see that the coefficients $b_{n} \left( \dfrac{n\pi \alpha}{L} \right)$ satisfy the same integral as the coefficients $a_{n}$ with the exception that  we are evaluating $g(x)$ not $f(x)$. Otherwise stated:

\begin{eqnarray*}
b_{n} \left( \dfrac{n\pi \alpha}{L} \right) & = & \dfrac{2}{L} \int_{0}^{L} g(x) \sin \left( \dfrac{n \pi x}{L} \right) {\rm d} x \\
\Rightarrow \quad b_{n}  & = & \dfrac{2}{n\pi \alpha} \int_{0}^{L} g(x) \sin \left( \dfrac{n \pi x}{L} \right) {\rm d} x \\
 & = & \dfrac{1}{n\pi} \int_{0}^{\pi} \dfrac{(x-\pi)^3}{3} \sin \left( n x \right) {\rm d} x \\
 & = & \dfrac{1}{n\pi} \left[ \dfrac{(x-\pi)^3}{3} \left( - \dfrac{\cos \left( n x \right)}{n} \right) \right]_{0}^{\pi} -  \dfrac{1}{n\pi} \int_{0}^{\pi} (x-\pi)^2 \left( - \dfrac{\cos \left( n x \right)}{n} \right) {\rm d} x \\
  & = & -\dfrac{\pi^2}{3n^2} + \dfrac{1}{n^2\pi} \int_{0}^{\pi} (x-\pi)^2 \cos \left( n x \right){\rm d} x \\
& = & -\dfrac{\pi^2}{3n^2} + \dfrac{1}{n^2\pi} \left[(x-\pi)^2 \left(  \dfrac{\sin \left( n x \right)}{n} \right) \right]_{0}^{\pi} - \dfrac{1}{n^2\pi} \int_{0}^{\pi} 2(x-\pi) \left(  \dfrac{\sin \left( n x \right)}{n} \right) {\rm d} x \\ 
& = & -\dfrac{\pi^2}{3n^2} +0 - \dfrac{2}{n^3\pi} \int_{0}^{\pi} (x-\pi) \sin \left( n x \right) {\rm d} x \\ 
& = & -\dfrac{\pi^2}{3n^2}  - \dfrac{2}{n^3\pi} \left[ (x-\pi)\left( - \dfrac{\cos \left( n x \right)}{n} \right) \right]_{0}^{\pi} + \dfrac{2}{n^3\pi}  \int_{0}^{\pi} (1) \left( - \dfrac{\cos \left( n x \right)}{n} \right) {\rm d} x \\ 
& = & -\dfrac{\pi^2}{3n^2}  - \dfrac{2}{n^4\pi} \left[ 0-(-\pi)\left( - 1 \right) \right] - \dfrac{2}{n^4\pi} \left[ \dfrac{\sin \left( n x \right)}{n}  \right ]_{0}^{\pi} \\
& = & -\dfrac{\pi^2}{3n^2} + \dfrac{2}{n^4} - 0 \\
& = & \dfrac{2}{n^4} -\dfrac{\pi^2}{3n^2}   \\
\end{eqnarray*}

Hence, our general solution is given by:

\begin{eqnarray*}
\boxed{u(x,t) = \sum_{n=1}^{\infty} \left[ \left(\dfrac{4( 2(-1)^{n+1} - 1)  }{n^3}  \right) \cos\left(2 n  t \right) +\left(\dfrac{2}{n^4} -\dfrac{\pi^2}{3n^2} \right) \sin\left(2 n  t \right) \right]  \sin\left( n x \right)}.
\end{eqnarray*}
\end{solution}






\end{document}
