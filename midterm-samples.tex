\documentclass[11pt]{article}
% MATH 201 Lab notes (c) by Carlos Contreras And Philippe Gaudreau
% MATH 201 Lab notes is licensed under a 
% Creative Commons Attribution 4.0 International license.
% CC BY 4.0

% You should have received a copy of the license along with this
% work. If not, see <http://creativecommons.org/licenses/by/4.0/>.

%% libraries
\usepackage[utf8x]{inputenc}
\usepackage{xcolor}
\usepackage[left=1.5cm,right=1.5cm,top=2.0cm,bottom=1.5cm,headheight=110pt]{geometry}
\usepackage{amsmath}
\usepackage{amssymb}
\usepackage{graphicx}
\usepackage{xifthen}
\usepackage{sverb}
\usepackage{fancyhdr}
\usepackage{mdframed}
\usepackage{textcomp}

%%%%%%%%%%%%%%%%%%%%%%%%%%%%%%%%%%%%%%%%%%%%%%%%%%%%%%%%%%%%%%%%%%%%%%%
% PDF compiling
\usepackage{ifpdf}
\ifpdf %
        \DeclareGraphicsExtensions{.pdf}%
\else %
        \DeclareGraphicsExtensions{.eps,.ps}%
\fi

%%%%%%%%%%%%%%%%%%%%%%%%%%%%%%%%%%%%%%%%%%%%%%%%%%%%%%%%%%%%%%%%%%%%%%%
% Figures path
\graphicspath{{figures/}}

%%%%%%%%%%%%%%%%%%%%%%%%%%%%%%%%%%%%%%%%%%%%%%%%%%%%%%%%%%%%%%%%%%%%%%%
% Problem counter
\newcounter{Problem}
\setcounter{Problem}{0}

%%%%%%%%%%%%%%%%%%%%%%%%%%%%%%%%%%%%%%%%%%%%%%%%%%%%%%%%%%%%%%%%%%%%%%%
% Definitions
\def\LabSolutions{\clearpage \newpage \begin{center} {\Large \it Solutions} \end{center} \setcounter{Problem}{0}}
\def\QuizSolutions{\newpage \begin{center} {\Large \it Solutions} \end{center} \setcounter{Problem}{0}}
\def\degree{\textdegree}
\def\grade#1{\begin{flushright} {\small [#1]}\\ \end{flushright} \vspace{-10pt}}
\def\codecolor{red!50!black}
\def\code#1{\textcolor{\codecolor}{\tt #1}}
\def\examname#1{%
                \ifnum\value{page}>1%
                    \newpage%
                \else%
                    \vspace*{5pt}%
                \fi%
                \large \textbf{#1} \setcounter{Problem}{0}\vspace{10pt}}
\def\topic#1{\par\needspace{2\baselineskip} \noindent \textsl{\footnotesize #1}}

 
%%%%%%%%%%%%%%%%%%%%%%%%%%%%%%%%%%%%%%%%%%%%%%%%%%%%%%%%%%%%%%%%%%%%%%%
% Environments
\newenvironment{problem}%
     {\stepcounter{Problem}%
      \begin{list}{\textbf{\arabic{Problem}}.~}{}%
      \item}%
     {\end{list}\vspace*{5pt}}

\newenvironment{solution}%
     {\indent \textit{Solution} \newline}%
     {\begin{flushright}$\blacksquare$\end{flushright}}

\newenvironment{preamble}%
     {\vspace*{1em}\begin{mdframed}[leftmargin=1cm,rightmargin=1cm]}%
     {\end{mdframed}\vspace*{1em}}

\newenvironment{multchoice}%
     {\begin{enumerate} \addtolength{\leftskip}{2em} \renewcommand{\labelenumi}{(\alph{enumi})}}
     {\end{enumerate}}

\newenvironment{formulaitem}%
     {\setlength{\leftmargini}{1.5em}\begin{itemize}%
      \setlength\itemindent{-\itemindent}%
      \renewcommand{\labelitemi}{$\rightarrow$}}%
     {\end{itemize}}


%%%%%%%%%%%%%%%%%%%%%%%%%%%%%%%%%%%%%%%%%%%%%%%%%%%%%%%%%%%%%%%%%%%%%%%
% New theorems
\newtheorem{theorem}{Theorem}


\makeatletter

%%%%%%%%%%%%%%%%%%%%%%%%%%%%%%%%%%%%%%%%%%%%%%%%%%%%%%%%%%%%%%%%%%%%%%%
%% New commands
\newcommand*{\course}[1]{\gdef\@course{#1}}
\newcommand*{\coursecode}[1]{\gdef\@coursecode{#1}}
\newcommand*{\term}[1]{\gdef\@term{#1}}
\newcommand*{\instructor}[1]{\gdef\@instructor{#1}}
\newcommand*{\lqnumber}[1]{\gdef\@lqnumber{#1}}
\newcommand*{\labtitle}[1]{\gdef\@labtitle{#1}}
\newcommand*{\quizversion}[1]{\gdef\@quizversion{#1}}
\newcommand*{\probleminfo}[1]{\noindent \textsl{\footnotesize #1}}

% Title header for labs
\newcommand\makelabtitle{%
  \begin{flushleft}%
  {\scshape \@coursecode~\@course~-- University of Alberta}\\%
  {\scshape \@term~-- Labs -- \@instructor}\\%
  {\scshape Authors: Carlos Contreras and Philippe Gaudreau}%
  \end{flushleft}%
  \begin{center}%
  {\Large \bf \@lqnumber:~\@labtitle}%
  \end{center}%
  \thispagestyle{empty}%
  \global\let\@course\@empty%
  \global\let\@labtitle\@empty%
}

% Title header for quizzes
\newcommand\makequiztitle{%
  \begin{flushleft}%
  {\scshape \@coursecode~\@course~-- University of Alberta}\\%
  {\scshape \@term~-- Labs -- \@instructor}\\%
  \end{flushleft}%
  \begin{center}%
  {\Large \bf \@lqnumber} \marginpar{\tiny\tt [\@quizversion]}%
  \end{center}%
  \thispagestyle{empty}%
  \global\let\@course\@empty%
  \global\let\@quizversion\@empty%
}

%%%%%%%%%%%%%%%%%%%%%%%%%%%%%%%%%%%%%%%%%%%%%%%%%%%%%%%%%%%%%%%%%%%%%%%
% Fancy header package
\fancyhead[L]{\small {\scshape \@coursecode~-- \@lqnumber~-- \@term~-- \@instructor}}
\pagestyle{fancy}

\makeatother


\usepackage{hyperref}
\usepackage{cancel}


\begin{document}


\course{Differential Equations}
\coursecode{MATH 201}
\term{Winter 2015}
\instructor{Carlos Contreras}
\lqnumber{Sample questions}
\labtitle{Midterm}
\makelabtitle

\examname{Midterm Fall 2006}

\begin{problem}
\textbf{[4 pts]} Solve:
\begin{equation*}
     \frac{dy}{dx}=\frac{3x^{2}+4x+2}{2(y-1)}, \quad y(0)=-1.
\end{equation*}
\end{problem}


\vspace{0pt}
\begin{problem}
\textbf{[5 pts]} Solve:
\begin{equation*}
     (3y^{ 3} e ^{3xy} − 1)dx + (2ye ^{3xy} + 3xy ^{2} e ^{3xy} )dy = 0, \quad y(0) = 1.
\end{equation*}
\end{problem}



\begin{problem}
\textbf{[6 pts]} An 1-kg mass is attached to a spring with stiffness 128 N/m. The damping constant for the
system is 16 N-sec/m. If the mass is moved 3/4 m to the right of equilibrium and given an
initial rightward velocity of 2 m/sec, determine the equation of motion of the mass and give
its damping factor, quasiperiod, and quasifrequency.
\end{problem}


\begin{problem}
\textbf{[7 pts]} Using the method of undetermined coefficients, find a general solution to
\begin{equation*}
y'' + 3y' − 4y = e ^{−4t} + te ^{−t}.
\end{equation*}
\end{problem}



\begin{problem}
\textbf{[7 pts]} Find a general solution to
\begin{equation*}
 y'' − 2y' + y = \frac{e^{t}}{t^{2}+1}.
\end{equation*}
\end{problem}



\begin{problem}
\textbf{[7 pts]} Find a particular solution to the variable coefficient equation
\begin{equation*}
 ty + (t − 1)y − y = −t^{ 2}
\end{equation*}
given that $y _{1} = e ^{−t}$ and $y ^{2} = t − 1$ are linearly independent solutions to the corresponding
homogeneous equation for $t > 0$.
\end{problem}



\begin{problem}
\textbf{[4 pts]} Determine the Laplace transform of the function
\begin{equation*}
f(x) = \left\{\begin{array}{ll}e^{t}, & 0<t<2 \\2. & t\geq 2 \end{array}\right. .
\end{equation*}
\end{problem}



\begin{problem}
\textbf{[6 pts]} Determine the inverse Laplace transform of the function
\begin{equation*}
F(s) = \frac{7s^{2}-23s+30}{(s+2)(s^{2}-2s+5)}.
\end{equation*}
\end{problem}



\begin{problem}
\textbf{[8 pts]} Using the method of the Laplace transforms, solve the initial value problem 
\begin{equation*}
 y'' − y' − 2y = e ^{t} , \qquad y(0) = 1, \quad y (0) = 2 .
\end{equation*} 
\end{problem}


\examname{Midterm Winter 2009}

\begin{problem}
\textbf{[4 pts]} Solve:
\begin{equation*}
 x \frac{dv}{dx} = \frac{1-4v^{2}}{3v}.
\end{equation*}
\end{problem}


\begin{problem}
\textbf{[5 pts]} Solve:
\begin{equation*}
 \frac{dy}{dx} + \frac{3}{x} y + 2 =3x, \quad y(1)=1.
\end{equation*}
\end{problem}


\begin{problem}
\textbf{[6 pts]} Solve:
\begin{equation*}
 \frac{dy}{d\theta} = \frac{\theta\sec(y/\theta)+y}{\theta}.
\end{equation*}
\end{problem}



\begin{problem}
\textbf{[10 pts]} An 2-kg mass is attached to a spring with stiffness $k = 50$N/m. The mass is displaced $1/4$ m to the left of the equilibrium point and given a velocity of 1 m/s to the left. Neglecting damping, find the equation of motion of the mass along with the amplitude, period and frequency. How long after release does the mass pass through the equilibrium position?
\end{problem}



\begin{problem}
\textbf{[7 pts]} Solve the given initial value problem:
\begin{equation*}
 y'' + 2y' + y = 1 + 2te^{ t} , \quad y(0) = y (0) = 0.
\end{equation*}
\end{problem}



\begin{problem}
\textbf{[4 pts]} Find the general solution of
\begin{equation*}
 \frac{dy}{dx} -y =e^{2x}y^{3}.
\end{equation*}
\end{problem}



\begin{problem}
\textbf{[4 pts]} Find the most general $N (x, y)$ so that the equation
\begin{equation*}
     (ye ^{xy} − 4x ^{3} y + 2)dx + N (x, y)dy = 0
\end{equation*}
is exact.
\end{problem}



\begin{problem}
\textbf{[4 pts]} Determine the Laplace transform of the function
\begin{equation*}
 f(t) = e ^{−2t} \sin 2t + e ^{3t} t ^{2}.
\end{equation*}
\end{problem}



\begin{problem}
\textbf{[6 pts]} Determine the Laplace transform of the function
\begin{equation*}
 f (t) = e^{ −t} t \sin(2t) .
\end{equation*}
\end{problem}



\begin{problem}
\textbf{[10 pts]} Find the general solution to the differential equation
\begin{equation*}
y'' + y = \tan t + e ^{3t} − 1.
\end{equation*}
\end{problem}


\examname{Midterm Winter 2010}

\begin{problem}
\textbf{[5 pts]} Find $y(x)$ such that
\begin{equation*}
 xy = 1 + 2y + x ^{3}, \quad y(1) = 1
\end{equation*}
\end{problem}



\begin{problem}
\textbf{[5 pts]} Find the general solution of the equation 
\begin{equation*}
 (2xy + 3)dx + (x ^{2} + 3y ^{2} )dy = 0.
\end{equation*}
(You may express the solution in implicit form.)
\end{problem}



\begin{problem}
\textbf{[5 pts]} Find the general solution of the following differential equations on the domain $x > 0$. For full marks, express the solution in explicit $y = f (x)$ form.
\begin{equation*}
 y' + \frac{2y}{ x} = 2x ^{2} y ^{2}.
\end{equation*}
\end{problem}



\begin{problem}
\textbf{[5 pts]} Find the general solution of the following differential equations on the domain $x > 0$. For full marks, express the solution in explicit $y = f (x)$ form.
\begin{equation*}
 x\frac{dy}{dx} = = xe ^{−y/x} + y.
\end{equation*}
\end{problem}



\begin{problem}
\textbf{[5 pts]} Find a particular solution of the following differential equation:
\begin{equation*}
 y'' (t) + 2y' (t) + y(t) = e ^{−t} .
\end{equation*}
\end{problem}



\begin{problem}
\textbf{[5 pts]} Find a particular solution of the following differential equation:
\begin{equation*}
 y'' (t) + 4y(t) = 2 \sec ^{2} (2t).
\end{equation*}
\end{problem}



\begin{problem}
\textbf{[10 pts]} Find the solution of the initial value problem
\begin{equation*}
 y'' (t) − 2y' (t) + 10y(t) = 20,\quad  y(0) = 2, y (0) = −3.
\end{equation*}
\end{problem}



\begin{problem}
\textbf{[5 pts]} Find the general solution of the differential equation 
\begin{equation*}
 x ^{2} y'' (x)+7xy' (x)+5y(x) = 0,\quad  x > 0.
\end{equation*}
\end{problem}



\begin{problem}
\textbf{[5 pts]} Find the Laplace transform $F (s)$ of the function
\begin{equation*}
 f(t) = \left\{\begin{array}{ll}t, & 0\leq t\leq 1 \\ e^{3t}, & 1\leq t<\infty \end{array}\right. .
\end{equation*}
For what domain of $s$-values is $F (s)$ defined?
\end{problem}


\examname{Midterm Fall 2011}

\begin{problem}
\textbf{[6 pts]} Find the general solution of the given differential equation. You may leave the solution in
implicit form if you wish.
\begin{equation*}
(y + x \sin x) dx + (e^{ y} + x) dy = 0 .
\end{equation*}
\end{problem}



\begin{problem}
\textbf{[6 pts]} Find the general solution of the given differential equation. You may leave the solution in
implicit form if you wish.
\begin{equation*}
 (x ^{2} + 2y ^{2}) dx − xydy = 0 .
\end{equation*}
\end{problem}



\begin{problem}
\textbf{[6 pts]} Find the general solution of the differential equation
\begin{equation*}
y''' − 3y'' + 5y' − 3y = 0.
\end{equation*}
\end{problem}



\begin{problem}
\textbf{[8 pts]} Find a particular solution of
\begin{equation*}
 y'' (t) + y(t) = t^{ 2} + \sec t \tan t .
\end{equation*}
\end{problem}



\begin{problem}
\textbf{[5 pts]} Find the Laplace transform $F (s)$ of the function
\begin{equation*}
 f(t) = \left\{\begin{array}{ll}1, & 0\leq t\leq 2 \\te^{2t}, & 2<t<\infty \end{array}\right. .
\end{equation*}
For what domain of $s$-values is $F (s)$ defined?
\end{problem}



\begin{problem}
\textbf{[5 pts]} Find the inverse Laplace transform of the function
\begin{equation*}
 F(s)=\frac{2s^{2}-2s +5}{s^{3}-2s^{2}+5s}.
\end{equation*}
\end{problem}



\begin{problem}
\textbf{[8 pts]} Find the general solution of the differential equation
\begin{equation*}
 t^{ 2} y'' (t) − 4ty' (t) + 6y(t) = t^{ 3} + 1 .
\end{equation*}
\end{problem}


\examname{Midterm Fall 2013}

\begin{problem}
\textbf{[6 pts]} (a) Use the definition of the Laplace transform to compute the Laplace transform $F (s) $ of 
\begin{equation*}
 f(t) = \left\{\begin{array}{ll}0, & 0\leq t< 1 \\\tfrac{1}{2}(e^{2t}+1), & 1\leq t<\infty \end{array}\right. .
\end{equation*}
\textbf{[3 pts]} (b) What is the domain of $F (s)$ in part (a)?
\end{problem}



\begin{problem}
\textbf{[6 pts]} Find the inverse Laplace transform o
\begin{equation*}
 F(s) = \frac{2}{(s-2)(s^{2}-s+2)}
\end{equation*}
\end{problem}



\begin{problem}
\textbf{[6 pts]} Find the solution of  
\begin{equation*}
 y' = \frac{1}{2x} y − y^{ 3} ,\quad x > 0,
\end{equation*}
such that $y(1) = − 12 $. 
\end{problem}


\begin{problem}
\textbf{[5 pts]} (i) Find all real values of a and b such that the equation
\begin{equation} \tag{\textborn} \label{equ:fa132bii}
 x^{ a} y ^{b} \left[ 2y^{ 4} e ^{−y} dx + (xy ^{3} e ^{−y^{2}} − 2xy ^{5} e ^{−y^{2}}) dy \right] = 0 
\end{equation}
\textbf{[4 pts]} (ii) When $a$ and $b$ are chosen so that equation \eqref{equ:fa132bii} is exact, find the general solution to
equation \eqref{equ:fa132bii}. You may express the solution in implicit form.
\end{problem}


\begin{problem}
\textbf{[6 pts]} Find a particular solution of 
\begin{equation*}
y'' (t) − 4y(t) = 4te ^{t}.
\end{equation*}
\end{problem}



\begin{problem}
\textbf{[5 pts]} (i) Find the general solution to the differential equation
\begin{equation*}
 x^{ 2} y'' (x) + xy' (x) − y(x) = 0 ,\quad x > 0 .
\end{equation*} 
\textbf{[4 pts]} (ii) Find a particular solution to
\begin{equation*}
x ^{2} y'' (x) + xy' (x) − y(x) = −x \ln x ,\quad x > 0 .
\end{equation*}
\end{problem}



\begin{problem}
\textbf{[3 pts]} Which of the the following equations is first-order homogeneous?
\begin{multchoice}
     \item $y'+\frac{xy}{x^{2}+y^{2}} = \tan\left( \frac{x+y}{x-y} \right)$
     \item $(x ^{2} + 2xy ^{− 3}y ^{2}) dx − (x ^{3} − x ^{2} y + 3y ^{3}) dy = 0$
     \item $\frac{dy}{dx} = \frac{2x^{2}+y^{2}e^{-y/x}}{xy-\sin(x/y)}$
     \item $y'+xy =1$
     \item All of the above are first order homogeneous.
\end{multchoice}
\end{problem}


\begin{problem}
\textbf{[3 pts]} If $\mathcal{L}{y(t)}(s) = \frac{1}{4s^{2}+3s}$ then $\mathcal{L}{e^{-t}y(t)}(s)=$
\begin{multchoice}
     \item $\frac{1}{4s^{2}+11s+7}$
     \item $\frac{1}{4s^{2}-11s+7}$
     \item $\frac{1}{4s^{2}+3s+\frac{1}{s}}$
     \item $\frac{1}{4s^{2}+3s-\frac{1}{s}}$
     \item $\frac{1}{4s^{2}-5s+1}$
\end{multchoice}
\end{problem}



\begin{problem}
\textbf{[3 pts]} Let $F(s) = \mathcal{L}{f(t)}(s)$ be the Laplace transform of $f (t)$ and let $k \neq 0$ be a constant. Then the Laplace transform of $f (kt)$ is
\begin{multchoice}
     \item $\frac{1}{k}F(s/k)$
     \item $kF(s/k)$
     \item $\frac{1}{k}F(ks)$
     \item $k F(s/k)$
     \item $F(s/k)$
\end{multchoice}
\end{problem}



\begin{problem}
\textbf{[3 pts]} Let $y''(t)+y(t) = t(1+\sin t)$. To find a particular solution using undetermined coefficients,
we must start with
\begin{multchoice}
     \item $y _{p} = At ^{2} \cos t + Bt ^{2} \sin t + Ct \cos t + Dt \sin t + Et + F$
     \item $y _{p} = At ^{2} \cos t + Bt ^{2} \sin t + Ct \cos t + Dt \sin t + Et^{2} + Ft$
     \item $y _{p} = At \cos t + Bt \sin t + C \cos t + D \sin t + Et + F$
     \item $y _{p} = At e^{ t} + B e^{t} Ct  + D$
     \item Non of the above will work.
\end{multchoice}
\end{problem}



\begin{problem}
\textbf{[3 pts]} The equation $5x'' (t) + bx' (t) + 5x(t) = 0$ describes an object of mass 5 moving in one
dimension attached to a spring with spring constant 5. Choose the correct statement:
\begin{multchoice}
     \item If $b \leq 10$, then there is at most one value of $t > 0$ for which $x(t) = 0$.
     \item If $b \leq 10$, then there is always one value of $t > 0$ for which $x(t) = 0$.
     \item If $b = 10$, then there are infinitely many values of $t > 0$ for which $x(t) = 0$.
     \item If $b > 10$, then there are infinitely many values of $t > 0$ for which $x(t) = 0$.
     \item There is no value of $b$ such that $x(t)$ has infinitely many zeroes.
\end{multchoice}
\end{problem}





%%%%%%%%%%%%%%%%%%%%%%%%%%%%%%%%%%%%%%%%%%%%%%%%%%%%%%%%%%%%%%%%%%%%%%%%%%%%%%%%%%%%%%%%%%%%%%%%%%%



% \LabSolutions






\end{document}