% MATH 201 Lab notes (c) by Carlos Contreras And Philippe Gaudreau
% MATH 201 Lab notes is licensed under a 
% Creative Commons Attribution 4.0 International license.
% CC BY 4.0

% You should have received a copy of the license along with this
% work. If not, see <http://creativecommons.org/licenses/by/4.0/>.

\documentclass[11pt]{article}
% MATH 201 Lab notes (c) by Carlos Contreras And Philippe Gaudreau
% MATH 201 Lab notes is licensed under a 
% Creative Commons Attribution 4.0 International license.
% CC BY 4.0

% You should have received a copy of the license along with this
% work. If not, see <http://creativecommons.org/licenses/by/4.0/>.

%% libraries
\usepackage[utf8x]{inputenc}
\usepackage{xcolor}
\usepackage[left=1.5cm,right=1.5cm,top=2.0cm,bottom=1.5cm,headheight=110pt]{geometry}
\usepackage{amsmath}
\usepackage{amssymb}
\usepackage{graphicx}
\usepackage{xifthen}
\usepackage{sverb}
\usepackage{fancyhdr}
\usepackage{mdframed}
\usepackage{textcomp}

%%%%%%%%%%%%%%%%%%%%%%%%%%%%%%%%%%%%%%%%%%%%%%%%%%%%%%%%%%%%%%%%%%%%%%%
% PDF compiling
\usepackage{ifpdf}
\ifpdf %
        \DeclareGraphicsExtensions{.pdf}%
\else %
        \DeclareGraphicsExtensions{.eps,.ps}%
\fi

%%%%%%%%%%%%%%%%%%%%%%%%%%%%%%%%%%%%%%%%%%%%%%%%%%%%%%%%%%%%%%%%%%%%%%%
% Figures path
\graphicspath{{figures/}}

%%%%%%%%%%%%%%%%%%%%%%%%%%%%%%%%%%%%%%%%%%%%%%%%%%%%%%%%%%%%%%%%%%%%%%%
% Problem counter
\newcounter{Problem}
\setcounter{Problem}{0}

%%%%%%%%%%%%%%%%%%%%%%%%%%%%%%%%%%%%%%%%%%%%%%%%%%%%%%%%%%%%%%%%%%%%%%%
% Definitions
\def\LabSolutions{\clearpage \newpage \begin{center} {\Large \it Solutions} \end{center} \setcounter{Problem}{0}}
\def\QuizSolutions{\newpage \begin{center} {\Large \it Solutions} \end{center} \setcounter{Problem}{0}}
\def\degree{\textdegree}
\def\grade#1{\begin{flushright} {\small [#1]}\\ \end{flushright} \vspace{-10pt}}
\def\codecolor{red!50!black}
\def\code#1{\textcolor{\codecolor}{\tt #1}}
\def\examname#1{%
                \ifnum\value{page}>1%
                    \newpage%
                \else%
                    \vspace*{5pt}%
                \fi%
                \large \textbf{#1} \setcounter{Problem}{0}\vspace{10pt}}
\def\topic#1{\par\needspace{2\baselineskip} \noindent \textsl{\footnotesize #1}}

 
%%%%%%%%%%%%%%%%%%%%%%%%%%%%%%%%%%%%%%%%%%%%%%%%%%%%%%%%%%%%%%%%%%%%%%%
% Environments
\newenvironment{problem}%
     {\stepcounter{Problem}%
      \begin{list}{\textbf{\arabic{Problem}}.~}{}%
      \item}%
     {\end{list}\vspace*{5pt}}

\newenvironment{solution}%
     {\indent \textit{Solution} \newline}%
     {\begin{flushright}$\blacksquare$\end{flushright}}

\newenvironment{preamble}%
     {\vspace*{1em}\begin{mdframed}[leftmargin=1cm,rightmargin=1cm]}%
     {\end{mdframed}\vspace*{1em}}

\newenvironment{multchoice}%
     {\begin{enumerate} \addtolength{\leftskip}{2em} \renewcommand{\labelenumi}{(\alph{enumi})}}
     {\end{enumerate}}

\newenvironment{formulaitem}%
     {\setlength{\leftmargini}{1.5em}\begin{itemize}%
      \setlength\itemindent{-\itemindent}%
      \renewcommand{\labelitemi}{$\rightarrow$}}%
     {\end{itemize}}


%%%%%%%%%%%%%%%%%%%%%%%%%%%%%%%%%%%%%%%%%%%%%%%%%%%%%%%%%%%%%%%%%%%%%%%
% New theorems
\newtheorem{theorem}{Theorem}


\makeatletter

%%%%%%%%%%%%%%%%%%%%%%%%%%%%%%%%%%%%%%%%%%%%%%%%%%%%%%%%%%%%%%%%%%%%%%%
%% New commands
\newcommand*{\course}[1]{\gdef\@course{#1}}
\newcommand*{\coursecode}[1]{\gdef\@coursecode{#1}}
\newcommand*{\term}[1]{\gdef\@term{#1}}
\newcommand*{\instructor}[1]{\gdef\@instructor{#1}}
\newcommand*{\lqnumber}[1]{\gdef\@lqnumber{#1}}
\newcommand*{\labtitle}[1]{\gdef\@labtitle{#1}}
\newcommand*{\quizversion}[1]{\gdef\@quizversion{#1}}
\newcommand*{\probleminfo}[1]{\noindent \textsl{\footnotesize #1}}

% Title header for labs
\newcommand\makelabtitle{%
  \begin{flushleft}%
  {\scshape \@coursecode~\@course~-- University of Alberta}\\%
  {\scshape \@term~-- Labs -- \@instructor}\\%
  {\scshape Authors: Carlos Contreras and Philippe Gaudreau}%
  \end{flushleft}%
  \begin{center}%
  {\Large \bf \@lqnumber:~\@labtitle}%
  \end{center}%
  \thispagestyle{empty}%
  \global\let\@course\@empty%
  \global\let\@labtitle\@empty%
}

% Title header for quizzes
\newcommand\makequiztitle{%
  \begin{flushleft}%
  {\scshape \@coursecode~\@course~-- University of Alberta}\\%
  {\scshape \@term~-- Labs -- \@instructor}\\%
  \end{flushleft}%
  \begin{center}%
  {\Large \bf \@lqnumber} \marginpar{\tiny\tt [\@quizversion]}%
  \end{center}%
  \thispagestyle{empty}%
  \global\let\@course\@empty%
  \global\let\@quizversion\@empty%
}

%%%%%%%%%%%%%%%%%%%%%%%%%%%%%%%%%%%%%%%%%%%%%%%%%%%%%%%%%%%%%%%%%%%%%%%
% Fancy header package
\fancyhead[L]{\small {\scshape \@coursecode~-- \@lqnumber~-- \@term~-- \@instructor}}
\pagestyle{fancy}

\makeatother


\usepackage{hyperref}
\usepackage{cancel}


\begin{document}


\course{Differential Equations}
\coursecode{MATH 201}
\term{Winter 2018}
\instructor{Carlos Contreras}
\lqnumber{Lab extra}
\labtitle{Final review}
\makelabtitle

\topic{Unit step function}
\begin{problem}
Determine the current as a function of time t for the given RLC series circuit. The current $I(t)$ in a RLC series circuit is governed by the initial value problem
\begin{equation*}
I^{\prime \prime}(t)+2I^{\prime}(t) +2I(t)=g(t); \quad I(0)=10, \quad I^{\prime}(0) = 0,
\end{equation*}
where,
\begin{equation*}
g(t) = \begin{cases} 
20, &\quad t<3\pi \\
0, &\quad 3\pi<t<4\pi \\
20, &\quad 4\pi<t
\end{cases}.
\end{equation*}
\end{problem}



\topic{Unit function (non-constant)}
\begin{problem}
Solve the initial value problem
\begin{equation*}
y^{\prime \prime}(t)+4y(t)=f(t); \quad y(0)=0, \quad y^{\prime}(0) = 0,
\end{equation*}
where,
\begin{equation*}
f(t) = \begin{cases} 
2t, &\quad 0\leq t<2 \\
4, &\quad 2 \leq t 
\end{cases}.
\end{equation*}
\end{problem}


\topic{Dirac delta function}
\begin{problem}
Solve the given symbolic initial value problem
\begin{equation*}
y^{\prime \prime} + 2 y^{\prime} - 3 y = \delta(t-1) - \delta(t-2) , \quad y(0) =2 , \quad y^{\prime}(0) = -2.
\end{equation*}
\end{problem}



\topic{Convolution/integro differential equation}
\begin{problem}
Solve the integro--differential equation
\[y'-2\int_{0}^{t}e^{t-v}y(v)dv=t,\quad y(0)=2.\]
\end{problem}



\topic{Laplace properties}
\begin{problem}
Find the Laplace transform of 
\[\int_{0}^{t}e^{w}cos(t-w)dw.\]
\end{problem}


\begin{problem}
Find the inverse Laplace transform of 
\[\tan^{-1} (s).\]
\end{problem}




\topic{Fourier series on $[-L, L]$.}
\begin{problem}
Compute the Fourier series for the given function on the specific interval
\begin{eqnarray*}
f(x) = x , \quad -\pi<x<\pi.
\end{eqnarray*}
\end{problem}


\topic{Fourier series on $[0, L]$.}
\begin{problem}
Compute the Fourier sine and cosine series for
\begin{equation*}
f(x) = x-x^2, \quad 0<x<1
\end{equation*}
\end{problem}



\topic{Eigenvalue problem with zero boundary conditions.}
\begin{problem}
Find nontrivial solutions to the eigenvalue problem
\begin{equation*}
y^{\prime \prime} - \lambda y =0\,; \qquad  0<x<L \,, \qquad y(0)=0  \,,\qquad y(L) =0. 
\end{equation*}
\end{problem}


\topic{Eigenvalue problem with zero derivative BC's.}
\begin{problem}
Find nontrivial solutions to the eigenvalue problem
\begin{equation*}
y^{\prime \prime} + \lambda y =0\,; \qquad  0<x<L \,, \qquad y'(0)=0  \,,\qquad y'(L) =0. 
\end{equation*}
\end{problem}

\topic{Eigenvalue problems (mixed boundary conditions)}
\begin{problem}
 Find the eigenvalues $\lambda$ for which the given problem has a nontrivial solution. Also determine the corresponding eigenfunctions.
\begin{equation*}
y^{\prime \prime} + \lambda y =0\,; \qquad  0<x<\pi \,, \qquad y(0) - y'(0)=0  \,,\qquad y(\pi) =0. 
\end{equation*}
\end{problem}



\topic{More complicated eigenvalue problem.}
\begin{problem}
 Find nontrivial solutions to the eigenvalue problem
\begin{equation*}
y^{\prime \prime} + 4 y' + \lambda y =0\,; \qquad  0<x<\pi/2 \,, \qquad y(0)=0  \,,\qquad y(\pi/2) =0. 
\end{equation*}
\end{problem}




\topic{Heat equation with zero BC's.}
\begin{problem}
Solve the heat flow problem
\begin{equation*} \begin{split}
& \dfrac{\partial u }{\partial t } (x,t) = \alpha \dfrac{\partial^2 u}{\partial x^2}(x,t), \qquad 0<x<1, \quad t>0, \\
& u(0,t) = u(1,t)=0, \qquad t>0, \\
& u(x,0) = x(1-x), \qquad 0<x<1.
\end{split}\end{equation*}
\end{problem}


\topic{Heat equation with zero derivative BC's.}
\begin{problem}
Solve the heat flow problem
\begin{equation*} \begin{split}
& \dfrac{\partial u }{\partial t } (x,t) = \alpha \dfrac{\partial^2 u}{\partial x^2}(x,t), \qquad 0<x<1, \quad t>0, \\
& \dfrac{\partial u }{\partial x } (0,t) = \dfrac{\partial u }{\partial x } (1,t)=0, \qquad t>0, \\
& u(x,0) = x(1-x), \qquad 0<x<1.
\end{split}\end{equation*}
\end{problem}


\topic{Heat equation with non-zero BC's.}
\begin{problem}
Solve the heat flow problem
\begin{equation*} \begin{split}
& \dfrac{\partial u }{\partial t } (x,t) = \dfrac{\partial^2 u}{\partial x^2}(x,t), \qquad 0<x<\pi, \quad t>0, \\
& u(0,t) = 0, \quad u(\pi,t)= 3\pi, \qquad t>0, \\
& u(x,0) = 0, \qquad 0<x<\pi.
\end{split}\end{equation*}
\end{problem}


\topic{Heat equation with non-zero BC's and external force.}
\begin{problem}
Find a formal solution to the initial value problem,
\begin{equation*} \begin{split}
& \dfrac{\partial u }{\partial t } (x,t) = \dfrac{\partial^2 u}{\partial x^2}(x,t) + 6 x -2, \qquad 0<x<1, \quad t>0, \\
& u(0,t) = 0, \quad u(1,t)= -1, \qquad t>0, \\
& u(x,0) = -x^{3}, \qquad 0<x<1.
\end{split}\end{equation*}
\end{problem}


%%%%%%%%%%%%%%%%%%%%%%%%%%%%%%%%%%%%%%%%%%%%%%%%%%%%%%%%%%%%%%%%%%%%%%%%%%%%%%%%%%%%%%%%%%%%%%%%%%%



\LabSolutions


Theory and problems from: Nagel, Saff \& Sneider, \textit{Fundamentals of Differential Equations}, Eighth Edition, Adisson--Wesley.

\vspace{20pt}


\begin{preamble}

\textbf{Formulas to remember}

\begin{formulaitem}
 
% $\rightarrow$ \textsl{Laplace transform.}
% \begin{table}[ht]
% \renewcommand{\arraystretch}{1.5}
% \setlength{\tabcolsep}{10pt}
% \begin{center}
% \begin{tabular}{|ll|ll|}
% \hline
% $f(t)$ & $F(s)=\mathcal{L}\{f\}(s)$ & $f(t)$ & $F(s)=\mathcal{L}\{f\}(s)$ \\
% \hline
% $e^{at}f(t)$ & $F(s-a)$                             &  $e^{at}$ & $\frac{1}{s-a} \qquad s>a$\\ 
% $f'(t)$ & $sF(s)-f(0)$                              &  $t^{n}$ & $\frac{n!}{s^{n+1}}$ \\
% $f''(t)$ & $s^{2}F(s)-sf(0)-f'(0)$                  &  $\sin bt$ & $\frac{b}{s^{2}+b^{2}}$ \\
% $t^{n}f(t)$ & $(-1)^{n}\frac{d^{n}}{ds^{n}}F(s)$    &  $\cos bt$ & $\frac{s}{s^{2}+b^{2}}$ \\
% $(f*g)(t)$ & $F(s)G(s)$                             &  $e^{at}t^{n}$ & $\frac{n!}{(s-a)^{n+1}} \qquad s>a$ \\
% $1$ & $\frac{1}{s}$                                 &  $e^{at}\sin bt$ & $\frac{b}{(s-a)^{2}+b^{2}} \qquad s>a$ \\
% $u(t-a)$ & $\frac{e^{-as}}{s}$                      &  $e^{at}\cos bt$ & $\frac{s-a}{(s-a)^{2}+b^{2}} \qquad s>a$ \\
% $f(t-a)u(t-a)$ & $e^{-as}F(s)$                      &  $\sinh bt$ & $\frac{b}{s^{2}-b^{2}}$ \\
% $\delta(t-a)$ & $e^{-as}$                           &  $\cosh bt$ & $\frac{s}{s^{2}-b^{2}}$ \\
% \hline 
% \end{tabular}
% \end{center}
% \end{table}
% 
% 

\item Brief \textbf{table of Laplace Transforms}.

\renewcommand{\arraystretch}{1.5}
\setlength{\tabcolsep}{10pt}
\begin{center}
\begin{tabular}{|ll|ll|}
\hline
$f(t)$ & $F(s)=\mathcal{L}\{f\}(s)$ & $f(t)$ & $F(s)=\mathcal{L}\{f\}(s)$ \\
\hline
$e^{at}f(t)$ & $F(s-a)$                                   &  $1$ & $\frac{1}{s}$ \\
$f'(t)$ & $sF(s)-f(0)$                                    &  $e^{at}$ & $\frac{1}{s-a} \qquad s>a$\\ 
$f''(t)$ & $s^{2}F(s)-sf(0)-f'(0)$                        &  $t^{n}$ & $\frac{n!}{s^{n+1}}$ \\
$t^{n}f(t)$ & $(-1)^{n}F^{(n)}(s)$                        &  $\sin bt$ & $\frac{b}{s^{2}+b^{2}}$ \\
$(f*g)(t)$ & $F(s)G(s)$                                   &  $\cos bt$ & $\frac{s}{s^{2}+b^{2}}$ \\
$u(t-a)$ & $\frac{e^{-as}}{s}$                            &  $e^{at}t^{n}$ & $\frac{n!}{(s-a)^{n+1}} \qquad s>a$ \\
$f(t-a)u(t-a)$ & $e^{-as}F(s)$                            &  $e^{at}\sin bt$ & $\frac{b}{(s-a)^{2}+b^{2}} \qquad s>a$ \\
$\delta(t-a)$ & $e^{-as}$                                 &  $e^{at}\cos bt$ & $\frac{s-a}{(s-a)^{2}+b^{2}} \qquad s>a$ \\
$\int_{0}^{s}f(\tau)d\tau$ & $\frac{1}{s}F(s)$            &  $\sinh bt$ & $\frac{b}{s^{2}-b^{2}}$ \\
$\frac{1}{t}f(t)$ & $\int_{s}^{\infty}F(\sigma)d\sigma$   &  $\cosh bt$ & $\frac{s}{s^{2}-b^{2}}$ \\
\hline 
\end{tabular}
\end{center}


\item \textsl{Unit step function.}
\begin{equation*}
u_{a}(t) = u(t-a)=\left\{\begin{array}{ll}
             0, & t<a, \\
             1, & a<t.
            \end{array} \right. 
\end{equation*}


\item \textsl{Convolution.} 
\begin{equation*}
(f*g)(t)=\int_{0}^{t}f(t-v)g(v)dv=\int_{0}^{t}f(v)g(t-v)dv.
\end{equation*}



\item \textsl{Fourier series.} $f(x)$ on $[-L, L]$
\[F(x) = \frac{a_{0}}{2}+\sum_{n=1}^{\infty}a_{n}\cos\left(\frac{n\pi x}{L}\right) + b_{n}\sin \left(\frac{n\pi x}{L}\right),\]
where
\[a_{n} = \frac{1}{L}\int_{-L}^{L}f(x)\cos \left(\frac{n\pi x}{L}\right) dx, \qquad b_{n}  = \frac{1}{L}\int_{-L}^{L}f(x)\sin \left(\frac{n\pi x}{L}\right) dx.\]

This sum converges to
\begin{equation*}
F(x) = \begin{cases} f(x) & \text{if $-L<x<L$ and $f$ is continuous at $x$,} \\[0.5em]
                                 \dfrac{f(x-)+f(x+)}{2} & \text{if $-L<x<L$ and $f$ is discontinuous at $x$,} \\[0.5em]
                                 \dfrac{f(-L+)+f(L-)}{2} & \text{if $x=L$ or $x=-L$.} \end{cases} 
\end{equation*}

\item \textsl{Fourier cosine series.} $f(x)$ on $[0, L]$ (even extension)
\[C(x) = \frac{a_{0}}{2}+\sum_{n=1}^{\infty}a_{n}\cos\left(\frac{n\pi x}{L}\right), \quad a_{n} = \frac{2}{L}\int_{0}^{L}f(x)\cos \left(\frac{n\pi x}{L}\right) dx.\]

\item \textsl{Fourier sine series.} $f(x)$ on $[0, L]$ (odd extension)
\[S(x) = \sum_{n=1}^{\infty}b_{n}\sin\left(\frac{n\pi x}{L}\right), \quad b_{n} = \frac{2}{L}\int_{0}^{L}f(x)\sin \left(\frac{n\pi x}{L}\right) dx.\]

\item \textsl{Eigenvalue problems.} 
\[\frac{X''}{X}=\lambda, \,\, X(0)=X(L)=0, \quad \Rightarrow \quad \boxed{\lambda_{n}=-\frac{n^{2}\pi^{2}}{L^{2}}, \,\, X_{n}(x)=\sin \left( \frac{n\pi x}{L} \right)}, \,\, n\geq1.\]
\[\frac{X''}{X}=\lambda, \,\, X'(0)=X'(L)=0, \quad \Rightarrow \quad \boxed{\lambda_{n}=-\frac{n^{2}\pi^{2}}{L^{2}}, \,\, X_{n}(x)=\cos \left( \frac{n\pi x}{L} \right)},\,\, n\geq0.\]
On the other hand, if we use $\frac{X''}{X}=-\lambda$ instead, the eigenvalues are $\lambda_{n}=\frac{n^{2}\pi^{2}}{L^{2}}$, and the eigenfunctions remain the same.

\item For the heat equation with \textbf{non-zero boundary conditions and external force}
\begin{equation*} \begin{split}
& \dfrac{\partial u }{\partial t } (x,t) = \alpha\dfrac{\partial^2 u}{\partial x^2}(x,t) + g(x), \qquad 0<x<L, \quad t>0, \\
& u(0,t) = U_{1}, \quad u(L,t)= U_{2}, \qquad t>0, \\
& u(x,0) = f(x), \qquad 0<x<L,
\end{split}\end{equation*}
we apply the change of variable
\[u(x,t)=v(x)+w(x,t), \,\, \Rightarrow w(x,t)=u(x, t)-v(x),\]
to arrive the zero BC's problem
\begin{equation*} \begin{split}
& \dfrac{\partial w }{\partial t } (x,t) = \alpha\dfrac{\partial^2 w}{\partial x^2}(x,t), \qquad 0<x<L, \quad t>0, \\
& w(0,t) = w(L,t)= 0, \qquad t>0, \\
& w(x,0) = f(x) - v(x), \qquad 0<x<L,
\end{split}\end{equation*}
and the second order problem
\begin{equation*} \begin{split}
& v''(x) = -\tfrac{1}{\alpha}g(x) \qquad 0<x<L, \\
& v(0) = U_{1}, \quad v(L)= U_{2}.
\end{split}\end{equation*}
We first solve for $v(x)$, then solve $w(x,t)$, and finally write the solution in terms of $u(x,t)$.

If $g(x)=0$, then, clearly
\[v(x)=(U_{2}-U_{1})\frac{x}{L}+U_{1}.\]

The functions $w(x, t)$ and $v(x)$ are called \textbf{transient} and \textbf{steady state} solutions of $u(x,t)$.
\end{formulaitem}
\end{preamble}







\begin{problem}
Determine the current as a function of time t for the given RLC series circuit. Plot the solution. The current $I(t)$ in a RLC series circuit is governed by the initial value problem
\begin{equation*}
I^{\prime \prime}(t)+2I^{\prime}(t) +2I(t)=g(t); \quad I(0)=10, \quad I^{\prime}(0) = 0,
\end{equation*}
where,
\begin{equation*}
g(t) = \begin{cases} 
20, &\quad t<3\pi \\
0, &\quad 3\pi<t<4\pi \\
20, &\quad 4\pi<t
\end{cases}.
\end{equation*}
\end{problem}
\begin{solution}
We can rewrite the function $g(t)$ using unit step function as follows:

\begin{equation*}
g(t) = 20-20 u(t-3\pi) +20 u(t-4\pi) = 20(1-u(t-3\pi) + u(t-4\pi))
\end{equation*}

Hence our IVP can be rewritten as:

\begin{equation*}
I^{\prime \prime}(t)+2I^{\prime}(t) +2I(t)=20(1-u(t-3\pi) + u(t-4\pi)); \quad I(0)=10, \quad I^{\prime}(0) = 0,
\end{equation*}

Taking the Laplace transform on both sides of this equation, we obtain:
\begin{equation*}
{\cal L}^{}\{ I^{\prime \prime}(t) \} + 2{\cal L}^{}\{ I^{\prime }(t) \}+2 {\cal L}^{}\{ I(t) \} =20 ( {\cal L}^{}\{ 1 \}- {\cal L}^{}\{ u(t-3\pi) \} + {\cal L}^{}\{ u(t-4\pi) \})
\end{equation*}
Expanding, we obtain:

\begin{eqnarray*}
(s^2J(s)-sI(0)-I^{\prime}(0)) + 2 ( sJ(s)-I(0))+2 J(s) & = & 20 \left( \dfrac{1}{s}- \dfrac{e^{-3\pi s}}{s} + \dfrac{e^{-4\pi s}}{s}\right) \\
s^2J(s)-10s + 2sJ(s)-20+2 J(s) & = & \dfrac{20}{s}- \dfrac{20e^{-3\pi s}}{s} + \dfrac{20e^{-4\pi s}}{s} \\
\end{eqnarray*}
Isolating for $J(s)$, we obtain:
\begin{eqnarray*}
J(s)  =   \dfrac{20}{s(s^2+2s+2)}- \dfrac{20e^{-3\pi s}}{s(s^2+2s+2)} + \dfrac{20e^{-4\pi s}}{s(s^2+2s+2)} + \dfrac{10s+20}{s^2+2s+2}\\
\end{eqnarray*}
We need to find the partial fraction decomposition of  $F(s)=\dfrac{20}{s(s^2+2s+2)}$.

\begin{eqnarray*}
F(s)=\dfrac{20}{s(s^2+2s+2)} & = & \dfrac{As+B}{s^2+2s+2} + \dfrac{C}{s} \\
 & = & \dfrac{(As+B)s+ C(s^2+2s+2)}{s(s^2+2s+2)} \\
 & = & \dfrac{s^2(A+C) + s(B+2C) + 2C }{s(s^2+2s+2)}
\end{eqnarray*}
This leads to the following system of equations:

\begin{eqnarray*}
A+C &= &0\\
B+2C & = & 0\\
2C & = & 20
\end{eqnarray*}
This can easily be solve. $A=-10$,$B=-20$ and $C=10$
Hence:

\begin{eqnarray*}
J(s)  & = &   -\dfrac{10s+20}{s^2+2s+2} + \dfrac{10}{s} -F(s)e^{-3\pi s} +F(s)e^{-4\pi s} + \dfrac{10s+20}{s^2+2s+2} \\
& = &    \dfrac{10}{s} -F(s)e^{-3\pi s} +F(s)e^{-4\pi s} \\
\end{eqnarray*}

The inverse Laplace transform of $F(s)$ is given by:

\begin{eqnarray*}
f(t) = {\cal L}^{-1}\{ F(s) \} & = & {\cal L}^{-1}\left\{ -\dfrac{10s+20}{s^2+2s+2} + \dfrac{10}{s} \right\} \\
& = & {\cal L}^{-1}\left\{ \dfrac{-10(s+1)-10}{(s+1)^2+1}\right\} +10 {\cal L}^{-1}\left\{ \dfrac{1}{s}\right\}\\
& = & -10{\cal L}^{-1}\left\{ \dfrac{s+1}{(s+1)^2+1}\right\} -10 {\cal L}^{-1}\left\{ \dfrac{1}{(s+1)^2+1}\right\} +10 {\cal L}^{-1}\left\{ \dfrac{1}{s}\right\}\\
&=& -10e^{-t}\cos(t)-10e^{-t}\sin(t) +10
\end{eqnarray*}
We can now find the function $I(t)$ via the property $ {\cal L}^{-1}\{ e^{-as}F(s) \}(t) = f(t-a)u(t-a)$. Taking the inverse Laplace transform of our function $J(s)$, we obtain:

\begin{eqnarray*}
I(t) & = & {\cal L}^{-1}\{ J(s) \}  \\
&=& 10 {\cal L}^{-1}\left\{ \dfrac{1}{s}\right\}  -{\cal L}^{-1}\left\{ e^{-3\pi s} F(s)\right\} + {\cal L}^{-1}\left\{ e^{-4\pi s} F(s)\right\} \\
& = & 10 - \left(-10e^{-(t-3\pi)}\cos(t-3\pi)-10e^{-(t-3\pi)}\sin(t-3\pi) +10\right)u(t-3\pi) \\
& & + \left(-10e^{-(t-4\pi)}\cos(t-4\pi)-10e^{-(t-4\pi)}\sin(t-4\pi) +10\right)u(t-4\pi) \\
& = & 10 - \left(10e^{-(t-3\pi)}\cos(t)+10e^{-(t-3\pi)}\sin(t) +10\right)u(t-3\pi) \\
& & + \left(-10e^{-(t-4\pi)}\cos(t)-10e^{-(t-4\pi)}\sin(t) +10\right)u(t-4\pi) 
\end{eqnarray*} 
Hence,
\begin{gather*}
\boxed{I(t) = 10 - 10 u(t-3\pi) \left[ 1+e^{-(t-3\pi )} \left(\cos t+\sin t\right)\right] + 10 u(t-4\pi) \left[1-e^{-(t-4\pi)}(\cos t+\sin t)\right]}.
\end{gather*}

\end{solution}





\begin{problem}
Solve the initial value problem
\begin{equation*}
y^{\prime \prime}(t)+4y(t)=f(t); \quad y(0)=0, \quad y^{\prime}(0) = 0,
\end{equation*}
where,
\begin{equation*}
f(t) = \begin{cases} 
2t, &\quad 0\leq t<2 \\
4, &\quad 2 \leq t 
\end{cases}.
\end{equation*}
\end{problem}

\begin{solution}
The first step is to write $f(t)$ in terms of unit step functions
\[f(t)=2t+ (4 - 2t)u(t-2).\]
Note that $h(t)=4 -2t$ is not of the form $f(t-2)$ so using the property for $f(t-a)u(t-a)$ is not straight forward. A trick that will always work is to evalute $h(t+2)$\footnote{For the property $\mathcal{L}\{f(t-a)u(t-a)\}=F(s)e^{-as}$, we need the Laplace of $f(t)$, not the Laplce of $f(t-a)$. If we have $h(t)u(t-2)$, let $f(t-a)=h(t)$, then $f(t)=h(t+a)$. So $\mathcal{L}\{h(t)u(t-a)\}=\mathcal{L}\{f(t-a)u(t-a)\}=\mathcal{L}\{f(t)\}e^{-as}=\mathcal{L}\{h(t+a)\}e^{-as}$.},
\[h(t+2)=4-2(t+2)=-2t,\quad  \Rightarrow \quad \mathcal{L}\{h(t+2)\}=-\frac{2}{s^2},\]
and multiply by $e^{-2s}$
\[\mathcal{L}\{(4-2t)u(t-2)\} = -\frac{2}{s^2}e^{-2s}.\]
Then,
\[F(s)= \mathcal{L}\{2t\} + \mathcal{L}\{(4-2t)u(t-2)\}= \frac{2}{s^2} - \frac{2}{s^{2}}e^{-2s}=2\left( \frac{1}{s^2} - \frac{e^{-2s}}{s^{2}} \right)\]

Alternatively, we can write 
\[f(t)=2t(u(t)-u(t-2))+4u(t-2),\]
then, its Laplace transform (using the property for $tf(t)$) is, as expected,
\[F(s)=2(-1)\frac{d}{ds}\left[ \frac{1}{s} - \frac{e^{-2s}}{s} \right] + 4 \frac{e^{-2s}}{s} = 2 \left( \frac{1}{s^{2}} - \frac{e^{-2s}}{s^{2}} \right).\]


Next step is to apply Laplace of both side of the ODE
\[s^{2}Y(s) -0s-0+ 4 Y(s)=2 \left( \frac{1}{s^{2}} - \frac{e^{-2s}}{s^{2}} \right).\]
Isolating $Y(s)$ we have
\[Y(s)=\frac{2}{s^{2}(s^{2}+4)}- \frac{2}{s^{2}(s^{2}+4)} e^{-2s}=G(s) -G(s)e^{-2s}.\]
Is practical to take $e^{-as}$ terms as a common factor, since this only shift the inverse Laplace of $G(s)$.

Now we take the inverse Laplace transform of $G(s)$ using partial fractions
\[G(s)=\frac{2}{s^{2}(s^{2}+4)}=\frac{As+B}{s^{2}}+\frac{Cs+D}{s^{2}+4}= \frac{1}{2s^{2}}-\frac{1}{2(s^{2}+4)}.\]

Hence,
\[g(t)=\frac{1}{2}t-\frac{1}{4}\sin 2t\]

Finally,
\[y(t)=g(t)-g(t-2)u(t-2)\]
\[\Rightarrow \boxed{y(t)=\frac{t}{2}-\frac{\sin 2t}{4}- \left(\frac{t-2}{2}-\frac{\sin 2(t-2)}{4}\right)u(t-2)}.\]
\end{solution}





\begin{problem}
Solve the given symbolic initial value problem
\begin{equation*}
y^{\prime \prime} + 2 y^{\prime} - 3 y = \delta(t-1) - \delta(t-2) , \quad y(0) =2 , \quad y^{\prime}(0) = -2.
\end{equation*}
\end{problem}
\begin{solution}
Applying a Laplace transform on both sides, we obtain:

\begin{equation*}
s^2 Y(s) -sy(0) - y^{\prime}(0) + 2sY(s) - 2y(0) - 3 Y(s) = e^{-s} - e^{-2s}
\end{equation*}
Subbing in our initial conditions, we obtain:

\begin{equation*}
s^2 Y(s) -2s +2 + 2sY(s) - 4 - 3 Y(s) = e^{-s} - e^{-2s}
\end{equation*}
Isolating $Y(s)$, we obtain:

\begin{eqnarray*}
Y(s) & = & \dfrac{2s+2}{s^2+2s-3}+ \dfrac{e^{-s}}{s^2+2s-3} - \dfrac{e^{-2s}}{s^2+2s-3}  \\
& = & \dfrac{2s+2}{(s+3)(s-1)}+\dfrac{e^{-s}}{(s+3)(s-1)} - \dfrac{e^{-2s}}{(s+3)(s-1)}  \\
\end{eqnarray*}
Decomposing these functions into partial fractions,we obtain:
\begin{eqnarray*}
Y(s) & = & \left( \dfrac{1}{s-1} + \dfrac{1}{s+3} \right)+ \dfrac{1}{4}\left( \dfrac{1}{s-1} - \dfrac{1}{s+3} \right) e^{-s}  -\dfrac{1}{4} \left( \dfrac{1}{s-1} - \dfrac{1}{s+3} \right) e^{-2s} \\
\end{eqnarray*}
Taking the inverse Laplace transform using the identity ${\cal L} \left\{ G(s)e^{-as} \right\}(t) = g(t-a)u(t-a)$, we have
\begin{equation*}
\boxed{y(t) = e^{t}+e^{-3t}   + \tfrac{1}{4} \left( e^{t-1}-e^{-3(t-1)} \right)u(t-1) - \tfrac{1}{4} \left( e^{t-2}-e^{-3(t-2)} \right)u(t-2)}.
\end{equation*}
\end{solution}






\begin{problem}
Solve the integro--differential equation
\[y'-2\int_{0}^{t}e^{t-v}y(v)dv=t,\quad y(0)=2.\]
\end{problem}
\begin{solution}
We can rewrite the integro--differential as
\[y'-2e^{t}*y=t,\]
and take Laplace transform to get (recall the convolution theorem here)
\[sY(s)-2-2\frac{1}{s-1}Y(s)=\frac{1}{s^{2}},\]
where
\[Y(s)=\frac{(2s^{2}+1)(s-1)}{s^{2}(s+1)(s-2)}=\frac{2s^{3}-2s^{2}+s-1}{s^{2}(s+1)(s-2)}=\frac{A}{s}+\frac{B}{s^{2}}+\frac{C}{s+1}+\frac{D}{s-2}.\]
Solving the partial fractions we get $A=-\frac{3}{4}$, $B=\frac{1}{2}$, $C=2$ and $D=\frac{3}{4}$.
Thus, we can take the inverse Laplace transforms right away to get
\[\boxed{y(t)=-\frac{3}{4}+\frac{1}{2}t+2e^{-t}+\frac{3}{4}e^{2t}}.\]
\end{solution}
\begin{solution}
We can rewrite the equation as
\[y'-2e^{t}\int_{0}^{t}e^{-v}y(v)dv=t,\]
take the derivative, and apply the product rule along with the fundamental theorem of calculus to get
\[y''-2e^{t}\int_{0}^{t}e^{-v}y(v)dv-2e^{t}e^{-t}y(t)=1.\]
Note that
\[-2e^{t}\int_{0}^{t}e^{-v}y(v)dv=t-y',\]
then we obtain the second order differential equation
\[y''-y'-2y=1-t,\]
with initial conditions (the second initial condition comes from evaluating the integro--differential equation at $t=0$)
\[y(0)=2,\quad y'(0)=0.\]
Apply Laplace transform we have
\[Y(s)=\frac{2s-2}{s^{2}-s-2}+\frac{1}{s(s^{2}-s-2)}-\frac{1}{s^{2}(s^{2}-s-2)}=\frac{2s^{3}-2s^{2}+s-1}{s^{2}(s+1)(s-2)},\]
which is the same partial fractions problem as before. Thus
\[\boxed{y(t)=-\frac{3}{4}+\frac{1}{2}t+2e^{-t}+\frac{3}{4}e^{2t}}.\]
\end{solution}



\begin{problem}
Find the Laplace transform of 
\[\int_{0}^{t}e^{w}\cos(t-w)dw.\]
\end{problem}
\begin{solution}
Since
\[\int_{0}^{t}e^{w}\cos(t-w)dw = e^{t}*\cos(t),\]
then,
\[\mathcal{L}\left\{\int_{0}^{t}e^{w}cos(t-w)dw\right\} = \mathcal{L}\left\{e^{t}\right\} \mathcal{L}\left\{\cos(t)\right\} = \frac{1}{s-1}\cdot\frac{2}{s^{2}+1}.\]
\end{solution}



\begin{problem}
Find the inverse Laplace transform of 
\[\tan^{-1} (s).\]
\end{problem}
\begin{solution}
In problems of inverse Laplace transform of trigonometric or trascendental functions that are not in the table, usually one of its derivatives is a fraction. In this case
\[(\tan^{-1}s)'=\frac{1}{1+s^{2}},\]
which looks like the Laplace transform of $\sinh$. For this problem we will use the property
\[\mathcal{L}\{t^{n}f(t)\}=(-1)^{n}F^{(n)}(s).\]
Let $F(s)=\tan^{-1}(s)$, then
\[-F'(s)=-\frac{1}{1+s^{2}}=-\frac{1}{s^{2}+1}.\]
Using the property
\[tf(t) =\mathcal{L}^{-1}\left\{-\frac{1}{s^{2}+1}\right\}=-\sin t.\]
Isolating $f(t)$, we have
\[\boxed{f(t)=-\frac{\sin t}{t}}\]
\end{solution}




\begin{problem}
Compute the Fourier series for the given function on the specific interval
\begin{eqnarray*}
f(x) = x , \quad -\pi<x<\pi.
\end{eqnarray*}
\end{problem}
\begin{solution}
First, note that $f(x)$ is \textsl{odd}, and $L=\pi$.
We want to write $f(x)$ in the form
\begin{eqnarray*}
f(x) = \dfrac{a_{0}}{2} + \sum_{n=1}^{\infty} \left[ a_{n}\cos\left(n x\right)+b_{n}\sin\left(n x\right) \right]
\end{eqnarray*}
Since $f(x)$ is an odd function, we know that
\begin{eqnarray*}
a_{n} = \dfrac{1}{\pi} \int_{-\pi}^{\pi} x\cos\left(n x\right) dx =0, \quad {\rm for} \quad n =1,2,3,\ldots,
\end{eqnarray*}
and
\begin{eqnarray*}
a_{0} = \dfrac{1}{\pi} \int_{-\pi}^{\pi} x dx =0.
\end{eqnarray*}
For $b_{n}$ we use one integration by parts.
\begin{align*}
b_{n}& = \frac{1}{\pi}\int_{-\pi}^{\pi} x\sin\left(n x\right) dx = \frac{2}{\pi}\int_{0}^{\pi} x\sin\left(n x\right) dx \\
     & = \frac{2}{\pi}\left[ -\frac{x}{n}\cos (nx) |^{\pi}_{0} + \frac{1}{n}\int_{0}^{\pi} \cos\left(n x\right) dx\right] \\
     & = \frac{2}{\pi}\left[ -\frac{\pi}{n}\cos (n\pi)  + \frac{1}{n^{2}}\sin (nx)|^{\pi}_{0} \right] = \frac{(-1)^{n+1}2}{n}.
\end{align*}
Thus,
\[\boxed{f(x) = \sum_{n=1}^{\infty}\frac{(-1)^{n+1}2}{n}\sin(nx)}.\]
\end{solution}



\begin{problem}
Compute the Fourier sine and cosine series for 
\begin{equation*}
f(x) = x-x^2, \quad 0<x<1.
\end{equation*}
\end{problem}

\begin{solution}
\textbf{Note:} the given domain is of the form $[0, L]$ instead of $[-L, L]$. In which case the problem is either to find the Fourier Sine series (also called odd extension of $f(x)$ to $[-L, L]$) or the Fourier Cosine series (also called even extension of $f(x)$ to $[-L, L]$).

1. The Fourier Sine Series of $f(x)$ on $[0,L]$ is

\begin{equation*}
f(x) = \sum_{n=1}^{\infty} b_{n} \sin \left(\dfrac{n\pi x}{L}\right)
\end{equation*}

where 

\begin{equation*}
b_{n} = \dfrac{2}{L} \int_{0}^{L} f(x) \sin \left( \dfrac{n \pi x}{L} \right) {\rm d} x, \quad n=1,2,3,\ldots
\end{equation*}

In our case, we have $L=1$. Let's compute $b_{n}$ (integration by parts is the most common technique here).

\begin{eqnarray*}
b_{n} & = & 2 \int_{0}^{1} (x-x^2) \sin \left( n \pi x\right) {\rm d} x \\
 & = & 2 \left[(x-x^2)\left(- \dfrac{\cos(n \pi x)}{n \pi} \right) \right]_{0}^{1} -2  \int_{0}^{1} (1-2x) \left(- \dfrac{\cos(n \pi x)}{n \pi} \right) {\rm d} x \\
  & = &0 + \dfrac{2}{n \pi}  \int_{0}^{1} (1-2x)  \cos(n \pi x) {\rm d} x \\
& = & \dfrac{2}{n \pi} \left[  (1-2x)\left( \dfrac{\sin(n \pi x)}{n \pi} \right) \right]_{0}^{1} - \dfrac{2}{n \pi} \int_{0}^{1} (-2)  \left( \dfrac{\sin(n \pi x)}{n \pi} \right) {\rm d} x \\
& = & 0  + \dfrac{4}{n^2 \pi^2} \int_{0}^{1}   \sin(n \pi x) {\rm d} x  = \dfrac{4}{n^2 \pi^2} \left[- \dfrac{\cos(n \pi x)}{n \pi} \right]_{0}^{1} \\
& = &\dfrac{4}{n^2 \pi^2} \left( \dfrac{1-\cos(n \pi)}{n \pi} \right) = \dfrac{4(1-(-1)^n)}{n^3 \pi^3}.
\end{eqnarray*}

Hence, our function $f(x) = x - x^2$  on the interval $ [0,1]$ can be written in the following Fourier Sine series

\begin{equation*}
\boxed{f(x) = \sum_{n=1}^{\infty} \dfrac{4(1-(-1)^n)}{n^3 \pi^3} \sin( n\pi x)}. 
\end{equation*}


\textbf{Note.} The previous is a valid answer, do the following only if you need to \textsl{need} to. The $(-1)^{n}$ term suggests that we can simplify the coefficients, since $b_{n}=0$ for even $n$. That is,

\begin{eqnarray*}
b_{n} & = & \dfrac{4(1-(-1)^n)}{n^3 \pi^3} \\
\Rightarrow b_{2n}& = & \dfrac{4(1-(-1)^{2n})}{(2n)^3 \pi^3} = \dfrac{4(1-1)}{(2n)^3 \pi^3} =0 \\
\Rightarrow b_{2n-1}& = & \dfrac{4(1-(-1)^{2n-1})}{(2n-1)^3 \pi^3} = \dfrac{4(1-(-1))}{(2n-1)^3 \pi^3} = \dfrac{8}{(2n-1)^3 \pi^3} \\
\end{eqnarray*}

Hence, we can write

\begin{equation*}
f(x) = \sum_{n=1}^{\infty} \dfrac{8}{(2n-1)^3 \pi^3}\sin( (2n-1)\pi x). 
\end{equation*}

2. The Fourier Cosine series of $f(x)$ on $[0, L]$ is
\begin{equation*}
f(x) = \frac{a_{0}}{2}+\sum_{n=1}^{\infty} a_{n} \cos \left(\dfrac{n\pi x}{L}\right)
\end{equation*}
where
\[a_{0}=\frac{2}{L}\int_{0}^{L}f(x)dx = \frac{1}{3}\]
and
\begin{equation*}
a_{n}=\frac{2}{L}\int_{0}^{L}f(x)\cos \left(\dfrac{n\pi x}{L}\right) dx =\cdots = \frac{2((-1)^{n+1}-1)}{n^{2}\pi^{2}}.
\end{equation*}
Hence, our function $f(x) = x - x^2$  on the interval $[0,1]$ can be written in the following Fourier cosine series
\begin{equation*}
\boxed{f(x) = \frac{1}{6}+\sum_{n=1}^{\infty} \frac{2((-1)^{n+1}-1)}{n^{2}\pi^{2}} \cos \left({n\pi x}\right)}.
\end{equation*}

\textbf{Note.} As before that is a valid answer, if you \textsl{need} to simplify, separate odd and even cases.

\[a_{2n}=-\frac{1}{n^{2}\pi^{2}}, \qquad a_{2n-1}=0.\]

Hence, we can write

\begin{equation*}
f(x) = \frac{1}{6}-\sum_{n=1}^{\infty} \frac{1}{n^{2}\pi^{2}} \cos \left({2n\pi x}\right).
\end{equation*}
\end{solution}







\begin{problem}
Find nontrivial solutions to the eigenvalue problem
\begin{equation*}
y^{\prime \prime} - \lambda y =0\,; \qquad  0<x<L \,, \qquad y(0)=0  \,,\qquad y(L) =0. 
\end{equation*}
\end{problem}

\begin{solution}
\textbf{Note.} \textsl{When you are solving an eigenvalue problem within a heat or wave equation problem \textbf{do not} go through all this cases. Do only the nontrivial solutions case, or use the known solution right away.}

First we find the roots of the auxiliary equation.
\[r=\pm\sqrt{\lambda}=\pm\sqrt{\Delta}.\]
We consider the three cases.\\

\par \textsl{Case 1.} $\Delta = \lambda >0.$ Then $r_{1}=\sqrt{\lambda}$, $r_{2}=-\sqrt{\lambda}$ are distinct real roots, and the solution to ODE is
\[y(x)=C_{1}e^{\sqrt{\lambda}x}+C_{2}e^{-\sqrt{\lambda}x}.\]
Using the BC's,
\begin{equation*}
\left\{\begin{array}{rrcc}
       C_{1} &+C_{2}&=&0\\
       e^{\sqrt{\lambda}L}C_{1} &+ e^{-\sqrt{\lambda}L}C_{2}&=&0
      \end{array}\right. .
\end{equation*}
For nontrivial solutions we need 
$$\det(A)=\left|\begin{matrix}1&1 \\ e^{\sqrt{\lambda}L} & e^{-\sqrt{\lambda}L}\end{matrix}\right|=e^{-\sqrt{\lambda}L}-e^{\sqrt{\lambda}\pi}=0,$$ 
which implies $e^{\sqrt{\lambda}L}=e^{-\sqrt{\lambda}L}$, and this is a contradiction since the exponential function is always positive and $\lambda\neq 0$. Hence, \textsl{there is no nontrivial solution}.

\par \textsl{Case 2.} $\Delta = \lambda =0.$ Then $r_{1}=r_{2}=0$ are repeated real roots, and the solution to ODE is
\[y(x)=C_{1}x+C_{2}.\]
Using the BC's,
\begin{equation*}
\left\{\begin{array}{rcc}
       C_{2} & = &0\\
       LC_{1} + C_{2} & = &0
      \end{array}\right. .
\end{equation*}
Hence, \textsl{there is no nontrivial solution}.

\par \textsl{Case 3.} $\Delta = \lambda <0.$ Then $r_{1}=i\sqrt{-\lambda}$, $r_{2}=-i\sqrt{-\lambda}$ are complex roots, and the solution to ODE is
\[y(x)=C_{1}\cos\sqrt{-\lambda}x+C_{2}\sin\sqrt{-\lambda}x.\]
Using the BC's,
\begin{equation*}
\left\{\begin{array}{rrcc}
       C_{1} & & = & 0\\
       \cos(\sqrt{-\lambda}L)C_{1} &+ \sin(\sqrt{-\lambda}L)C_{2}&=&0
      \end{array}\right. ,
\end{equation*}
which implies 
\[\sin(\sqrt{-\lambda}L)C_{2}=0 \,\Leftrightarrow\, C_{2} = 0 \text{ or } \sin(\sqrt{-\lambda}L)=0.\]
For nontrivial solutions we need 
\[\sin(\sqrt{-\lambda}L)=0 \,\Leftrightarrow \, \sqrt{-\lambda}L=n\pi.\] 
Thus, the eigenvalues are
\[\boxed{\lambda_{n}=-\frac{n^{2}\pi^{2}}{L^{2}}, \quad n\geq 1},\]
with eigenfunctions
\[\boxed{y_{n}(x)=C_{n}\sin\left( \frac{n\pi x}{L} \right)}, \quad n\geq 1,\]
for some arbitrary constants $C_{n}$.

\noindent \textbf{Note.} $y''-\lambda y = 0$ comes from $\frac{y''}{y}=\lambda$ when solving the heat or wave equation.

\noindent \textbf{Note.} $y'' + \lambda y = 0$ only changes the eigenvalue to $\lambda_{n}=\frac{n^{2}\pi^{2}}{L^{2}}$ in the solution.

\end{solution}








\begin{problem}
Find nontrivial solutions to the eigenvalue problem
\begin{equation*}
y^{\prime \prime} + \lambda y =0\,; \qquad  0<x<L \,, \qquad y'(0)=0  \,,\qquad y'(L) =0. 
\end{equation*}
\end{problem}
\begin{solution}
First we find the roots of the auxiliary equation.
\[r=\pm\sqrt{-\lambda}=\pm\sqrt{\Delta}.\]
We consider the three cases.\\

\par \textsl{Case 1.} $\Delta = -\lambda >0 \,\, \Rightarrow \, \lambda<0.$ Then $r_{1}=\sqrt{-\lambda}$, $r_{2}=-\sqrt{-\lambda}$ are distinct real roots, and the solution to ODE is
\[y(x)=C_{1}e^{\sqrt{-\lambda}x}+C_{2}e^{-\sqrt{-\lambda}x}.\]
We need $y'(x)$ in order to use the initial conditions
\[y'(x)=C_{1}\sqrt{-\lambda}e^{\sqrt{-\lambda}x}-C_{2}\sqrt{-\lambda}e^{-\sqrt{-\lambda}x}.\]
Using the BC's,
\begin{equation*}
\left\{\begin{array}{rrcc}
       \sqrt{-\lambda}C_{1} & - \sqrt{-\lambda}C_{2}&=&0\\
       \sqrt{-\lambda}e^{\sqrt{-\lambda}L}C_{1} &- \sqrt{-\lambda}e^{-\sqrt{-\lambda}L}C_{2}&=&0
      \end{array}\right. .
\end{equation*}
For nontrivial solutions we need 
$$\det(A)=\left|\begin{matrix} \sqrt{-\lambda} & - \sqrt{-\lambda}\\ \sqrt{-\lambda}e^{\sqrt{-\lambda}L} & -\sqrt{-\lambda}e^{-\sqrt{-\lambda}L}\end{matrix}\right|=\lambda e^{-\sqrt{-\lambda}L} + \lambda e^{\sqrt{-\lambda}L}=0,$$ 
which implies $e^{\sqrt{-\lambda}L}=-e^{-\sqrt{-\lambda}L}$, which is a contradiction since the exponential function is always positive and $\lambda\neq 0$. Hence, \textsl{there is no nontrivial solution}.

\par \textsl{Case 2.} $\Delta = -\lambda =0 \,\, \Rightarrow \, \lambda=0.$ Then $r_{1}=r_{2}=0$ are repeated real roots, and the solution to ODE is
\[y(x)=C_{1}x+C_{2}.\]
We need $y'(x)$ in order to use the initial conditions
\[y'(x)=C_{1}.\]
Using the BC's,
\begin{equation*}
\left\{\begin{array}{rcc}
       C_{1} & = &0\\
       C_{1} & = &0
      \end{array}\right. .
\end{equation*}
Hence, \[\boxed{\lambda=0}\] is an eigenvalue with eigenfunction \[\boxed{y(x)=C},\]
for some constant $C$.

\par \textsl{Case 3.} $\Delta = -\lambda <0 \,\, \Rightarrow \, \lambda>0.$ Then $r_{1}=i\sqrt{\lambda}$, $r_{2}=-i\sqrt{\lambda}$ are complex roots, and the solution to ODE is
\[y(x)=C_{1}\cos\sqrt{\lambda}x+C_{2}\sin\sqrt{\lambda}x.\]
We need $y'(x)$ in order to use the initial conditions
\[y'(x)=-\sqrt{\lambda}C_{1}\sin\sqrt{\lambda}x+\sqrt{\lambda}C_{2}\cos\sqrt{\lambda}x.\]
Using the BC's,
\begin{equation*}
\left\{\begin{array}{rrcc}
        & \sqrt{\lambda}C_{2} & = & 0\\
       -\sqrt{\lambda}\sin(\sqrt{\lambda}L)C_{1} &+ \sqrt{\lambda}\cos(\sqrt{\lambda}L)C_{2}&=&0
      \end{array}\right. ,
\end{equation*}
which implies 
\[\sin(\sqrt{\lambda}L)C_{1}=0 \,\Leftrightarrow\, C_{1} = 0 \text{ or } \sin(\sqrt{\lambda}L)=0.\]
For nontrivial solutions we need 
\[\sin(\sqrt{\lambda}L)=0 \,\Leftrightarrow \, \sqrt{\lambda}L=n\pi.\] 
Thus, the eigenvalues are
\[\boxed{\lambda_{n}=\frac{n^{2}\pi^{2}}{L^{2}}}, \quad n\geq 1,\]
with eigenfunctions
\[\boxed{y_{n}(x)=C_{n}\cos\left( \frac{n \pi x}{L} \right)}, \quad n\geq 1,\]
for some arbitrary constants $C_{n}$.

\textsl{Combining 2 and 3.} We can combine \textsl{2} and \textsl{3} since $\cos(0)=1$. Hence, the eigenvalues are
\[\boxed{\lambda_{n}=\frac{n^{2}\pi^{2}}{L^{2}}}, \quad n\geq 0,\]
with eigenfunctions
\[\boxed{y_{n}(x)=C_{n}\cos\left( \frac{n \pi x}{L} \right)}, \quad n\geq 0,\]
for some arbitrary constants $C_{n}$.
\end{solution}



\begin{problem}
 Find the eigenvalues $\lambda$ for which the given problem has a nontrivial solution. Also determine the corresponding eigenfunctions.
\begin{equation*}
y^{\prime \prime} + \lambda y =0\,; \qquad  0<x<\pi \,, \qquad y(0) - y'(0)=0  \,,\qquad y(\pi) =0. 
\end{equation*}
\end{problem}
\begin{solution}
First we find the roots of the auxiliary equation.
\[r=\pm\sqrt{-\lambda}=\pm\sqrt{\Delta}.\]
We consider the three cases.

\par \textsl{Case 1.} $\Delta = -\lambda >0 \,\, \Rightarrow \, \lambda<0.$ Then $r_{1}=\sqrt{-\lambda}$, $r_{2}=-\sqrt{-\lambda}$ are distinct real roots, and the solution to ODE is
\[y(x)=C_{1}e^{\sqrt{-\lambda}x}+C_{2}e^{-\sqrt{-\lambda}x}.\]
We need $y'(x)$ in order to use the first initial conditions
\[y'(x)=\sqrt{-\lambda}C_{1}e^{\sqrt{-\lambda}x}-\sqrt{-\lambda}C_{2}e^{-\sqrt{-\lambda}x}.\]
Using the BC's,
\begin{equation*}
\left\{\begin{array}{rcl}
       (1-\sqrt{-\lambda})C_{1} + (1+\sqrt{-\lambda})C_{2}&=&0\\
       e^{\sqrt{-\lambda}\pi}C_{1} +e^{-\sqrt{-\lambda}\pi}C_{2}&=&0
      \end{array}\right. .
\end{equation*}
For nontrivial solutions we need 
$$\det(A)=\left|\begin{matrix}1-\sqrt{-\lambda} & 1+\sqrt{-\lambda} \\ e^{\sqrt{-\lambda}\pi} & e^{-\sqrt{-\lambda}\pi}\end{matrix}\right|=(1-\sqrt{-\lambda})e^{-\sqrt{-\lambda}\pi}-(1+\sqrt{-\lambda})e^{\sqrt{-\lambda}\pi}=0.$$ 
Or 
\[1-\sqrt{-\lambda}-(1+\sqrt{-\lambda})e^{2\sqrt{-\lambda}\pi}=0\]
Since $\lambda <0$, then $-e^{2\sqrt{-\lambda}\pi}<1$, and
\[1-\sqrt{-\lambda}-(1+\sqrt{-\lambda})e^{2\sqrt{-\lambda}\pi}<1-\sqrt{-\lambda}-1-\sqrt{-\lambda}=-2\sqrt{-\lambda}<0,\]
which is a contradiction. Thus, no non-trivial solution.

\par \textsl{Case 2.} $\Delta = -\lambda =0 \,\, \Rightarrow \, \lambda=0.$ Then $r_{1}=r_{2}=0$ are repeated real roots, and the solution to ODE is
\[y(x)=C_{1}x+C_{2}.\]
We need $y'(x)$ in order to use the first initial conditions
\[y'(x)=C_{1}.\]
Using the BC's,
\begin{equation*}
\left\{\begin{array}{rcl}
       C_{2} -C_{1}& = &0\\
       C_{1}\pi+C_{2} & = &0
      \end{array}\right. \quad \Leftrightarrow \quad C_{1}=C_{2}=0.
\end{equation*}
Hence, \textsl{there is no nontrivial solution}.

\par \textsl{Case 3.} $\Delta = -\lambda <0 \,\, \Rightarrow \, \lambda>0.$ Then $r_{1}=i\sqrt{\lambda}$, $r_{2}=-i\sqrt{\lambda}$ are complex roots, and the solution to ODE is
\[y(x)=C_{1}\cos\sqrt{\lambda}x+C_{2}\sin\sqrt{\lambda}x.\]
We need $y'(x)$ in order to use the initial conditions
\[y'(x)=-\sqrt{\lambda}C_{1}\sin\sqrt{\lambda}x+\sqrt{\lambda}C_{2}\cos\sqrt{\lambda}x.\]
Using the BC's,
\begin{equation*}
\left\{\begin{array}{rcl}
       C_{1}-\sqrt{\lambda}C_{2}  & = & 0\\
       \cos(\sqrt{\lambda}\pi)C_{1} + \sin(\sqrt{\lambda}\pi)C_{2}&=&0
      \end{array}\right. ,
\end{equation*}
which implies 
\[C_{1}=\sqrt{\lambda}C_{2}\Rightarrow \sqrt{\lambda}\cos(\sqrt{\lambda}\pi)C_{2}+\sin(\sqrt{\lambda}\pi)C_{2}=0 \,\Rightarrow\, \sqrt{\lambda} + \tan(\sqrt{\lambda}\pi)=0,\]
which has infinite solutions. Thus, the eigenvalues are given by the implicit equation
\[\boxed{\lambda_{n}+ \tan\left(\sqrt{\lambda_{n}}\pi\right)=0}, \quad n\geq 1,\]
with eigenfunctions
\[\boxed{y_{n}(x)=C_{n}\left(\sqrt{\lambda_{n}}\cos \left(\sqrt{\lambda_{n}}x\right)+ \sin \left(\sqrt{\lambda_{n}}x\right)\right)}, \quad n\geq 1,\]
for some arbitrary constants $C_{n}$.
\end{solution}





\begin{problem}
 Find nontrivial solutions to the eigenvalue problem
\begin{equation*}
y^{\prime \prime} + 4 y' + \lambda y =0\,; \qquad  0<x<\pi/2 \,, \qquad y(0)=0  \,,\qquad y(\pi/2) =0. 
\end{equation*}
\end{problem}
\begin{solution}
First we find the roots of the auxiliary equation.
\[r=-2\pm\sqrt{4-\lambda}=2\pm\sqrt{\Delta}.\]
We consider the three cases.\\

\par \textsl{Case 1.} $\Delta = 4-\lambda >0 \,\, \Rightarrow \, \lambda<4.$ Then $r_{1}=-2+\sqrt{4-\lambda}$, $r_{2}=-2-\sqrt{4-\lambda}$ are distinct real roots, and the solution to ODE is
\[y(x)=C_{1}e^{r_{1}x}+C_{2}e^{r_{2}x}.\]
Using the BC's,
\begin{equation*}
\left\{\begin{array}{rrcc}
       C_{1} &+C_{2}&=&0\\
       e^{r_{1}\pi/2}C_{1} &+ e^{r_{2}\pi/2}C_{2}&=&0
      \end{array}\right. .
\end{equation*}
For nontrivial solutions we need 
$$\det(A)=\left|\begin{matrix} 1 & 1 \\ e^{r_{1}\pi/2} & e^{r_{2}\pi/2}\end{matrix}\right|=e^{r_{1}\pi/2}-e^{r_{2}\pi/2}=0,$$ 
which implies $r_{1}=r_{2}$, which is a contradiction since the roots are different. Hence, \textsl{there is no nontrivial solution}.

\par \textsl{Case 2.} $\Delta = 4-\lambda =0 \,\, \Rightarrow \, \lambda=4.$ Then $r_{1}=r_{2}=-2$ are repeated real roots, and the solution to ODE is
\[y(x)=C_{1}e^{-2x}+C_{2}xe^{-2x}.\]
Using the BC's,
\begin{equation*}
\left\{\begin{array}{rrcc}
       C_{1} & &= &0\\
       e^{-\pi}C_{1} & +\tfrac{\pi}{2}e^{\pi}C_{2} &= &0
      \end{array}\right. \quad \Leftrightarrow \quad C_{1}=C_{2}=0.
\end{equation*}
Hence, \textsl{there is no nontrivial solution}.

\par \textsl{Case 3.} $\Delta = 4-\lambda <0 \,\, \Rightarrow \, \lambda>4.$ Then $r_{1}=-2+i\sqrt{\lambda-4}$, $r_{2}=-2-i\sqrt{\lambda-4}$ are complex roots, and the solution to ODE is
\[y(x)=C_{1}e^{-2x}\cos\sqrt{\lambda-4}x+C_{2}e^{-2x}\sin\sqrt{\lambda-4}x.\]
Using the BC's,
\begin{equation*}
\left\{\begin{array}{rrcc}
       C_{1} & & = & 0\\
       e^{-\pi}\cos(\sqrt{\lambda-4}\tfrac{\pi}{2})C_{1} &+ e^{-\pi}\sin(\sqrt{\lambda-4}\tfrac{\pi}{2})C_{2}&=&0
      \end{array}\right. ,
\end{equation*}
which implies 
\[\sin(\sqrt{\lambda-4}\tfrac{\pi}{2})C_{2}=0 \,\Leftrightarrow\, C_{2} = 0 \text{ or } \sin(\sqrt{\lambda-4}\tfrac{\pi}{2})=0.\]
For nontrivial solutions we need 
\[\sin(\sqrt{\lambda-4}\tfrac{\pi}{2})=0 \,\Leftrightarrow \, \sqrt{\lambda-4}\tfrac{\pi}{2}=n\pi \,\Leftrightarrow \, \sqrt{\lambda-4}=2n.\] 
Thus, the eigenvalues are
\[\boxed{\lambda_{n}=4n^{2}+4}, \quad n > 0,\]
with eigenfunctions
\[\boxed{y_{n}(x)=C_{n}e^{-2x}\sin(2nx)}, \quad n > 0,\]
for some arbitrary constants $C_{n}$.
\end{solution}




\begin{problem}
Solve the heat flow problem
\begin{equation*} \begin{split}
& \dfrac{\partial u }{\partial t } (x,t) = \alpha \dfrac{\partial^2 u}{\partial x^2}(x,t), \qquad 0<x<1, \quad t>0, \\
& u(0,t) = u(1,t)=0, \qquad t>0, \\
& u(x,0) = x(1-x), \qquad 0<x<1.
\end{split}\end{equation*}
\end{problem}

\begin{solution}
Using the separation of variables method
\begin{eqnarray*}
u(x,t) = X(x)T(t), \qquad
\dfrac{\partial u }{\partial t }(x,t) = X(x)T^{\prime}(t), \qquad
\dfrac{\partial^2 u}{\partial x^2}(x,t) = X^{\prime \prime}(x)T(t).
\end{eqnarray*}
Using the partial differential equation
\begin{eqnarray*}
X(x)T^{\prime}(t) = \alpha X^{\prime \prime}(x)T(t) \quad 
\Rightarrow \quad \dfrac{T^{\prime}(t)}{\alpha T(t)} = \dfrac{X^{\prime \prime}(x)}{X(x)} = \lambda.
\end{eqnarray*}
The last trick is the key to solve heat (and wave) equation. First, we divide over $T$ and $X$ to find out that the equation has only $T$ on one side, and only $X$ on the other. From here we can put every on the $T$ side so the $X$ side becomes simply $\frac{X''}{X}$. Then, we use a constant value $\lambda$ that we need to determine by solving the eigenvalue problem. You can use $\lambda$ or $-\lambda$, but you have to be clear on what is solution to each case. Finally, we can split that equation into two parts
\[\frac{X''(x)}{X(x)}=\lambda, \qquad \frac{T'(x)}{\alpha T(x)}=\lambda.\]

Then, solving the heat flow problem reduces now to solving the following three problems.

\textsl{1.} The eigenvalue problem
\begin{eqnarray*}
X^{\prime \prime}(x) - \lambda X(x) =0, \quad 0<x<1; \qquad
X(0)=0, \quad & X(1) =0,
\end{eqnarray*}
has eigenvalues $$\lambda_{n} = -n^2\pi^{2}, \quad n = 1,2 , \dots$$ with eigenfunctions $$\boxed{X_{n}(x) = C_{n} \sin(n\pi x)},\quad n = 1,2 , \dots.$$
for some constants $C_{n}$.


\textsl{2. } Using the eigenvalue $\lambda = -n^2\pi^{2}$ found in \textsl{1}, is easy to see that the first order differential equations
\begin{eqnarray*}
T_{n}^{\prime}(t) = - \alpha n^{2}\pi^{2} T_{n}(t),
\end{eqnarray*}
has solution
\begin{eqnarray*}
\boxed{T_{n}(t) = A_{n}e^{-\alpha n^{2}\pi^{2} t}},\quad n = 1,2 , \dots
\end{eqnarray*}
for some constants $A_{n}$. Recall that, this is the simplest first order ODE.

Note now that we $n$ solutions of the form $u_{n}(x,t)=X_{n}(x)T_{n}(t)$. The superposition principle says that a general solution is

\[u(x,t)=\sum_{n=1}^{\infty} B_{n}X(x)T(t)=\sum_{n=1}^{\infty} b_{n} \sin (n\pi x) e^{-\alpha n^{2}\pi^{2} t}.\]

The constants $A_{n}$, $B_{n}$ and $C_{n}$ where combined into a single constant $b_{n}$. Actually, the only constant that is important is $b_{n}$ (we can ignore $C_{n}$ and $A_{n}$), because when we use the initial condition $u(x,0)=x(1-x)$ we have

\[u(x,0)= x(1-x) = \sum_{n=1}^{\infty} b_{n} \sin (n\pi x).\]
Which is nothing but the Fourier since series of $x(1-x)$, the third problem.

\textsl{3. } The Fourier sine series representation for the function $f(x)=x(1-x)$ 
\begin{eqnarray*}
x(1-x) = \sum_{n=1}^{\infty} b_{n} \sin(n\pi x),
\end{eqnarray*}
has coefficients (see a previous problem in this notes)
\begin{eqnarray*}
\boxed{b_{n} = \dfrac{4(1-(-1)^n)}{n^3 \pi^3}}.
\end{eqnarray*}
Besides we can separate $b_{n}$ in odd and even cases, there is no need to (plus we save time).

Putting the three previous results together, the solution to the heat flow problem is then given by
\begin{eqnarray*}
\boxed{u(x,t) = \sum_{n=1}^{\infty}\frac{4(1-(-1)^{n})}{n^{3}\pi^{3}}e^{-\alpha n^{2}\pi^{2}t}\sin (n\pi x)}.
\end{eqnarray*}

\textbf{Note.} The steps the I used are slightly different from the same problem in Lab 11, however this is more clear to deal with the wave equation.

\end{solution}





\begin{problem}
Solve the heat flow problem
\begin{equation*} \begin{split}
& \dfrac{\partial u }{\partial t } (x,t) = \alpha \dfrac{\partial^2 u}{\partial x^2}(x,t), \qquad 0<x<1, \quad t>0, \\
& \dfrac{\partial u }{\partial x } (0,t) = \dfrac{\partial u }{\partial x } (1,t)=0, \qquad t>0, \\
& u(x,0) = x(1-x), \qquad 0<x<1.
\end{split}\end{equation*}
\end{problem}

\begin{solution}
\textbf{Note.} The solution for this problem will be straight forward since the steps are explained in the previous problem.

Using the separation of variables method
\begin{eqnarray*}
u(x,t) = X(x)T(t), \qquad
\dfrac{\partial u }{\partial t }(x,t) = X(x)T^{\prime}(t), \qquad
\dfrac{\partial^2 u}{\partial x^2}(x,t) = X^{\prime \prime}(x)T(t),
\end{eqnarray*}
in the partial differential equation
\begin{eqnarray*}
X(x)T^{\prime}(t) = \alpha X^{\prime \prime}(x)T(t) \quad 
\Rightarrow \quad \dfrac{T^{\prime}(t)}{\alpha T(t)} = \dfrac{X^{\prime \prime}(x)}{X(x)} = \lambda.
\end{eqnarray*}

Then, solving the heat flow problem reduces now to solving the following three problems.

\textsl{1.} The eigenvalue problem
\begin{eqnarray*}
X^{\prime \prime}(x) - \lambda X(x) =0, \quad 0<x<1; \qquad
X'(0)=0, \quad & X'(1) =0,
\end{eqnarray*}
has eigenvalues 
$$\lambda_{n} = -n^2\pi^{2}, \quad n = 0,1,2 , \dots$$ 
with eigenfunctions 
$$\boxed{X_{n}(x) = B_{n}\cos(n\pi x)},\quad n = 0,1,2 , \dots$$


\textsl{2. } Using the eigenvalue $\lambda = -n^2\pi^{2}$ found in \textsl{1}, is easy to see that 
\begin{eqnarray*}
T_{n}^{\prime}(t) = - \alpha n^{2}\pi^{2} T_{n}(t),
\end{eqnarray*}
has solution
\begin{eqnarray*}
\boxed{T_{n}(t) = A_{n}e^{-\alpha n^{2}\pi^{2} t}},\quad n = 0,1,2 , \dots
\end{eqnarray*}

\textbf{Note.} The eigenfunction $X_{n}=\cos(n\pi x)$ implies that we need the Fourier \textsl{cosine} series.

\textsl{3. } The Fourier cosine series representation for the function $f(x)=x(1-x)$ 
\begin{eqnarray*}
x(1-x) = \sum_{n=1}^{\infty} b_{n} \sin(n\pi x),
\end{eqnarray*}
has coefficients (see a previous problem in this notes)
\begin{eqnarray*}
\boxed{b_{0} = \frac{1}{3}, \quad b_{n} = \frac{2((-1)^{n+1}-1)}{n^{2}\pi^{2}}}, \quad n = 1,2,\dots
\end{eqnarray*}

Putting the three previous results together, the solution to the heat flow problem is then given by
\begin{eqnarray*}
\boxed{u(x,t) = \frac{1}{6} + \sum_{n=1}^{\infty}\frac{2((-1)^{n+1}-1)}{n^{2}\pi^{2}}e^{-\alpha n^{2}\pi^{2}t}\cos (n\pi x)}.
\end{eqnarray*}


\end{solution}






\begin{problem}
Solve the heat flow problem
\begin{equation*} \begin{split}
& \dfrac{\partial u }{\partial t } (x,t) = \dfrac{\partial^2 u}{\partial x^2}(x,t), \qquad 0<x<\pi, \quad t>0, \\
& u(0,t) = 0, \quad u(\pi,t)= 3\pi, \qquad t>0, \\
& u(x,0) = 0, \qquad 0<x<\pi.
\end{split}\end{equation*}
\end{problem}

\begin{solution}
Note that we have non-zero boundary conditions. So, we apply the appropriate change of variable
\[u(x,t)=v(x)+w(x, t)=3x + w(x,t),\]
since
\[v(x)=(3\pi-0)\frac{x}{\pi}+0=3x.\]

Then, the new problem is to solve the alternate Heat equation \textsl{with zero boundary conditions}
\begin{equation*} \begin{split}
& \dfrac{\partial w }{\partial t } (x,t) = \dfrac{\partial^2 u}{\partial x^2}(x,t), \qquad 0<x<\pi, \quad t>0, \\
& w(0,t) = 0, \quad w(\pi,t)= 0, \qquad t>0, \\
& w(x,0) = -3x, \qquad 0<x<\pi.
\end{split}\end{equation*}
Now, we use separation of variables
\[w(x,t)=X(x)T(t),\]
\[\Rightarrow\, \frac{\partial w}{\partial t}(x,t)=X(x)T'(t),\quad \frac{\partial^{2} w}{\partial x^{2}}(x,t)=X''(x)T(t).\]
Then 
\[X(x)T'(t)=X''(x)T(t) \,\,\Rightarrow\,\, \frac{T'(t)}{T(t)}=\frac{X''(x)}{X(x)}=\lambda.\]
Which becomes
\[\begin{split}T'(t)-\lambda T(t)&=0 \\ X''(x)-\lambda X(x)&=0 \end{split}.\]

\textsl{1. } First, the eigenvalue problem
\[X''(x)-\lambda X(x)=0, \qquad 0<x<\pi, \qquad  X(x)=X(\pi)=0,\]
has solution 
\[\begin{split}
  \lambda_{n}=-n^{2}, \,\, n=1, 2, \dots, \\
  \boxed{X_{n}(x)=B_{n}\sin(nx)}, \,\, n=1, 2, \dots.
  \end{split}
\]

\textsl{2. } Now, we use the previous $\lambda$ to solve the first order differential equation
\[T'(t) + n^{2} T(t)=0\]
with solution
\[\boxed{T_{n}(t)=A_{n}e^{-n^{2}t}}.\]

Now, the eigenvalues will be used in the other differential equation (eigenvalue problem) for $T(t)$, while the coefficients $b_{n}$ will be determined by the initial condition $w(x,0)=-3x$. 

\textsl{3.} We require
\[-3x=\sum_{i=1}^{\infty}b_{n}\sin(nx),\]
the Fourier sine series of $-3x$. The coefficients are given by
\begin{align*}
\boxed{b_{n}=\frac{2}{\pi}\int_{0}^{\pi}(-3x)\sin(nx)dx=-\frac{6}{\pi}\left[ -\frac{1}{c}x\cos(nx)|^{\pi}_{0} + \frac{1}{n^{2}}\sin(nx)|^{\pi}_{0} \right] = \frac{6(-1)^{n}}{n}}.
\end{align*}

Putting the three previous steps together 
\[w(x,t)=\sum_{n=1}^{\infty}X_{n}(x)T_{n}(t),\]
that is 
\[\boxed{w(x, t) =  \sum_{n=1}^{\infty}\frac{6(-1)^{n}}{n}\sin(nx)e^{-n^{2}t}}.\]

Finally, changing back to $u(x,t)$
\[\boxed{u(x, t) = 3x + 6 \sum_{n=1}^{\infty}\frac{(-1)^{n}}{n}\sin(nx)e^{-n^{2}t}}.\]
\end{solution}




\begin{problem}
Find a formal solution to the initial value problem,
\begin{equation*} \begin{split}
& \dfrac{\partial u }{\partial t } (x,t) = \dfrac{\partial^2 u}{\partial x^2}(x,t) + 6 x -2, \qquad 0<x<1, \quad t>0, \\
& u(0,t) = 0, \quad u(1,t)= -1, \qquad t>0, \\
& u(x,0) = -x^{3}, \qquad 0<x<1.
\end{split}\end{equation*}
\end{problem}

\begin{solution}
Note that we have non-zero boundary conditions and external force. So, we apply the appropriate change of variable
\[u(x,t)=w(x, t)+v(x).\]

Then,
\[\frac{\partial u}{\partial t}(x,t) = \frac{\partial w}{\partial t}, \qquad \frac{\partial^{2} u}{\partial x^{2}}(x, t) = \frac{\partial^{2} w}{\partial x^{2}}+v''(x).\]

Using the Heat equation we have
\[\frac{\partial w}{\partial t}(x,t) = \frac{\partial^{2} w}{\partial x^{2}}(x,t) + v''(x) + 6x -2,\]
where we want $v(x)$ to absorbe the external force (see the second order equation later).

Using the boundary conditions we have
\[w(0, t) + v(0) = 0, \qquad w(1, t) + v(1) = -1,\]
where we want $v(x)$ to absorbe the non-zero boundary conditions (see second order equation later).

Using the initial value we have
\[w(x, 0) + v(x) = -x^{3} \quad \Rightarrow \quad  w(x, 0) = -x^{2} - v(x),\]
where we relate $v(x)$ to $w(x,t)$ (see initial conditions in the Heat equation later).


Then, the new two problem is to solve the alternate Heat equation \textsl{with zero boundary conditions}
\begin{equation*} \begin{split}
& \dfrac{\partial w }{\partial t } (x,t) = \dfrac{\partial^2 u}{\partial x^2}(x,t), \qquad 0<x<1, \quad t>0, \\
& w(0,t) = 0, \quad w(1,t)= 0, \qquad t>0, \\
& w(x,0) = -x^{3} - v(x), \qquad 0<x<1,
\end{split}\end{equation*}

after solving the second order equation \textsl{with non-zero boundary conditions}
\begin{equation*} 
\begin{split}
&v'' = -6x+2, \quad 0< x< 1\\
&v(0) = 0,\quad v(1) = -1.
\end{split}
\end{equation*}

First, we solve the second order problem, to then solve the new Heat equation problem.

\textsl{1. } The solution to $v(x)$ is (verify it)
\[\boxed{v(x) = -x^{3} + x^{2} - x}.\]
Using this $v(x)$, the previous Heat equation problem in $w(x, t)$ changes to


\begin{equation*} \begin{split}
& \dfrac{\partial w }{\partial t } (x,t) = \dfrac{\partial^2 u}{\partial x^2}(x,t), \qquad 0<x<1, \quad t>0, \\
& w(0,t) = 0, \quad w(1,t)= 0, \qquad t>0, \\
& w(x,0) = x(1-x), \qquad 0<x<1,
\end{split}\end{equation*}

\textsl{2. } The solution to this problem is (see a previous problem in this notes)

\begin{eqnarray*}
\boxed{w(x,t) = \sum_{n=1}^{\infty}\frac{4(1-(-1)^{n})}{n^{3}\pi^{3}}e^{-\alpha n^{2}\pi^{2}t}\sin (n\pi x)}.
\end{eqnarray*}

Finally, combining the two previous solutions we get
\[\boxed{u(x, t) = -x^{3} + x^{2} - x + \sum_{n=1}^{\infty}\frac{4(1-(-1)^{n})}{n^{3}\pi^{3}}e^{-\alpha n^{2}\pi^{2}t}\sin (n\pi x)}.\]
\end{solution}



\end{document}
