% MATH 201 Lab notes (c) by Carlos Contreras And Philippe Gaudreau
% MATH 201 Lab notes is licensed under a 
% Creative Commons Attribution 4.0 International license.
% CC BY 4.0

% You should have received a copy of the license along with this
% work. If not, see <http://creativecommons.org/licenses/by/4.0/>.

\documentclass[11pt]{article}
% MATH 201 Lab notes (c) by Carlos Contreras And Philippe Gaudreau
% MATH 201 Lab notes is licensed under a 
% Creative Commons Attribution 4.0 International license.
% CC BY 4.0

% You should have received a copy of the license along with this
% work. If not, see <http://creativecommons.org/licenses/by/4.0/>.

%% libraries
\usepackage[utf8x]{inputenc}
\usepackage{xcolor}
\usepackage[left=1.5cm,right=1.5cm,top=2.0cm,bottom=1.5cm,headheight=110pt]{geometry}
\usepackage{amsmath}
\usepackage{amssymb}
\usepackage{graphicx}
\usepackage{xifthen}
\usepackage{sverb}
\usepackage{fancyhdr}
\usepackage{mdframed}
\usepackage{textcomp}

%%%%%%%%%%%%%%%%%%%%%%%%%%%%%%%%%%%%%%%%%%%%%%%%%%%%%%%%%%%%%%%%%%%%%%%
% PDF compiling
\usepackage{ifpdf}
\ifpdf %
        \DeclareGraphicsExtensions{.pdf}%
\else %
        \DeclareGraphicsExtensions{.eps,.ps}%
\fi

%%%%%%%%%%%%%%%%%%%%%%%%%%%%%%%%%%%%%%%%%%%%%%%%%%%%%%%%%%%%%%%%%%%%%%%
% Figures path
\graphicspath{{figures/}}

%%%%%%%%%%%%%%%%%%%%%%%%%%%%%%%%%%%%%%%%%%%%%%%%%%%%%%%%%%%%%%%%%%%%%%%
% Problem counter
\newcounter{Problem}
\setcounter{Problem}{0}

%%%%%%%%%%%%%%%%%%%%%%%%%%%%%%%%%%%%%%%%%%%%%%%%%%%%%%%%%%%%%%%%%%%%%%%
% Definitions
\def\LabSolutions{\clearpage \newpage \begin{center} {\Large \it Solutions} \end{center} \setcounter{Problem}{0}}
\def\QuizSolutions{\newpage \begin{center} {\Large \it Solutions} \end{center} \setcounter{Problem}{0}}
\def\degree{\textdegree}
\def\grade#1{\begin{flushright} {\small [#1]}\\ \end{flushright} \vspace{-10pt}}
\def\codecolor{red!50!black}
\def\code#1{\textcolor{\codecolor}{\tt #1}}
\def\examname#1{%
                \ifnum\value{page}>1%
                    \newpage%
                \else%
                    \vspace*{5pt}%
                \fi%
                \large \textbf{#1} \setcounter{Problem}{0}\vspace{10pt}}
\def\topic#1{\par\needspace{2\baselineskip} \noindent \textsl{\footnotesize #1}}

 
%%%%%%%%%%%%%%%%%%%%%%%%%%%%%%%%%%%%%%%%%%%%%%%%%%%%%%%%%%%%%%%%%%%%%%%
% Environments
\newenvironment{problem}%
     {\stepcounter{Problem}%
      \begin{list}{\textbf{\arabic{Problem}}.~}{}%
      \item}%
     {\end{list}\vspace*{5pt}}

\newenvironment{solution}%
     {\indent \textit{Solution} \newline}%
     {\begin{flushright}$\blacksquare$\end{flushright}}

\newenvironment{preamble}%
     {\vspace*{1em}\begin{mdframed}[leftmargin=1cm,rightmargin=1cm]}%
     {\end{mdframed}\vspace*{1em}}

\newenvironment{multchoice}%
     {\begin{enumerate} \addtolength{\leftskip}{2em} \renewcommand{\labelenumi}{(\alph{enumi})}}
     {\end{enumerate}}

\newenvironment{formulaitem}%
     {\setlength{\leftmargini}{1.5em}\begin{itemize}%
      \setlength\itemindent{-\itemindent}%
      \renewcommand{\labelitemi}{$\rightarrow$}}%
     {\end{itemize}}


%%%%%%%%%%%%%%%%%%%%%%%%%%%%%%%%%%%%%%%%%%%%%%%%%%%%%%%%%%%%%%%%%%%%%%%
% New theorems
\newtheorem{theorem}{Theorem}


\makeatletter

%%%%%%%%%%%%%%%%%%%%%%%%%%%%%%%%%%%%%%%%%%%%%%%%%%%%%%%%%%%%%%%%%%%%%%%
%% New commands
\newcommand*{\course}[1]{\gdef\@course{#1}}
\newcommand*{\coursecode}[1]{\gdef\@coursecode{#1}}
\newcommand*{\term}[1]{\gdef\@term{#1}}
\newcommand*{\instructor}[1]{\gdef\@instructor{#1}}
\newcommand*{\lqnumber}[1]{\gdef\@lqnumber{#1}}
\newcommand*{\labtitle}[1]{\gdef\@labtitle{#1}}
\newcommand*{\quizversion}[1]{\gdef\@quizversion{#1}}
\newcommand*{\probleminfo}[1]{\noindent \textsl{\footnotesize #1}}

% Title header for labs
\newcommand\makelabtitle{%
  \begin{flushleft}%
  {\scshape \@coursecode~\@course~-- University of Alberta}\\%
  {\scshape \@term~-- Labs -- \@instructor}\\%
  {\scshape Authors: Carlos Contreras and Philippe Gaudreau}%
  \end{flushleft}%
  \begin{center}%
  {\Large \bf \@lqnumber:~\@labtitle}%
  \end{center}%
  \thispagestyle{empty}%
  \global\let\@course\@empty%
  \global\let\@labtitle\@empty%
}

% Title header for quizzes
\newcommand\makequiztitle{%
  \begin{flushleft}%
  {\scshape \@coursecode~\@course~-- University of Alberta}\\%
  {\scshape \@term~-- Labs -- \@instructor}\\%
  \end{flushleft}%
  \begin{center}%
  {\Large \bf \@lqnumber} \marginpar{\tiny\tt [\@quizversion]}%
  \end{center}%
  \thispagestyle{empty}%
  \global\let\@course\@empty%
  \global\let\@quizversion\@empty%
}

%%%%%%%%%%%%%%%%%%%%%%%%%%%%%%%%%%%%%%%%%%%%%%%%%%%%%%%%%%%%%%%%%%%%%%%
% Fancy header package
\fancyhead[L]{\small {\scshape \@coursecode~-- \@lqnumber~-- \@term~-- \@instructor}}
\pagestyle{fancy}

\makeatother


\usepackage{hyperref}
\usepackage{cancel}


\begin{document}

\course{Differential Equations}
\coursecode{MATH 201}
\term{Winter 2018}
\instructor{Carlos Contreras}
\lqnumber{Lab 0}
\labtitle{Integration techniques}
\makelabtitle

% 1
\begin{problem}
Find the given integral to compute
\[e^{\int \frac{-4}{x}dx}.\]
\end{problem}

%2
\begin{problem}
Find
\[\int\frac{x}{\sqrt{1+x}}dx.\]
\end{problem}


%3
\begin{problem}
Find the following integrals
\[\int^{\pi/2}_{-\pi/2}\cosh(ix)dx, \qquad \int^{\pi/2}_{-\pi/2}\sinh(ix)dx.\]
\end{problem}



\begin{problem}
Find the following integrals
\[\int x^{3}\cos(n\pi x)dx, \qquad \int x^{3}\sin(n\pi x)dx.\]
\end{problem}


\begin{problem}
For constants $a$ and $b$, find the integral
\[\int_{0}^{\infty} \sin(bt)e^{-at}dt.\]
\end{problem}


\begin{problem}
Show that
\[\int \frac{dx}{a^{2}+x^{2}}=\frac{1}{a}\arctan \left( \frac{x}{a} \right)+C.\]
\end{problem}






%%%%%%%%%%%%%%%%%%%%%%%%%%%%%%%%%%%%%%%%%%%%%%%%%%%%%%%%%%%%%%%%%%%%%%%%%%%%%%%%%%%%%%%%%%%%%%%%%%%



\LabSolutions

% 1
\begin{problem}
Find the given integral to compute
\[e^{\int \frac{-4}{x}dx}.\]
\end{problem}

\begin{solution}
Integrating and applying logarithm properties
\[\int \frac{-4}{x}dx = -4 \ln |x| + C = \ln |x|^{-4} + C = \ln |x^{-4}| +C.\]
Thus
\[\boxed{e^{\int \frac{-4}{x}dx} = e^{\ln |x^{-4}| +C} = K x^{-4}},\]
for some constant $K$.
\end{solution}


%2
\begin{problem}
Find
\[\int\frac{x}{\sqrt{1+x}}dx.\]
\end{problem}
\begin{solution}
Using substitution $t = x +1$, we have
\[\int\frac{x}{\sqrt{1+x}}dx = \int\frac{t-1}{\sqrt{t}}dt=\int t^{1/2}dt - \int t^{-1/2}dt = \frac{2}{3}t^{3/2}-2t^{1/2}+C,\]
Subtituting back to $x$ 
\[\boxed{\int\frac{x}{\sqrt{1+x}}dx=\frac{2}{3}(x+1)^{3/2}-2(x+1)^{1/2}+C}.\]
\end{solution}



%3
\begin{problem}
Find the following integrals
\[\int^{\pi/2}_{-\pi/2}\cosh(ix)dx, \qquad \int^{\pi/2}_{-\pi/2}\sinh(ix)dx.\]
\end{problem}
\begin{solution}
Recall that
\[\cosh(ix)=\frac{e^{ix}+e^{-ix}}{2}=\cos(x),\]
\[\sin(ix)=\frac{e^{ix}-e^{-ix}}{2}=i\sin(x).\]
Thus,
\[\int^{\pi/2}_{-\pi/2}\cosh(ix)dx = \int^{\pi/2}_{-\pi/2}\cos(x)dx= 2 \int^{\pi/2}_{0}\cos(x)dx = 2 \left.\sin x \right|_{0}^{\pi/2} = 2.\]
and
\[\int^{\pi/2}_{-\pi/2}\sinh(ix)dx = i \int^{\pi/2}_{-\pi/2}\sin(x)dx = 0.\]
\end{solution}




\begin{problem}
Find the following integrals
\[ \int x^{3}\cos(n\pi x)dx, \qquad \int x^{3}\sin(n\pi x)dx.\]
\end{problem}
\begin{solution}
Consider the first integral. We need to integrate by parts three time, so the following table becomes handy.
\begin{center}
\begin{tabular}{c|c|c}
sign & derivative & integral \\
\hline
    &  -- --   & $ \cos(n\pi x)$ \\
$+$ & $x^{3}$  & $ \frac{1}{n \pi} \sin(n\pi x)$ \\
$-$ & $3x^{2}$ & $-\left( \frac{1}{n \pi} \right)^{2} \cos(n\pi x)$ \\
$+$ & $6x$     & $-\left( \frac{1}{n \pi} \right)^{3} \sin(n\pi x)$ \\
$-$ & $6$      & $ \left( \frac{1}{n \pi} \right)^{4} \cos(n\pi x)$ \\
$+$ & $0$      & $ \left( \frac{1}{n \pi} \right)^{5} \sin(n\pi x)$
\end{tabular}
\end{center}
There is actually no need to compute the last row since the derivative becomes zero. From there, is easy to compute all terms of the integral at once by multiplying each row and add all terms.
\[ \boxed{\int x^{3}\cos(n\pi x)dx = x^{3}\frac{1}{n\pi}\sin(n \pi x) + 3x^{2}\left( \frac{1}{n \pi} \right)^{2} \cos(n\pi x) - 6x\left( \frac{1}{n \pi} \right)^{3} \sin(n\pi x) - 6\left( \frac{1}{n \pi} \right)^{4} \cos(n\pi x)}.\]
Consider now the second integral. The correspondent integration by parts table is
\begin{center}
\begin{tabular}{c|c|c}
sign & derivative & integral \\
\hline
    &   -- --  & $ \sin(n\pi x)$ \\
$+$ & $x^{3}$  & $-\frac{1}{n \pi} \cos(n\pi x)$ \\
$-$ & $3x^{2}$ & $-\left( \frac{1}{n \pi} \right)^{2} \sin(n\pi x)$ \\
$+$ & $6x$     & $ \left( \frac{1}{n \pi} \right)^{3} \cos(n\pi x)$ \\
$-$ & $6$      & $ \left( \frac{1}{n \pi} \right)^{4} \sin(n\pi x)$
\end{tabular}
\end{center}
Thus, 
\[\boxed{\int x^{3}\sin(n\pi x)dx = - x^{3}\frac{1}{n \pi} \cos(n\pi x) + 3x^{2}\left( \frac{1}{n \pi} \right)^{2} \sin(n\pi x) + 6x \left( \frac{1}{n \pi} \right)^{3} \cos(n\pi x) - 6\left( \frac{1}{n \pi} \right)^{4} \sin(n\pi x)}.\]

If you want to step further, you can compute both integrals at the same time by computing the single integral
\[ \int x^{3}e^{i n\pi x}dx = \int x^{3}\cos(n\pi x)dx + i \int x^{3}\sin(n\pi x)dx.\]
and then separating real and complex part. That is, using the table
\begin{center}
\begin{tabular}{c|c|c}
sign & derivative & integral \\
\hline
    &   -- --  & $ e^{i n\pi x}$ \\
$+$ & $x^{3}$  & $-i\left( \frac{1}{n \pi} \right) e^{i n\pi x}$ \\
$-$ & $3x^{2}$ & $- \left( \frac{1}{n \pi} \right)^{2} e^{i n\pi x}$ \\
$+$ & $6x$     & $ i\left( \frac{1}{n \pi} \right)^{3} e^{i n\pi x}$ \\
$-$ & $6$      & $  \left( \frac{1}{n \pi} \right)^{4} e^{i n\pi x}$
\end{tabular}
\end{center} 
Try it!
\end{solution}



\begin{problem}
For constants $a$ and $b$, find the integral
\[\int_{0}^{\infty} \sin(bt)e^{-at}dt.\]
\end{problem}
\begin{solution}
If we integrate by parts twice we arrive to the same integral.
\[\int_{0}^{\infty} \sin(bt)e^{-at}dt  = \left[ -\frac{1}{a}\sin (bt)e^{-at}\right]^{\infty}_{0} + \left[- \frac{b}{a^{2}}\cos (bt)e^{-at}\right]^{\infty}_{0} - \frac{b^{2}}{a^{2}}\int_{0}^{\infty}\sin(bt)e^{-at}dt.\]
Denoting $I=\int_{0}^{\infty} \sin(bt)e^{-at}dt$ and multiplying by $a^{2}$, we get. From there we can isolate the integral.
\[I = \left[ -\frac{1}{a}\sin (bt)e^{-at}\right]^{\infty}_{0} + \left[- \frac{b}{a^{2}}\cos (bt)e^{-at}\right]^{\infty}_{0} - \frac{b^{2}}{a^{2}}I.\]
Recall that
\[\lim_{t\rightarrow{\infty}}\cos(bt)e^{-at}=\lim_{t\rightarrow{\infty}}\sin(bt)e^{-at}=0,\]
then 
\[\left[ -\frac{1}{a}\sin (bt)e^{-at}\right]^{\infty}_{0} = 0 , \qquad \left[ -\frac{b}{a^{2}}\cos (bt)e^{-at}\right]^{\infty}_{0} = \frac{b}{a^{2}}.\]
Using and isolating $I$
\[I= \frac{b}{a^{2}+b^{2}} \quad \Rightarrow \quad  \boxed{\int_{0}^{\infty} \sin(bt)e^{-at}dt= \frac{b}{a^{2}+b^{2}}}.\]
\end{solution}



\begin{problem}
Show that
\[\int \frac{dx}{a^{2}+x^{2}}=\frac{1}{a}\arctan \left( \frac{x}{a} \right)+C.\]
\end{problem}
\begin{solution}
Using substitution $x = a \tan t$, ($dx = a \sec^{2}tdt$)
\[\int \frac{dx}{a^{2}+x^{2}} = \int \frac{a\sec^{2} t dt}{a^{2}\sec^{2}t} = \frac{1}{a}t + C = \boxed{\frac{1}{a}\arctan\left( \frac{x}{a} \right) + C}.\]
\end{solution}




\end{document}
