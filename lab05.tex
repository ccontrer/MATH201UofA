% MATH 201 Lab notes (c) by Carlos Contreras And Philippe Gaudreau
% MATH 201 Lab notes is licensed under a 
% Creative Commons Attribution 4.0 International license.
% CC BY 4.0

% You should have received a copy of the license along with this
% work. If not, see <http://creativecommons.org/licenses/by/4.0/>.

\documentclass[11pt]{article}
% MATH 201 Lab notes (c) by Carlos Contreras And Philippe Gaudreau
% MATH 201 Lab notes is licensed under a 
% Creative Commons Attribution 4.0 International license.
% CC BY 4.0

% You should have received a copy of the license along with this
% work. If not, see <http://creativecommons.org/licenses/by/4.0/>.

%% libraries
\usepackage[utf8x]{inputenc}
\usepackage{xcolor}
\usepackage[left=1.5cm,right=1.5cm,top=2.0cm,bottom=1.5cm,headheight=110pt]{geometry}
\usepackage{amsmath}
\usepackage{amssymb}
\usepackage{graphicx}
\usepackage{xifthen}
\usepackage{sverb}
\usepackage{fancyhdr}
\usepackage{mdframed}
\usepackage{textcomp}

%%%%%%%%%%%%%%%%%%%%%%%%%%%%%%%%%%%%%%%%%%%%%%%%%%%%%%%%%%%%%%%%%%%%%%%
% PDF compiling
\usepackage{ifpdf}
\ifpdf %
        \DeclareGraphicsExtensions{.pdf}%
\else %
        \DeclareGraphicsExtensions{.eps,.ps}%
\fi

%%%%%%%%%%%%%%%%%%%%%%%%%%%%%%%%%%%%%%%%%%%%%%%%%%%%%%%%%%%%%%%%%%%%%%%
% Figures path
\graphicspath{{figures/}}

%%%%%%%%%%%%%%%%%%%%%%%%%%%%%%%%%%%%%%%%%%%%%%%%%%%%%%%%%%%%%%%%%%%%%%%
% Problem counter
\newcounter{Problem}
\setcounter{Problem}{0}

%%%%%%%%%%%%%%%%%%%%%%%%%%%%%%%%%%%%%%%%%%%%%%%%%%%%%%%%%%%%%%%%%%%%%%%
% Definitions
\def\LabSolutions{\clearpage \newpage \begin{center} {\Large \it Solutions} \end{center} \setcounter{Problem}{0}}
\def\QuizSolutions{\newpage \begin{center} {\Large \it Solutions} \end{center} \setcounter{Problem}{0}}
\def\degree{\textdegree}
\def\grade#1{\begin{flushright} {\small [#1]}\\ \end{flushright} \vspace{-10pt}}
\def\codecolor{red!50!black}
\def\code#1{\textcolor{\codecolor}{\tt #1}}
\def\examname#1{%
                \ifnum\value{page}>1%
                    \newpage%
                \else%
                    \vspace*{5pt}%
                \fi%
                \large \textbf{#1} \setcounter{Problem}{0}\vspace{10pt}}
\def\topic#1{\par\needspace{2\baselineskip} \noindent \textsl{\footnotesize #1}}

 
%%%%%%%%%%%%%%%%%%%%%%%%%%%%%%%%%%%%%%%%%%%%%%%%%%%%%%%%%%%%%%%%%%%%%%%
% Environments
\newenvironment{problem}%
     {\stepcounter{Problem}%
      \begin{list}{\textbf{\arabic{Problem}}.~}{}%
      \item}%
     {\end{list}\vspace*{5pt}}

\newenvironment{solution}%
     {\indent \textit{Solution} \newline}%
     {\begin{flushright}$\blacksquare$\end{flushright}}

\newenvironment{preamble}%
     {\vspace*{1em}\begin{mdframed}[leftmargin=1cm,rightmargin=1cm]}%
     {\end{mdframed}\vspace*{1em}}

\newenvironment{multchoice}%
     {\begin{enumerate} \addtolength{\leftskip}{2em} \renewcommand{\labelenumi}{(\alph{enumi})}}
     {\end{enumerate}}

\newenvironment{formulaitem}%
     {\setlength{\leftmargini}{1.5em}\begin{itemize}%
      \setlength\itemindent{-\itemindent}%
      \renewcommand{\labelitemi}{$\rightarrow$}}%
     {\end{itemize}}


%%%%%%%%%%%%%%%%%%%%%%%%%%%%%%%%%%%%%%%%%%%%%%%%%%%%%%%%%%%%%%%%%%%%%%%
% New theorems
\newtheorem{theorem}{Theorem}


\makeatletter

%%%%%%%%%%%%%%%%%%%%%%%%%%%%%%%%%%%%%%%%%%%%%%%%%%%%%%%%%%%%%%%%%%%%%%%
%% New commands
\newcommand*{\course}[1]{\gdef\@course{#1}}
\newcommand*{\coursecode}[1]{\gdef\@coursecode{#1}}
\newcommand*{\term}[1]{\gdef\@term{#1}}
\newcommand*{\instructor}[1]{\gdef\@instructor{#1}}
\newcommand*{\lqnumber}[1]{\gdef\@lqnumber{#1}}
\newcommand*{\labtitle}[1]{\gdef\@labtitle{#1}}
\newcommand*{\quizversion}[1]{\gdef\@quizversion{#1}}
\newcommand*{\probleminfo}[1]{\noindent \textsl{\footnotesize #1}}

% Title header for labs
\newcommand\makelabtitle{%
  \begin{flushleft}%
  {\scshape \@coursecode~\@course~-- University of Alberta}\\%
  {\scshape \@term~-- Labs -- \@instructor}\\%
  {\scshape Authors: Carlos Contreras and Philippe Gaudreau}%
  \end{flushleft}%
  \begin{center}%
  {\Large \bf \@lqnumber:~\@labtitle}%
  \end{center}%
  \thispagestyle{empty}%
  \global\let\@course\@empty%
  \global\let\@labtitle\@empty%
}

% Title header for quizzes
\newcommand\makequiztitle{%
  \begin{flushleft}%
  {\scshape \@coursecode~\@course~-- University of Alberta}\\%
  {\scshape \@term~-- Labs -- \@instructor}\\%
  \end{flushleft}%
  \begin{center}%
  {\Large \bf \@lqnumber} \marginpar{\tiny\tt [\@quizversion]}%
  \end{center}%
  \thispagestyle{empty}%
  \global\let\@course\@empty%
  \global\let\@quizversion\@empty%
}

%%%%%%%%%%%%%%%%%%%%%%%%%%%%%%%%%%%%%%%%%%%%%%%%%%%%%%%%%%%%%%%%%%%%%%%
% Fancy header package
\fancyhead[L]{\small {\scshape \@coursecode~-- \@lqnumber~-- \@term~-- \@instructor}}
\pagestyle{fancy}

\makeatother


\usepackage{hyperref}
\usepackage{cancel}


\begin{document}

\course{Differential Equations}
\coursecode{MATH 201}
\term{Winter 2018}
\instructor{Carlos Contreras}
\lqnumber{Lab 5}
\labtitle{Power series solutions: regular points}
\makelabtitle



\topic{Recurrence relation and first terms}

\begin{problem}
{Find a recurrence relation and the first five non-zero terms in the power series approximation for the given initial value problem}
\begin{equation*}
y'' + (x+2)y = 0, \quad y(0)=1, \quad y'(0)=1.
\end{equation*}
\end{problem}


\begin{problem}
{Find a recurrence relation and the first four non-zero terms in the power series approximation about $x=0$}
\begin{equation*}
z^{\prime \prime} - x^2 z^{\prime} - xz =0.
\end{equation*}
\end{problem}


\begin{problem}
{Find a recurrence relation and the first four non-zero terms in the power series approximation about $x=0$}
\begin{equation*}
z^{\prime \prime} - x^2 z^{\prime} - xz =x^2.
\end{equation*}
\end{problem}


\topic{First terms}

\begin{problem}
Find at least the first four nonzero terms in a power series expansion about $x=0$ for the solution to the given initial value problem,
\begin{equation*}
(x^2-x+1)y^{\prime \prime} - y^{\prime} -y=x^{2},\quad y(0)=0, y^{\prime}(0)=1.
\end{equation*}
\end{problem}

\topic{Full power series}

\begin{problem}
{ Find a power series expansion about $x =0$ for a general solution to the given differential equation. Your answer should include a general formula for the coefficients}
\begin{equation*}
z^{\prime \prime} - x^2 z^{\prime} - xz =0.
\end{equation*}
\end{problem}




\begin{problem}
{ Find a power series expansion about $x =0$ for a general solution to the given differential equation. Your answer should include a general formula for the coefficients}
\begin{equation*}
(x^2+1)y^{\prime \prime} - x y^{\prime} + y =0.
\end{equation*}
\end{problem}



\topic{Cauchy product}

\begin{problem}
Find at least the first four nonzero terms in a power series expansion about $x=0$ for the solution to the given initial value problem,
\begin{equation*}
y'' + x y' + e^{x} y =0, \quad y(0) =1, y'(0)=1.
\end{equation*}
\end{problem}




\begin{problem}
Find at least four non-zero terms in the power series expansion to the initial value problem
\[y''-(\sin x) y = 0, \quad y(\pi)=1, y'(\pi)=0.\]
\end{problem}
 
 
 
 




% \begin{problem}
% Find at least the first four nonzero terms in a power series expansion about $x=0$ for the solution to the given initial value problem,
% \begin{equation*}
% y^{\prime} - e^{x} y =0, \quad y(0) =1.
% \end{equation*}
% \end{problem}
% 
% 
% \begin{problem}
% {Determine the first three nonzero terms in the Taylor polynomial approximations for the given initial value problem}
% \begin{equation*}
% y^{\prime} = x^2 + y^2, \quad y(0)=1.
% \end{equation*}
% \end{problem}
% 
% 
% 
% \begin{problem}
% {Find a power series expansion about $x =0$ for a general solution to the given differential equation. Your answer should include a general formula for the coefficients}
% \begin{equation*}
% y^{\prime} - 2xy =0.
% \end{equation*}
% \end{problem}

















%%%%%%%%%%%%%%%%%%%%%%%%%%%%%%%%%%%%%%%%%%%%%%%%%%%%%%%%%%%%%%%%%%%%%%%%%%%%%%%%%%%%%%%%%%%%%%%%%%%



\LabSolutions




Theory and problems from: Nagel, Saff \& Sneider, \textit{Fundamentals of Differential Equations}, Eighth Edition, Adisson--Wesley.


\begin{preamble}
\begin{formulaitem}
\item A \textbf{power series } of $f(x)$ about $x_{0}$ is an expansion of the form
\[f(x)=\sum_{n=0}^{\infty}a_{n}(x-x_{0})^{n}.\]

\item \begin{theorem}[Ratio test]
If, for $n$ large, the coefficients $a_{n}\neq 0$ satisfy
\[\lim_{n\rightarrow \infty}\left|\frac{a_{n+1}}{a_{n}}\right|=L \quad(0\leq L\leq \infty),\]
then the radius of convergence is $L$.
\end{theorem}

\item Recall also that we can \textit{differentiate} and \textit{integrate} power series
\[f'(x)=\sum_{n=1}^{\infty}na_{n}(x-x_{0})^{n-1},\]
and
\[\int f(x) dx=\sum_{n=0}^{\infty}\frac{a_{n}}{n+1}(x-x_{0})^{n+1} + C,\]
with the same radius of convergence of $f(x)$.

\item \textbf{Cauchy product for series}
\begin{equation*}
\left(\sum_{n=0}^{\infty} a_{n} x^n \right)\left(\sum_{n=0}^{\infty} b_{n} x^n \right) =\sum_{n=0}^{\infty} c_{n} x^n, \quad \text{where,} \quad 
c_{n} = \sum_{k=0}^{n} a_{k} b_{n-k}.
\end{equation*}

\end{formulaitem}
\end{preamble}


%------------------------------------------------------------------------------
\begin{problem}
{Find a recurrence relation and the first five non-zero terms in the power series approximation for the given initial value problem}
\begin{equation*}
y'' + (x+2)y = 0, \quad y(0)=1, \quad y'(0)=1.
\end{equation*}
\end{problem}
\begin{solution}
Assume a power solution of the form
\begin{equation*}
y(x) = \sum_{n=0}^{\infty} a_{n} x^n,
\end{equation*}
with derivatives
\begin{eqnarray*}
y^{\prime}(x) & = & \sum_{n=1}^{\infty} n a_{n} x^{n-1}, \\
y^{\prime \prime}(x) & = & \sum_{n=2}^{\infty} (n-1)n a_{n} x^{n-2}.
\end{eqnarray*}
Using the equation we have
\begin{gather*}
\sum_{n=2}^{\infty}n(n-1) a_{n} x^{n-2} + (x+2)\sum_{n=0}^{\infty} a_{n} x^n = 0,\\
\Rightarrow \sum_{n=2}^{\infty}n(n-1) a_{n} x^{n-2} + \sum_{n=0}^{\infty} a_{n} x^{n+1} + \sum_{n=0}^{\infty} 2a_{n} x^n = 0,
\end{gather*}
change the exponent in all summations to $n$
\begin{equation*}
\Rightarrow \sum_{n=0}^{\infty}(n+2)(n+1) a_{n+2} x^{n} + \sum_{n=1}^{\infty} a_{n-1} x^{n} + \sum_{n=0}^{\infty} 2a_{n} x^n = 0,
\end{equation*}
and combine the summations leaving the first extra terms in two of the summations
\begin{equation*}
\Rightarrow 2a_{2} + 2 a_{0} + \sum_{n=1}^{\infty}\left[(n+2)(n+1) a_{n+2} + a_{n-1} + 2a_{n}\right] x^n = 0.
\end{equation*}
Matching left hand side and right hand side
\begin{equation*}
\begin{cases}
2a_{2} +2a_{0}=0 \\
(n+2)(n+1)a_{n+2} + a_{n-1} + 2a_{n} =0, \quad n\geq1.
\end{cases}
\end{equation*}
Using the the initial conditions we find $a_{0}=y(0)=1$ and $a_{1}=y'(0)=1$. Thus, the recursive relation is given by
\begin{equation*}
\boxed{\begin{cases}
a_{0}=1, \, a_{1}=1,\, a_{2} =-1, \\
a_{n+2} = -\dfrac{a_{n-1} + 2a_{n}}{(n+2)(n+1)}, \quad n\geq 1.
\end{cases}
}
\end{equation*}
Now, we find the first 5 non-zero terms using the recurrence relation
\begin{equation*}
a_{0}=1, \; a_{1}=1,\; a_{2} =-1,\; a_{3} = -\frac{2+1}{3\cdot2}= -\frac{1}{2}, \; a_{4} = -\frac{2(-1)+1}{4\cdot3}= \frac{1}{12}.
\end{equation*}
Finally, the power series solution with the first five non-zero terms is
\begin{equation*}
\boxed{y(x)=1+x-x^{2} -\frac{1}{2}x^{3} +\frac{1}{12}x^{4}+\cdots}
\end{equation*}
\end{solution}




%------------------------------------------------------------------------------
\begin{problem}
{Find a recurrence relation and the first four non-zero terms in the power series approximation about $x=0$}
\begin{equation*}
z^{\prime \prime} - x^2 z^{\prime} - xz =0.
\end{equation*}
\end{problem}
\begin{solution}
We expand the solution $z(t)$ into the following power series about $x=0$.

\begin{equation*}
z(x) = \sum_{n=0}^{\infty} a_{n} x^n
\end{equation*}
Taking the first two derivatives, we have:

\begin{eqnarray*}
z^{\prime}(x) & = & \sum_{n=1}^{\infty} n a_{n} x^{n-1} \\
z^{\prime \prime}(x) & = & \sum_{n=2}^{\infty} (n-1)n a_{n} x^{n-2} \\
\end{eqnarray*}

Substituting these equations into our ODE, we obtain:

\begin{eqnarray*}
0 & = & \sum_{n=2}^{\infty} (n-1)n a_{n} x^{n-2} - x^2 \sum_{n=1}^{\infty} n a_{n} x^{n-1} - x \sum_{n=0}^{\infty} a_{n} x^n \\
& = & \sum_{n=2}^{\infty} (n-1)n a_{n} x^{n-2} - \sum_{n=1}^{\infty} n a_{n} x^{n+1} -  \sum_{n=0}^{\infty} a_{n} x^{n+1} \\
& = & \sum_{n=-1}^{\infty} (n+2)(n+3) a_{n+3} x^{n+1} - \sum_{n=1}^{\infty} n a_{n} x^{n+1} -  \sum_{n=0}^{\infty} a_{n} x^{n+1} \\
& = & (1)(2)a_{2} + (2)(3)a_{3}x - a_{0}x+ \sum_{n=1}^{\infty} (n+2)(n+3) a_{n+3} x^{n+1} - \sum_{n=1}^{\infty} n a_{n} x^{n+1} -  \sum_{n=1}^{\infty} a_{n} x^{n+1} \\
& = & 2a_{2} + (6a_{3} - a_{0})x+ \sum_{n=1}^{\infty} \left[ (n+2)(n+3) a_{n+3}-n a_{n}-a_{n} \right] x^{n+1}\\
& = & 2a_{2} + (6a_{3} - a_{0})x+ \sum_{n=1}^{\infty} \left[ (n+2)(n+3) a_{n+3}-(n+1)a_{n} \right] x^{n+1}\\
\end{eqnarray*}
Since this expression must be true for all $x$, we must have:

\begin{eqnarray*}
2a_{2} & = & 0, \\
6a_{3} - a_{0} & = & 0, \\
(n+2)(n+3) a_{n+3}-(n+1)a_{n} & = & 0, \quad n\geq 1.
\end{eqnarray*}
This leads to the following recurrence relation:

\begin{eqnarray*}
a_{2} & = & 0 \\
a_{3} & = & \dfrac{a_{0}}{6} \\
a_{n+3} & = & \left(\dfrac{(n+1)}{(n+2)(n+3)}\right)a_{n}
\end{eqnarray*}

Let's find the first coefficients

\begin{eqnarray*}
a_{0} & = & a_{0} \\
a_{1} & = & a_{1}\\
a_{2} & = & 0 \\
a_{3} & = & \dfrac{a_{0}}{6}  \\
a_{4} & = & \left(\dfrac{2}{(3)(4)}\right)a_{1}  =  \left(\dfrac{1}{6}\right)a_{1}  
\end{eqnarray*}

Hence, the solution to the this ODE is given by:

\begin{equation*}
\boxed{y(x) = a_{0} + a_{1}x + \frac{a_{0}}{6}x^{3} + \frac{a_{1}}{6}x^{4}+\cdots}
\end{equation*}
\end{solution}



%------------------------------------------------------------------------------
\begin{problem}
{Find a recurrence relation and the first four non-zero terms in the power series approximation about $x=0$}
\begin{equation*}
z^{\prime \prime} - x^2 z^{\prime} - xz =x^2.
\end{equation*}
\end{problem}

\begin{solution}
We expand the solution $z(t)$ into the following power series about $x=0$.

\begin{equation*}
z(x) = \sum_{n=0}^{\infty} a_{n} x^n
\end{equation*}
Taking the first two derivatives, we have:

\begin{eqnarray*}
z^{\prime}(x) & = & \sum_{n=1}^{\infty} n a_{n} x^{n-1} \\
z^{\prime \prime}(x) & = & \sum_{n=2}^{\infty} (n-1)n a_{n} x^{n-2} \\
\end{eqnarray*}

Substituting these equations into our ODE (be aware of the $x^{2}$ term in the right hand side), we obtain:

\begin{eqnarray*}
x^{2} & = & \sum_{n=2}^{\infty} (n-1)n a_{n} x^{n-2} - x^2 \sum_{n=1}^{\infty} n a_{n} x^{n-1} - x \sum_{n=0}^{\infty} a_{n} x^n \\
& = & \sum_{n=2}^{\infty} (n-1)n a_{n} x^{n-2} - \sum_{n=1}^{\infty} n a_{n} x^{n+1} -  \sum_{n=0}^{\infty} a_{n} x^{n+1} \\
& = & \sum_{n=-1}^{\infty} (n+2)(n+3) a_{n+3} x^{n+1} - \sum_{n=1}^{\infty} n a_{n} x^{n+1} -  \sum_{n=0}^{\infty} a_{n} x^{n+1} \\
& = & (1)(2)a_{2} + (2)(3)a_{3}x - a_{0}x+ \sum_{n=1}^{\infty} (n+2)(n+3) a_{n+3} x^{n+1} - \sum_{n=1}^{\infty} n a_{n} x^{n+1} -  \sum_{n=1}^{\infty} a_{n} x^{n+1} \\
& = & 2a_{2} + (6a_{3} - a_{0})x+ \sum_{n=1}^{\infty} \left[ (n+2)(n+3) a_{n+3}-n a_{n}-a_{n} \right] x^{n+1}\\
& = & 2a_{2} + (6a_{3} - a_{0})x+ [12 a_{4}-3a_{1}]x^{2} + \sum_{n=2}^{\infty} \left[ (n+2)(n+3) a_{n+3}-(n+1)a_{n} \right] x^{n+1}.
\end{eqnarray*}
Since this expression must be true for all $x$, we must have:

\begin{eqnarray*}
2a_{2} & = & 0, \\
6a_{3} - a_{0} & = & 0, \\
12 a_{4}-2a_{1} & = & 1, \\
(n+2)(n+3) a_{n+3}-(n+1)a_{n} & = & 0, \quad n\geq 2.
\end{eqnarray*}
This leads to the following recurrence relation:

\begin{eqnarray*}
a_{2} & = & 0, \\
a_{3} & = & \dfrac{a_{0}}{6}, \\
a_{4} & = & \dfrac{1}{12} + \dfrac{2a_{1}}{12}, \\
 a_{n+3} & = & \left(\dfrac{(n+1)}{(n+2)(n+3)}\right)a_{n}, \quad n\geq 2.
\end{eqnarray*}

Let's find the first coefficients

\begin{eqnarray*}
a_{0} & = & a_{0} \\
a_{1} & = & a_{1}\\
a_{2} & = & 0 \\
a_{3} & = & \dfrac{a_{0}}{6}  \\
a_{4} & = & \dfrac{1}{12} + \dfrac{a_{1}}{6} 
\end{eqnarray*}

Hence, the solution to the this ODE is given by:

\begin{equation*}
\boxed{y(x) = a_{0} + a_{1}x + \frac{a_{0}}{6}x^{3} + \left(\frac{1}{12} + \frac{a_{1}}{6}\right) x^{4}+\cdots}
\end{equation*}
Compare to the previous solution.

\textbf{Extra work:} Note that we can find more terms
\begin{eqnarray*}
a_{5} & = & \dfrac{3}{4\cdot5}a_{2} = 0  \\
a_{6} & = & \dfrac{4}{5\cdot6}a_{3} = \dfrac{4}{5\cdot 6\cdot 6}a_{0} = \dfrac{1}{45}a_{0}  \\
a_{7} & = & \dfrac{5}{6\cdot7}a_{4} = \dfrac{5}{6\cdot 7\cdot 12} + \dfrac{5}{6\cdot 7\cdot 6}a_{1} = \dfrac{5}{504} + \dfrac{5}{252}a_{1} 
\end{eqnarray*}
in the separate this solution in terms of the homogeneous and particular solutions
\begin{equation*}
\boxed{y(x) = a_{0}\underbrace{\left(1+\frac{1}{6}x^{3}+ \frac{1}{45}x^{6} \cdots\right)}_{\text{1st lin. ind. sol.}} + a_{1}\underbrace{\left(x + \frac{1}{6}x^{4}+\frac{5}{252}x^{7}+\cdots\right)}_{\text{2nd lin. ind. sol.}} + \underbrace{\left(\frac{1}{12}x^{4}+ \frac{5}{504}x^{7}+\cdots\right)}_{\text{part. sol.}}.} 
\end{equation*}
\end{solution}



%------------------------------------------------------------------------------
\begin{problem}
Find at least the first four nonzero terms in a power series expansion about $x=0$ for the solution to the given initial value problem,
\begin{equation*}
(x^2-x+1)y^{\prime \prime} - y^{\prime} -y=x^{2},\quad y(0)=0, y^{\prime}(0)=1.
\end{equation*}
\end{problem}
\begin{solution}
\indent Let $$y(x)= \displaystyle \sum_{n=0}^{\infty} a_{n} x^n = a_{0} + a_{1}x +a_{2}x^{2}+a_{3}x^{3} + a_{4}x^{4} + \cdots $$. Taking the first two derivatives, we have

\begin{align*}
y^{\prime}(x) & = a_{1} + 2 a_{2}x+ 3a_{3}x^{2} + 4a_{4}x^{3} + \cdots \\
y^{\prime \prime}(x) & = 2a_{2} + 6a_{3}x + 12a_{4}x^{2} + \cdots\\
\end{align*}
Using the initial conditions we see that $a_{0}=0$ and $a_{1}=1$.
Substituting these result into our equation, we obtain
\begin{align*}
x^{2} & =  (x^2-x+1) \cdot (2a_{2} + 6a_{3}x + 12a_{4}x^{2} + \cdots) \\
& \qquad - (1 + 2 a_{2}x+ 3a_{3}x^{2} + 4a_{4}x^{3} + \cdots) - (x +a_{2}x^{2}+a_{3}x^{3} + a_{4}x^{4} + \cdots) \\
& = (2a_{2}x^{2} + 6a_{3}x^{3} + 12a_{4}x^{4} + \cdots) + (- 2a_{2}x - 6a_{3}x^{2} - 12a_{4}x^{3} + \cdots) + (2a_{2} + 6a_{3}x + 12a_{4}x^{2} + \cdots) \\
& \qquad + ( -1 - 2 a_{2}x- 3a_{3}x^{2} - 4a_{4}x^{3} + \cdots) + (-x -a_{2}x^{2}-a_{3}x^{3} - a_{4}x^{4} + \cdots) \\
& = (-1 + 2a_{2})+ (-1-4a_{2}+6a_{3})x + (a_{2}-9a_{3}+12a_{4})x^{2}+\cdots 
\end{align*}

Matching terms of each side of this equation



\begin{align*}
0 = 2a_{2}-1  & \Rightarrow a_{2} = \frac{1}{2} \\
0 = -1-4\cancelto{\frac{1}{2}}{a_{2}}+6a_{3}  & \Rightarrow a_{3} = \frac{1}{2} \\
1 = \cancelto{\frac{1}{2}}{a_{2}}-9\cancelto{\frac{1}{2}}{a_{3}}+12a_{4}  & \Rightarrow a_{4} = \frac{5}{12}
\end{align*}

Thus, the solution with the first four non-zero terms is given by

\begin{equation*}
\boxed{y(x) = x +\dfrac{1}{2}x^2+\dfrac{1}{2}x^3 + \dfrac{5}{12}x^4 + \ldots}
\end{equation*}
\end{solution}



%------------------------------------------------------------------------------
\begin{problem}
{ Find a power series expansion about $x =0$ for a general solution to the given differential equation. Your answer should include a general formula for the coefficients}
\begin{equation*}
z^{\prime \prime} - x^2 z^{\prime} - xz =0.
\end{equation*}
\end{problem}

\begin{solution}
We expand the solution $z(t)$ into the following power series about $x=0$.

\begin{equation*}
z(x) = \sum_{n=0}^{\infty} a_{n} x^n
\end{equation*}
Taking the first two derivatives, we have:

\begin{eqnarray*}
z^{\prime}(x) & = & \sum_{n=1}^{\infty} n a_{n} x^{n-1} \\
z^{\prime \prime}(x) & = & \sum_{n=2}^{\infty} (n-1)n a_{n} x^{n-2} \\
\end{eqnarray*}

Substituting these equations into our ODE, we obtain:

\begin{eqnarray*}
0 & = & \sum_{n=2}^{\infty} (n-1)n a_{n} x^{n-2} - x^2 \sum_{n=1}^{\infty} n a_{n} x^{n-1} - x \sum_{n=0}^{\infty} a_{n} x^n \\
& = & \sum_{n=2}^{\infty} (n-1)n a_{n} x^{n-2} - \sum_{n=1}^{\infty} n a_{n} x^{n+1} -  \sum_{n=0}^{\infty} a_{n} x^{n+1} \\
& = & \sum_{n=-1}^{\infty} (n+2)(n+3) a_{n+3} x^{n+1} - \sum_{n=1}^{\infty} n a_{n} x^{n+1} -  \sum_{n=0}^{\infty} a_{n} x^{n+1} \\
& = & (1)(2)a_{2} + (2)(3)a_{3}x - a_{0}x+ \sum_{n=1}^{\infty} (n+2)(n+3) a_{n+3} x^{n+1} - \sum_{n=1}^{\infty} n a_{n} x^{n+1} -  \sum_{n=1}^{\infty} a_{n} x^{n+1} \\
& = & 2a_{2} + (6a_{3} - a_{0})x+ \sum_{n=1}^{\infty} \left[ (n+2)(n+3) a_{n+3}-n a_{n}-a_{n} \right] x^{n+1}\\
& = & 2a_{2} + (6a_{3} - a_{0})x+ \sum_{n=1}^{\infty} \left[ (n+2)(n+3) a_{n+3}-(n+1)a_{n} \right] x^{n+1}\\
\end{eqnarray*}
Since this expression must be true for all $x$, we must have:

\begin{eqnarray*}
2a_{2} & = & 0 \\
6a_{3} - a_{0} & = & 0 \\
(n+2)(n+3) a_{n+3}-(n+1)a_{n} & = & 0
\end{eqnarray*}
This leads to the following recurrence relation:

\begin{eqnarray*}
a_{2} & = & 0 \\
a_{3} & = & \dfrac{a_{0}}{6} \\
 a_{n+3} & = & \left(\dfrac{(n+1)}{(n+2)(n+3)}\right)a_{n}
\end{eqnarray*}

Let's try to find a pattern for the coefficients.

\begin{eqnarray*}
a_{0} & = & a_{0} \\
a_{1} & = & a_{1}\\
a_{2} & = & 0 \\
a_{3} & = & \dfrac{a_{0}}{6}  = \dfrac{a_{0}}{3!}  \\
a_{4} & = & \left(\dfrac{2}{(3)(4)}\right)a_{1}  =  \left(\dfrac{2^2}{4!}\right)a_{1}  \\
a_{5} & = & 0 \\
a_{6} & = & \left(\dfrac{(4)}{(5)(6)}\right)a_{3} = \left(\dfrac{(4)}{(2)(3)(5)(6)}\right)a_{0} = \left(\dfrac{4^2}{6!}\right)a_{0} \\
a_{7} & = & \left(\dfrac{(5)}{(6)(7)}\right)a_{4} = \left(\dfrac{(2 \cdot 5)^2}{7!}\right) a_{1} \\
a_{8} & = & \left(\dfrac{(6)}{(7)(8)}\right)a_{5} = 0 \\
a_{9} & = & \left(\dfrac{(7)}{(8)(9)}\right)a_{6} = \left(\dfrac{(7)}{(8)(9)}\right)\left(\dfrac{4^2}{6!}\right)a_{0} =\left(\dfrac{(1 \cdot 4 \cdot 7)^2}{9!}\right)a_{0}  \\
a_{10} & = & \left(\dfrac{(8)}{(9)(10)}\right)a_{7} = \left(\dfrac{(8)}{(9)(10)}\right)\left(\dfrac{(2 \cdot 5)^2}{7!}\right) a_{1} = \left(\dfrac{(2 \cdot 5 \cdot 8)^2}{10!}\right)a_{1}
\end{eqnarray*}

Although complicated ,the pattern is starting to get more obvious:

we have:

\begin{eqnarray*}
a_{0} & = & a_{0}\\
a_{1} & = & a_{1}\\
a_{3n} & = & \dfrac{\left[1 \cdot 4 \cdot 7 \cdots (3n-2) \right]^2}{(3n)!}a_{0} \quad n=1,2, \ldots \\
a_{3n+1} & = & \dfrac{\left[2 \cdot 5 \cdot 8 \cdots (3n-1) \right]^2}{(3n+1)!}a_{1} \quad n=1,2, \ldots \\
a_{3n+2} & = & 0, \quad n=0,1,2, \ldots \\
\end{eqnarray*}

Hence, the solution to the this ODE is given by:

\begin{equation*}
\boxed{y(x) = a_{0} \left(1 + \sum_{n=1}^{\infty} \left[ \dfrac{\left[1 \cdot 4 \cdot 7 \cdots (3n-2) \right]^2}{(3n)!} \right] x^{3n} \right) + a_{1} \left(x+ \sum_{n=1}^{\infty} \left[\dfrac{\left[2 \cdot 5 \cdot 8 \cdots (3n-1) \right]^2}{(3n+1)!} \right] x^{3n+1} \right).}
\end{equation*}
\end{solution}




%------------------------------------------------------------------------------
\begin{problem}
{ Find a power series expansion about $x =0$ for a general solution to the given differential equation. Your answer should include a general formula for the coefficients}
\begin{equation*}
(x^2+1)y^{\prime \prime} - x y^{\prime} + y =0.
\end{equation*}
\end{problem}
\begin{solution}
 We expand the solution $y(x)$ into the following power series about $x=0$.
\begin{equation*}
y(x) = \sum_{n=0}^{\infty} a_{n} x^n
\end{equation*}
Taking the first two derivatives, we have
\begin{gather*}
y^{\prime}(x) =  \sum_{n=1}^{\infty} n a_{n} x^{n-1}, \\
y^{\prime \prime}(x) = \sum_{n=2}^{\infty} (n-1)n a_{n} x^{n-2}.
\end{gather*}
Using the equation we have
\begin{gather*}
(x^2 + 1)\sum_{n=2}^{\infty}n(n-1)a_{n}x^{n-2} -x \sum_{n=1}^{\infty} na_{n}x^{n-1} + \sum_{n=0}^{\infty} a_{n}x^{n} = 0, \\
\Rightarrow \sum_{n=2}^{\infty}n(n-1)a_{n}x^{n} + \underbrace{\sum_{n=2}^{\infty}n(n-1)a_{n}x^{n-2}}_{n-2\rightarrow n} - \sum_{n=1}^{\infty} na_{n}x^{n} + \sum_{n=0}^{\infty} a_{n}x^{n} = 0, \\
\Rightarrow \sum_{n=2}^{\infty}n(n-1)a_{n}x^{n} + \sum_{n=0}^{\infty}(n+2)(n+1)a_{n+2}x^{n} - \sum_{n=1}^{\infty} na_{n}x^{n} + \sum_{n=0}^{\infty} a_{n}x^{n} = 0, \\
\Rightarrow 2a_{2} + 6 a_{3}x - a_{1}x + a_{0} + a_{1}x + \sum_{n=2}^{\infty}[(n+2)(n+1)a_{n+2} + (n-1)^{2}a_{n}]x^{n} =0 \\
\Rightarrow (2a_{2} + a_{0}) + 6 a_{3} x + \sum_{n=2}^{\infty}[(n+2)(n+1)a_{n+2} + (n-1)^{2}a_{n}]x^{n} =0.
\end{gather*}
Matching both side sides  of the equation
\begin{equation*}
\begin{array}{cccc}
a_{0},a_{1}\in \mathbb{R}, & a_{2}=-\frac{1}{2}a_{0}, & a_{3}=0, & a_{n+2}=-\dfrac{(n-1)^{2}}{(n+2)(n+1)}a_{n},\,n\geq 2.
\end{array}
\end{equation*}
Use the recursive relation to find the pattern
\begin{align*}
n=2, & \qquad a_{4}=-\dfrac{1}{4\cdot 3} a_{2} = \dfrac{1}{4\cdot 3\cdot 2}a_{0}\\
n=3, & \qquad a_{5}=-\dfrac{2^{2}}{5\cdot 4} a_{3} = 0\\
n=4, & \qquad a_{6}=-\dfrac{3^{2}}{6\cdot 5} a_{4} = -\dfrac{(3\cdot 1)^{2}}{6!}a_{0}\\
n=5, & \qquad a_{7}=-\dfrac{4^{2}}{7\cdot 6} a_{5} = 0\\
n=6, & \qquad a_{8}=-\dfrac{5^{2}}{8\cdot 7} a_{6} = -\dfrac{(5\cdot 3\cdot 1)^{2}}{8!}a_{0}
\end{align*}
We can see (with extra effort to find the pattern) that the even coefficients are given by
\begin{align*}
a_{2n} & = \frac{(-1)^{n}((2n-3)(2n-5)\cdots 1)^{2}}{(2n)!}a_{0} 
% \qquad \text{multiply by} \quad \frac{(2n-2)(2n-4)\cdots}{(2n-2)(2n-4)\cdots} \\
%        & = \frac{(-1)^{n}(2n-2)!}{2^{n-2}(n-1)!(2n)!}a_{0} = \frac{(-1)^{n}(2n-2)!}{2^{n-2}(n-1)!(2n)(2n-1)(2n-2)!}a_{0} \\
%        & = \frac{(-1)^{n}}{2^{n}(2n-1)n!}a_{0}.
\end{align*}
Verify that this is true with $a_{8}$, for example.

Thus, the general solution is given by
\begin{equation*}
\boxed{y(x)=a_{0}\sum_{n=0}^{\infty} \frac{(-1)^{n}((2n-3)(2n-5)\cdots 1)^{2}}{(2n)!} x^{2n} + a_{1}x.}
\end{equation*}




\end{solution}



%------------------------------------------------------------------------------
\begin{problem}
Find at least the first four non-zero terms in a power series expansion about $x=0$ for the solution to the given initial value problem,
\begin{equation*}
y'' + x y' + e^{x} y =0, \quad y(0) =1, y'(0)=1.
\end{equation*}
\end{problem}
\begin{solution}
We expand the solution $y(x)$ with a few terms into the following power series about $x=0$
\begin{equation*}
y(x) =  a_{0} + a_{1}x + a_{2}x^{2} + a_{3}x^{3} + a_{4}x^{4}+\cdots 
\end{equation*}
Taking the first two derivatives, we have
\begin{gather*}
y^{\prime}(x) =  a_{1} + 2a_{2}x + 3a_{3}x^{2} + 4a_{4}x^{3}+\cdots \\
y^{\prime \prime}(x) = 2a_{2} + 6a_{3}x + 12a_{4}x^{2}+ 20a_{5}x^{3}\cdots 
\end{gather*}
Using initial conditions we have
\begin{equation}
a_{0}=1,\qquad a_{1} = 1.
\end{equation}
First we need the product of functions $e^{x}y$ written in power series
\begin{align*}
e^{x}y(x) &= \sum_{n=0}^{\infty}\frac{1}{n!}x^{n} \sum_{n=0}^{\infty}a_{n}x^{n} = \left(1 + \frac{1}{1!}x + \frac{1}{2!}x^{2} + \frac{1}{3!}x^{3} + \dots\right)\left(1 + x + a_{2}x^{2} + a_{3}x^{3}+ \cdots\right) \\
          &= 1 + (1 + 1)x + \left(a_{2}+ 1 + \frac{1}{2!}1\right)x^{2} + \left(a_{3} + a_{2} + \frac{1}{2!} + \frac{1}{3!}\right)x^{3}+ \cdots
\end{align*}
Using the equation we have
\begin{multline*}
(2a_{2} + 6a_{3}x + 12a_{4}x^{2} + 20a_{5}x^{3} + \cdots ) + (a_{1}x + 2a_{2}x^{2} + 3a_{3}x^{3} +\cdots) \\+ (1 + 2x + (a_{2}+ \tfrac{3}{2})x^{2}+ (a_{3} + a_{2} + \tfrac{1}{2} + \tfrac{1}{6})x^{3} + \cdots) = 0.
\end{multline*}
Matching coefficients on each side we have
\begin{align*}
& 2a_{2} + 1= 0 \Rightarrow a_{2}=-\frac{1}{2}, \\
& 6a_{3} + 1  + 2 =0 \Rightarrow a_{3}=-\frac{1}{2}, \\
& 12a_{4} + 2a_{2}  + a_{2} +\frac{3}{2} =0 \Rightarrow a_{4}= 0, \\
& 20a_{5} + 3a_{3}  + a_{3} + a_{2} +\frac{1}{2} + \frac{1}{6} =0 \Rightarrow a_{5}= \frac{11}{120}.
\end{align*}
Finally, the solution with 5 non-zero terms is given by
\begin{equation*}
\boxed{y(x)= 1 + x -\frac{1}{2}x^{2} -\frac{1}{2}x^{3} + \frac{11}{120}x^{5} + \cdots}
\end{equation*}

\end{solution}



%------------------------------------------------------------------------------
\begin{problem}
Find at least four non-zero terms in the power series expansion to the initial value problem
\[y''-(\sin x) y = 0, \quad y(\pi)=1, y'(\pi)=0.\]
\end{problem}
\begin{solution}
We are given initial conditions at $x_{0}=\pi$, the trick here is to find a power series expansion around $x_{0}=\pi$, which is a regular point for the equation,
\[y(x)=\sum_{n=0}^{\infty}a_{n}(x-\pi)^{n}.\]
In order to do this we first make a change of variables. Let
\[t=x-\pi \Rightarrow x = t + \pi, \,\, \text{ and } \,\, Y(t)=y(t+\pi) \Rightarrow Y' = y', \,\, Y'' = y'',\]
in which case 
\[\sin x = \sin(t+\pi)=\sin t \cos \pi +\cos t \sin \pi = -\sin t,\]
and the initial value problem becomes
\[Y''+\sin t Y = 0, \quad Y(0)=1,\, Y'(0)=0.\]
If we try to solve without using this change of variables, the series expansion of $\sin x$ (around $x=0$) is a very bad approximation around $x=\pi$ (where we are solving for $y(x)$). 

Now, we solve this new problem using a simple power series expansion (recall that we only need a few terms)
\begin{eqnarray*}
Y(t)&=&\sum_{n=0}^{\infty}a_{n}t^{n}=a_{0}+a_{1}t+a_{2}t^{2}+a_{3}t^{3}+a_{4}t^{4}+a_{5}t^{5}+a_{6}t^{6}\dots\\
    &=&  1+a_{1}t+a_{2}t^{2}+a_{3}t^{3}+a_{4}t^{4}+a_{5}t^{5}+a_{6}t^{6}\dots\\
Y'(t)&=&\sum_{n=1}^{\infty}na_{n}t^{n-1}=a_{1}+2a_{2}t+3a_{3}t^{2}+4a_{4}t^{3}+5a_{5}t^{4}+6a_{6}t^{5}+\dots\\
    &=&  0+2a_{2}t+3a_{3}t^{2}+4a_{4}t^{3}+5a_{5}t^{4}+6a_{6}t^{5}+\dots \\
Y''(t)&=&\sum_{n=2}^{\infty}n(n-1)a_{n}t^{n-2}=2a_{2}+3\cdot2a_{3}t+4\cdot3a_{4}t^{2}+5\cdot4a_{5}t^{3}+6\cdot5a_{6}t^{4}+\dots
\end{eqnarray*}
Here $a_{0}=1$ and $a_{1}=0$. Since $a_{1}=0$, we still have three more terms to find.

For the term $\sin (t) Y(t)$ we need to expand the sine in power series and multiply it with the expansion of $Y(t)$. For that, recall the Cauchy product
\begin{equation*}
\left(\sum_{n=0}^{\infty} b_{n} x^n \right)\left(\sum_{n=0}^{\infty} d_{n} x^n \right) =\sum_{n=0}^{\infty} c_{n} x^n,
\end{equation*}
where
\begin{equation*}
c_{n} = \sum_{k=0}^{n} b_{k} d_{n-k} = b_{0}d_{n}+b_{1}d_{n-1}+b_{2}d_{n-2}+\dots+b_{n-1}d_{1}+b_{n}d_{0}.
\end{equation*}

The power series of sine around $0$ is 
\[\sin t =\sum_{n=0}^{\infty}\frac{(-1)^{n}t^{2n+1}}{(2n+1)!}=t - \frac{1}{3!}t^{3}+ \frac{1}{5!}t^{5}-\frac{1}{7!}t^{7}+\frac{1}{9!}t^{9}\dots.\]
\textbf{Be careful!} Do not confuse the terms in this expansion with the ones in $\sum_{n=0}^{\infty} b_{n}t^{n}$, they have different exponents. In our case
\[b_{0}=0,\,\, b_{1}=1, \,\,b_{2}=0, \,\,b_{3}=-\frac{1}{3!},\,\,b_{4}=0, b_{5}=\frac{1}{5!}, \dots\]
and $d_{n}=a_{n}$.
Finding the value of the coefficients $c_{n}$
\begin{eqnarray*}
c_{0} & = & b_{0}a_{0} = 0\\
c_{1} & = & b_{0}a_{1} + b_{1}a_{0} = 1 \\
c_{2} & = & b_{0}a_{2} + b_{1}a_{1} + b_{2}a_{0} = 0 \\
c_{3} & = & b_{0}a_{3} + b_{1}a_{2} + b_{2}a_{1} + b_{3}a_{0} = a_{2} - \frac{1}{3!} \\
c_{4} & = & b_{0}a_{4} + b_{1}a_{3} + b_{2}a_{2} + b_{3}a_{1} + b_{4}a_{0} = a_{3}. \\
\end{eqnarray*}
Using the equation we have
\begin{gather}
\left(2a_{2} + 6 a_{3}t +12 a_{4}t^{2}+20 a_{5}t^{3}+30 a_{6}t^{4}+\dots\right)
     +\left( 0 + t + 0 \cdot t^{2}+ \left(a_{2}-\frac{1}{3!}\right)t^{3}+\dots\right) =0,
\end{gather}
where matching coefficients multiplying $x$ with the same power we get,
\begin{align*}
  2a_{2} = 0 & \Rightarrow a_{2} = 0 \\
  6 a_{3} = -1 & \Rightarrow a_{3} = -\frac{1}{6} \\
  12 a_{4}= 0 & \Rightarrow a_{4} = 0 \\
  20 a_{5} = -a_{2} +\frac{1}{3!} & \Rightarrow a_{5} = \frac{1}{120}\\
  30 a_{6} = -a_{3} & \Rightarrow a_{6} = \frac{1}{180}.
\end{align*}
Suing this coefficients in the expansion of $Y(t)$ we have
\[Y(t)=1-\frac{1}{6}t^{3}+\frac{1}{120}t^{5}+\frac{1}{180}t^{6}+\dots\]

Thus, the solution to the initial problem is 
\[\boxed{y(x)=1-\frac{1}{6}(x-\pi)^{3}+\frac{1}{120}(x-\pi)^{5}+\frac{1}{180}(x-\pi)^{6}+\dots}\]
\end{solution}





% \begin{problem}
% { Determine the first three nonzero terms in the Taylor polynomial approximations for the given initial value problem}
% \begin{equation*}
% y^{\prime} = x^2 + y^2, \quad y(0)=1.
% \end{equation*}
% \end{problem}
% \begin{solution}
% We make the following guess for a solution:
% 
% \begin{eqnarray*}
% y(x)  & =  & \sum_{n=0}^{\infty} a_{n} x^n \\
% & =  & a_{0} +a_{1}x + a_{2}x^2 + \ldots \\
% & =  & 1 +a_{1}x + a_{2}x^2 + \ldots
% \end{eqnarray*}
% since $y(0)=1$,
% 
% Taking the derivative we obtain:
% 
% \begin{eqnarray*}
% y^{\prime}(x)  & =  & \sum_{n=1}^{\infty} n a_{n} x^{n-1} \\
% & =  & a_{1} + 2a_{2}x + 3a_{3}x^2 +  \ldots \\
% \end{eqnarray*}
% 
% Substituting this into our equation, we have:
% 
% \begin{eqnarray*}
% a_{1} + 2a_{2}x + 3a_{3}x^2 +  \ldots & = & x^2 + (1 +a_{1}x + a_{2}x^2 + \ldots )^2 \\
% & = & x^2 + (1 + 2 a_{1} x + (a_{1}^2 +2 a_{2})x^2  + \ldots) \\
% & = & 1 + 2 a_{1} x + (a_{1}^2 +2 a_{2}+1)x^2 +  \ldots\\
% \end{eqnarray*}
% 
% Since, we require these two sides to be equal we have:
% 
% \begin{eqnarray*}
% a_{1} & = & 1 \\
% 2a_{2} & = & 2 a_{1}  = 2 \quad \Rightarrow a_{2} = 1
% \end{eqnarray*}
% 
% Hence,
% 
% \begin{equation*}
% y(x) = 1 + x +x^2 + \ldots
% \end{equation*}
% \end{solution}




% 
% \begin{problem}
% { Find a power series expansion about $x =0$ for a general solution to the given differential equation. Your answer should include a general formula for the coefficients}
% \begin{equation*}
% y^{\prime} - 2xy =0.
% \end{equation*}
% \end{problem}
% \begin{solution}
% We propose the following series solution:
% 
% \begin{eqnarray*}
% y(x) & = & \sum_{n=0}^{\infty} a_{n} x^n
% \end{eqnarray*}
% 
% Taking the derivative, we obtain:
% 
% \begin{eqnarray*}
% y(x) & = & \sum_{n=1}^{\infty} n a_{n} x^{n-1} \\
% \end{eqnarray*}
% 
% Inserting these results into our equation, we obtain:
% 
% \begin{equation*}
% \sum_{n=1}^{\infty} n a_{n} x^{n-1} - \sum_{n=0}^{\infty} 2 a_{n} x^{n+1} =0
% \end{equation*}
% 
% Our goal is now to shift the index of summation of one of these sums to make the exponent of $x$ equal in both.
% Shifting the index of summation in the first sum by 2, we obtain:
% 
% \begin{equation*}
% \sum_{n=-1}^{\infty} (n+2) a_{n+2} x^{n+1} - \sum_{n=0}^{\infty} 2 a_{n} x^{n+1} =0
% \end{equation*}
% 
% Now that the exponents are equal, we want to combine these two summations.
% 
% \begin{eqnarray*}
% a_{1}  + \sum_{n=0}^{\infty} (n+2) a_{n+2} x^{n+1} - \sum_{n=0}^{\infty} 2 a_{n} x^{n+1} & = & 0 \\
% \Rightarrow \quad a_{1}  + \sum_{n=0}^{\infty} \left[ (n+2) a_{n+2} - 2 a_{n} \right] x^{n+1}  & = & 0
% \end{eqnarray*}
% 
% Since the left hand side and right hand side must agree for all $x$, we must have:
% 
% \begin{eqnarray*}
% a_{1} & = & 0\\
% (n+2) a_{n+2} - 2 a_{n} & = & 0 \Rightarrow a_{n+2} = \left(\dfrac{2}{n+2} \right)a_{n}
% \end{eqnarray*}
% 
% Let's look at a few coefficients:
% 
% \begin{eqnarray*}
% a_{0} & = & a_{0} \\
% a_{1} & = & 0\\
% a_{2} & = & \left(\dfrac{2}{2} \right)a_{0} =a_{0} \\
% a_{3} & = & 0 \\
% a_{4} & = & \left(\dfrac{2}{4} \right) \left(\dfrac{2}{2} \right)a_{0} = \dfrac{a_{0}}{(2)(1)} = \dfrac{a_{0}}{2!}  \\
% a_{5} & = & 0 \\
% a_{6} & = & \left(\dfrac{2}{6} \right)  \left(\dfrac{2}{4} \right) \left(\dfrac{2}{2} \right)a_{0} = \dfrac{a_{0}}{(3)(2)(1)} =   \dfrac{a_{0}}{3!} \\
% \end{eqnarray*}
% 
% The pattern is now obvious, it can be shown using induction that:
% 
% \begin{eqnarray*}
% a_{2n+1} & = & 0 , \quad n \geq 0\\
% a_{2n} & = & \dfrac{a_{0}}{n!} , \quad n \geq 0 \\
% \end{eqnarray*}
% 
% Hence the solution to the given differential equation is given by:
% 
% \begin{eqnarray*}
% y(x) & = & \sum_{n=0}^{\infty} a_{n} x^n \\
% & = & \sum_{n=0}^{\infty} a_{2n+1} x^{2n+1} + \sum_{n=0}^{\infty} a_{2n} x^{2n} \\
% & = & \sum_{n=0}^{\infty} \dfrac{a_{0}}{n!} x^{2n} \\
% & = & a_{0}\sum_{n=0}^{\infty} \dfrac{x^{2n}}{n!} \\
% & = & a_{0} e^{x^2}
% \end{eqnarray*}
% \end{solution}
% 
% %------------------------------------------------------------------------------
% \begin{problem}
% Find at least the first four nonzero terms in a power series expansion about $x=0$ for the solution to the given initial value problem,
% \begin{equation*}
% (x^2-x+1)y^{\prime \prime} - y^{\prime} -y=0,\quad y(0)=0, y^{\prime}(0)=1.
% \end{equation*}
% \end{problem}
% \begin{solution}
% \indent Let $$y(x)= \displaystyle \sum_{n=0}^{\infty} a_{n} x^n = a_{0} + a_{1}x +a_{2}x^{2}+a_{3}x^{3} + a_{4}x^{4} + \cdots $$. Taking the first two derivatives, we have
% 
% \begin{align*}
% y^{\prime}(x) & = a_{1} + 2 a_{2}x+ 3a_{3}x^{2} + 4a_{4}x^{3} + \cdots \\
% y^{\prime \prime}(x) & = 2a_{2} + 6a_{3}x + 12a_{4}x^{2} + \cdots\\
% \end{align*}
% Using the initial conditions we see that $a_{0}=0$ and $a_{1}=1$.
% Substituting these result into our equation, we obtain
% \begin{align*}
% 0 & =  (x^2-x+1) \cdot (2a_{2} + 6a_{3}x + 12a_{4}x^{2} + \cdots) \\
% & \qquad - (1 + 2 a_{2}x+ 3a_{3}x^{2} + 4a_{4}x^{3} + \cdots) - (x +a_{2}x^{2}+a_{3}x^{3} + a_{4}x^{4} + \cdots) \\
% & = (2a_{2}x^{2} + 6a_{3}x^{3} + 12a_{4}x^{4} + \cdots) + (- 2a_{2}x - 6a_{3}x^{2} - 12a_{4}x^{3} + \cdots) + (2a_{2} + 6a_{3}x + 12a_{4}x^{2} + \cdots) \\
% & \qquad + ( -1 - 2 a_{2}x- 3a_{3}x^{2} - 4a_{4}x^{3} + \cdots) + (-x -a_{2}x^{2}-a_{3}x^{3} - a_{4}x^{4} + \cdots) \\
% & = (-1 + 2a_{2})+ (-1-4a_{2}+6a_{3})x + (a_{2}-9a_{3}+12a_{4})x^{2}+\cdots 
% \end{align*}
% 
% Matching terms of each side of this equation
% 
% 
% 
% \begin{align*}
% 0 = 2a_{2}-1  & \Rightarrow a_{2} = \frac{1}{2} \\
% 0 = -1-4\cancelto{\frac{1}{2}}{a_{2}}+6a_{3}  & \Rightarrow a_{3} = \frac{1}{2} \\
% 0 = \cancelto{\frac{1}{2}}{a_{2}}-9\cancelto{\frac{1}{2}}{a_{3}}+12a_{4}  & \Rightarrow a_{4} = \frac{1}{3}
% \end{align*}
% 
% Thus, the solution with the first four non-zero terms is given by
% 
% \begin{equation*}
% \boxed{y(x) = x +\dfrac{1}{2}x^2+\dfrac{1}{2}x^3 + \dfrac{1}{3}x^4 + \ldots}
% \end{equation*}
% \end{solution}
% 
% 
% 
% \begin{problem}
% Find at least the first four nonzero terms in a power series expansion about $x=0$ for the solution to the given initial value problem,
% \begin{equation*}
% y^{\prime} - e^{x} y =0, \quad y(0) =1.
% \end{equation*}
% \end{problem}
% \begin{solution}
% \indent Let $y(x) = \displaystyle \sum_{n=0}^{\infty} a_{n} x^n$.  Taking the first derivative , we obtain:
% 
% \begin{equation*}
% y^{\prime}(x) = \displaystyle \sum_{n=1}^{\infty} n a_{n} x^{n-1}
% \end{equation*}
% We also know that that we can expand $e^{x}$ around $x=0$ into the following power series:
% 
% \begin{equation*}
% e^{x} = \displaystyle \sum_{n=0}^{\infty} \dfrac{x^{n}}{n!}
% \end{equation*}
% For this following problem, we will need to use the Cauchy-product of two infinite series defined as follows:
% 
% \begin{equation*}
% \left(\sum_{n=0}^{\infty} b_{n} x^n \right)\left(\sum_{n=0}^{\infty} d_{n} x^n \right) =\sum_{n=0}^{\infty} c_{n} x^n,
% \end{equation*}
% where
% 
% \begin{equation*}
% c_{n} = \sum_{k=0}^{n} b_{k} d_{n-k}
% \end{equation*}
% 
% Plugging this results into our equation, we obtain:
% 
% \begin{eqnarray*}
% 0 & = & \sum_{n=1}^{\infty} n a_{n} x^{n-1} -\left(\sum_{n=0}^{\infty} \dfrac{x^{n}}{n!}\right) \left(\sum_{n=0}^{\infty} a_{n} x^n\right) \\
% & = & \sum_{n=0}^{\infty} (n+1) a_{n+1} x^{n} -\sum_{n=0}^{\infty}\left(\sum_{k=0}^{n} \dfrac{a_{k}}{(n-k)!} \right)x^n  \\
% & = & \sum_{n=0}^{\infty} \left[(n+1) a_{n+1} - \sum_{k=0}^{n} \dfrac{a_{k}}{(n-k)!} \right] x^n
% \end{eqnarray*}
% 
% This implies that:
% \begin{equation*}
% (n+1) a_{n+1} - \sum_{k=0}^{n} \dfrac{a_{k}}{(n-k)!}=0 , \quad \Rightarrow  a_{n+1} = \dfrac{1}{n+1} \sum_{k=0}^{n} \dfrac{a_{k}}{(n-k)!}
% \end{equation*}
% With our initial condition setting $a_{0}=1$, we have:
% 
% \begin{eqnarray*}
%  a_{1} & = & 1\\
%  a_{2} & = & 1 \\
% a_{3}  & = & \dfrac{5}{6}
% \end{eqnarray*}
% 
% Hence, the first four nonzero terms in a power series expansion about $x=0$ for the solution to the given initial value problem is given by:
% 
% \begin{equation*}
% y(x) = 1 + x + x^2 + \dfrac{5}{6} x^3 + \ldots
% \end{equation*}
% \end{solution}
% 




\end{document}
