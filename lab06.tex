% MATH 201 Lab notes (c) by Carlos Contreras And Philippe Gaudreau
% MATH 201 Lab notes is licensed under a 
% Creative Commons Attribution 4.0 International license.
% CC BY 4.0

% You should have received a copy of the license along with this
% work. If not, see <http://creativecommons.org/licenses/by/4.0/>.

\documentclass[11pt]{article}
% MATH 201 Lab notes (c) by Carlos Contreras And Philippe Gaudreau
% MATH 201 Lab notes is licensed under a 
% Creative Commons Attribution 4.0 International license.
% CC BY 4.0

% You should have received a copy of the license along with this
% work. If not, see <http://creativecommons.org/licenses/by/4.0/>.

%% libraries
\usepackage[utf8x]{inputenc}
\usepackage{xcolor}
\usepackage[left=1.5cm,right=1.5cm,top=2.0cm,bottom=1.5cm,headheight=110pt]{geometry}
\usepackage{amsmath}
\usepackage{amssymb}
\usepackage{graphicx}
\usepackage{xifthen}
\usepackage{sverb}
\usepackage{fancyhdr}
\usepackage{mdframed}
\usepackage{textcomp}

%%%%%%%%%%%%%%%%%%%%%%%%%%%%%%%%%%%%%%%%%%%%%%%%%%%%%%%%%%%%%%%%%%%%%%%
% PDF compiling
\usepackage{ifpdf}
\ifpdf %
        \DeclareGraphicsExtensions{.pdf}%
\else %
        \DeclareGraphicsExtensions{.eps,.ps}%
\fi

%%%%%%%%%%%%%%%%%%%%%%%%%%%%%%%%%%%%%%%%%%%%%%%%%%%%%%%%%%%%%%%%%%%%%%%
% Figures path
\graphicspath{{figures/}}

%%%%%%%%%%%%%%%%%%%%%%%%%%%%%%%%%%%%%%%%%%%%%%%%%%%%%%%%%%%%%%%%%%%%%%%
% Problem counter
\newcounter{Problem}
\setcounter{Problem}{0}

%%%%%%%%%%%%%%%%%%%%%%%%%%%%%%%%%%%%%%%%%%%%%%%%%%%%%%%%%%%%%%%%%%%%%%%
% Definitions
\def\LabSolutions{\clearpage \newpage \begin{center} {\Large \it Solutions} \end{center} \setcounter{Problem}{0}}
\def\QuizSolutions{\newpage \begin{center} {\Large \it Solutions} \end{center} \setcounter{Problem}{0}}
\def\degree{\textdegree}
\def\grade#1{\begin{flushright} {\small [#1]}\\ \end{flushright} \vspace{-10pt}}
\def\codecolor{red!50!black}
\def\code#1{\textcolor{\codecolor}{\tt #1}}
\def\examname#1{%
                \ifnum\value{page}>1%
                    \newpage%
                \else%
                    \vspace*{5pt}%
                \fi%
                \large \textbf{#1} \setcounter{Problem}{0}\vspace{10pt}}
\def\topic#1{\par\needspace{2\baselineskip} \noindent \textsl{\footnotesize #1}}

 
%%%%%%%%%%%%%%%%%%%%%%%%%%%%%%%%%%%%%%%%%%%%%%%%%%%%%%%%%%%%%%%%%%%%%%%
% Environments
\newenvironment{problem}%
     {\stepcounter{Problem}%
      \begin{list}{\textbf{\arabic{Problem}}.~}{}%
      \item}%
     {\end{list}\vspace*{5pt}}

\newenvironment{solution}%
     {\indent \textit{Solution} \newline}%
     {\begin{flushright}$\blacksquare$\end{flushright}}

\newenvironment{preamble}%
     {\vspace*{1em}\begin{mdframed}[leftmargin=1cm,rightmargin=1cm]}%
     {\end{mdframed}\vspace*{1em}}

\newenvironment{multchoice}%
     {\begin{enumerate} \addtolength{\leftskip}{2em} \renewcommand{\labelenumi}{(\alph{enumi})}}
     {\end{enumerate}}

\newenvironment{formulaitem}%
     {\setlength{\leftmargini}{1.5em}\begin{itemize}%
      \setlength\itemindent{-\itemindent}%
      \renewcommand{\labelitemi}{$\rightarrow$}}%
     {\end{itemize}}


%%%%%%%%%%%%%%%%%%%%%%%%%%%%%%%%%%%%%%%%%%%%%%%%%%%%%%%%%%%%%%%%%%%%%%%
% New theorems
\newtheorem{theorem}{Theorem}


\makeatletter

%%%%%%%%%%%%%%%%%%%%%%%%%%%%%%%%%%%%%%%%%%%%%%%%%%%%%%%%%%%%%%%%%%%%%%%
%% New commands
\newcommand*{\course}[1]{\gdef\@course{#1}}
\newcommand*{\coursecode}[1]{\gdef\@coursecode{#1}}
\newcommand*{\term}[1]{\gdef\@term{#1}}
\newcommand*{\instructor}[1]{\gdef\@instructor{#1}}
\newcommand*{\lqnumber}[1]{\gdef\@lqnumber{#1}}
\newcommand*{\labtitle}[1]{\gdef\@labtitle{#1}}
\newcommand*{\quizversion}[1]{\gdef\@quizversion{#1}}
\newcommand*{\probleminfo}[1]{\noindent \textsl{\footnotesize #1}}

% Title header for labs
\newcommand\makelabtitle{%
  \begin{flushleft}%
  {\scshape \@coursecode~\@course~-- University of Alberta}\\%
  {\scshape \@term~-- Labs -- \@instructor}\\%
  {\scshape Authors: Carlos Contreras and Philippe Gaudreau}%
  \end{flushleft}%
  \begin{center}%
  {\Large \bf \@lqnumber:~\@labtitle}%
  \end{center}%
  \thispagestyle{empty}%
  \global\let\@course\@empty%
  \global\let\@labtitle\@empty%
}

% Title header for quizzes
\newcommand\makequiztitle{%
  \begin{flushleft}%
  {\scshape \@coursecode~\@course~-- University of Alberta}\\%
  {\scshape \@term~-- Labs -- \@instructor}\\%
  \end{flushleft}%
  \begin{center}%
  {\Large \bf \@lqnumber} \marginpar{\tiny\tt [\@quizversion]}%
  \end{center}%
  \thispagestyle{empty}%
  \global\let\@course\@empty%
  \global\let\@quizversion\@empty%
}

%%%%%%%%%%%%%%%%%%%%%%%%%%%%%%%%%%%%%%%%%%%%%%%%%%%%%%%%%%%%%%%%%%%%%%%
% Fancy header package
\fancyhead[L]{\small {\scshape \@coursecode~-- \@lqnumber~-- \@term~-- \@instructor}}
\pagestyle{fancy}

\makeatother


\usepackage{hyperref}
\usepackage{cancel}


\begin{document}

\course{Differential Equations}
\coursecode{MATH 201}
\term{Winter 2018}
\instructor{Carlos Contreras}
\lqnumber{Lab 6}
\labtitle{Series solutions: singular points}
\makelabtitle



\topic{Singular points}
\begin{problem}
{ Determine all the singular points of the given differential equation}
\begin{equation*}
(t^2-t-2)x^{\prime \prime} + (t+1)x^{\prime} - (t-2)x =0.
\end{equation*}
\end{problem}

\begin{problem}
Classify all singular points of the following equation
\begin{equation*}
(x^{2}-x+1)y'' -y' - y = 0.
\end{equation*}
\end{problem}

\begin{problem}
Classify all singular points of the following equation
\begin{equation*}
(x^{3}-x)y'' + (\cos (x) - 1)y' + (x^2 -x)y = 0.
\end{equation*}
\end{problem}


\topic{Radius of convergence}

\begin{problem}
Find a minimum value for the radius of convergence of a power series solution about $x_{0} =1$,
\begin{equation*}
y'' + \frac{2}{x^{2}-2x-3}y'-\frac{3}{x^2+x+1}y =0.
\end{equation*}
\end{problem}


% \topic{Solution around singular points}
% 
% \begin{problem}
% Find a power series solution about $x=0$ to the problem
% \[4x^{2}y''+2x^{2}y'-(x+3)y=0.\]
% \end{problem}



%%%%%%%%%%%%%%%%%%%%%%%%%%%%%%%%%%%%%%%%%%%%%%%%%%%%%%%%%%%%%%%%%%%%%%%%%%%%%%%%%%%%%%%%%%%%%%%%%%%



\LabSolutions


Theory and problems from: Nagel, Saff \& Sneider, \textit{Fundamentals of Differential Equations}, Eighth Edition, Adisson--Wesley.

\begin{preamble}
Consider the problem
\begin{equation}
y''+p(x)y'+q(x)y=0.
\label{equ:secondorderlinear}
\end{equation}
\begin{formulaitem}
     \item The point $x_{0}$ is a \textbf{regular point} for \eqref{equ:secondorderlinear} if both $p(x)$ and $q(x)$ are analytic about $x_{0}$ (i.e., have a power series about $x_{0}$ with positive radius of convergence). The point $x_{0}$ is a \textbf{singular point}  for \eqref{equ:secondorderlinear} if it is not regular.
     \item \textbf{Distance} between two complex points $z_{1}=\alpha_{1} + i\beta_{1}$ and $z_{2}=\alpha_{2} + i\beta_{2}$.
     \begin{equation*}
     d(z_{1},z_{2})=|(\alpha_{1}-\alpha_{2}) + i(\beta_{1}-\beta_{2})|=\sqrt{(\alpha_{1}-\alpha_{2})^{2} + (\beta_{1} - \beta_{2})^{2}}.
     \end{equation*}
     \item The \textbf{minimum radius of convergence} of \eqref{equ:secondorderlinear} about $x_{0}$ is
     \[\rho = \min_{k}d(x_{0},z_{k}),\]
     where $z_{k}$ are all singular points of \eqref{equ:secondorderlinear}. That is, the minimun radius of convergence is the distance between $x_{0}$ and the closest singular point of the equation.
%      \item The point $x=x_{0}$ is a \textbf{regular singular point} if it is singular point but not a singularity for both $(x-x_{0})p(x)$ and $(x-x_{0})^2q(x)$. i.e., if 
%      \[\lim_{x\rightarrow x_0}p(x)=\pm \infty, \qquad \lim_{x\rightarrow x_0}q(x)=\pm \infty,\]
%      but
%      \[\lim_{x\rightarrow x_0}(x-x_{0})p(x)<\infty, \qquad \lim_{x\rightarrow x_0}(x-x_{0})^2q(x)<\infty.\]
%      Otherwise, the point $x=x_{0}$ is a \textbf{irregular singular point}.
%      \item Define
%      \[a(x) = (x-x_{0})p(x) = \sum_{n=0}^{\infty}a_{n}(x-x_{0})^{n}= a_{0} + a_{1}(x-x_{0})+a_{2}(x-x_{0})^{2}+\cdots\]
%      \[b(x) = (x-x_{0})^{2}q(x) = \sum_{n=0}^{\infty}b_{n}(x-x_{0})^{n} = b_{0} + b_{1} (x-x_{0}) +b_{2}(x-x_{0})^{2}+\cdots\]
%      \item For $a_{0}$ and $b_{0}$ defined above, the \textbf{indicial equation} is 
%      \[f(r) = r^{2} + (a_{0}-1)r+b_{0}=0,\]
%      has two roots $r_{1}$, the largerst, and $r_{2}$.
%      \item The \textbf{Frobenious method} consists on assuming a solution of the form 
%      \[y_{1}(x)=(x-x_{0})^{r_{1}}\sum_{n=0}^{\infty}c_{n}(x-x_{0})^{n} = \sum_{n=0}^{\infty}c_{n}(x-x_{0})^{n+r_{1}},\]
%      where $r_{1}$ is the largest root of the indicial equation, and 
%      \[c_{n}(r)=\frac{-1}{f(r+n)}\sum_{k=0}^{n-1}[(k+r)a_{n-k}+b_{n-k}]c_{k}.\]
%      Recall that $c_{0}=1$ can be chosen, and $c_{-1}=c_{-2}=\dots=0$. 
%      \item A \textbf{second solution} can be found depending on the other root of the indicial equation, $r_{1}$ and $r_{2}$,
%      \begin{itemize}
%      \item[\textbullet] $r_{1}-r_{2}$ is \textsl{non-integer}. Then
%           \[y_{2}(x)=(x-x_{0})^{r_{2}}\sum_{n=0}^{\infty}c_{n}(x-x_{0})^{n}\]
%           where $c_{n}$ is defined as before
%      \item[\textbullet] $r_{1}=r_{2}$. Then
%           \[y_{2}(x)=y_{1}(x)\ln(x)+(x-x_{0})^{r_{1}}\sum_{n=1}^{\infty}d_{n}(x-x_{0})^{n}\]
%           where
%           \[d_{n}(r)=-\frac{1}{n^{2}}\sum_{k=0}^{n-1}\{[(k+r)a_{n-k}+b_{n-k}]d_{k} + a_{n-k}c_{k}\}-\frac{2}{n}c_{n}.\]
%      \item[\textbullet] $r_{1}-r_{2}=m$ is \textsl{integer}. Then
%           \[y_{2}(x)=\alpha y_{1}(x)\ln(x)+(x-x_{0})^{r_{2}}\sum_{n=0}^{\infty}e_{n}(x-x_{0})^{n}\]
%           where for $0<n<m$
%           \[e_{n}(r_{2})=\frac{-1}{n(n-m)}\sum_{k=0}^{n-1}[(k+r_{2})a_{n-k}+b_{n-k}]e_{k},\]
%           and for $n>m$
%           \begin{equation*}\begin{split} e_{n}(r_{2})= & \frac{-1}{n(n-m)}\sum_{k=0}^{n-1}[(k+r_{2})a_{n-k}+b_{n-k}]e_{k}\\& -\frac{\alpha}{n(n-m)}\left( (2n-m)c_{n-m} + \sum_{k=m}^{n-1}a_{n-k}c_{k-m} \right),\end{split}
%           \end{equation*}
%           and 
%           \[\alpha = \frac{-1}{m}\sum_{k=0}^{m-1}[(k+r_{2})a_{m-k}+b_{m-k}]e_{k}.\]
%      \end{itemize}
\end{formulaitem}
\end{preamble}







%------------------------------------------------------------------------------
\begin{problem}
{ Determine all the singular points of the given differential equation}
\begin{equation*}
(t^2-t-2)x^{\prime \prime} + (t+1)x^{\prime} - (t-2)x =0.
\end{equation*}
\end{problem}
\begin{solution}
Normalizing this equation, we have:

\begin{equation*}
x^{\prime \prime} + \left(\dfrac{t+1}{t^2-t-2}\right)x^{\prime} - \left(\dfrac{t-2}{t^2-t-2}\right)x =0
\end{equation*}

\begin{equation*}
x^{\prime \prime} + \left(\dfrac{t+1}{(t-2)(t+1)}\right)x^{\prime} - \left(\dfrac{t-2}{(t-2)(t+1)}\right)x =0
\end{equation*}

Hence, we have:

\begin{equation*}
p(t) = \dfrac{t+1}{(t-2)(t+1)}, \quad q(t)  = \dfrac{t-2}{(t-2)(t+1)}
\end{equation*}
As we can see,  $p(t)$ is analytic on $(-\infty, 2) \cup (2, \infty)$ and $q(t)$ is analytic on  $(-\infty, -1) \cup (-1, \infty)$. 
Hence, $ t = \{-1,2\}$ are singular points of this equation.
\end{solution}

%------------------------------------------------------------------------------
\begin{problem}
Classify all singular points of the following equation
\begin{equation*}
(x^{2}-x+1)y'' -y' - y = 0.
\end{equation*}
\end{problem}



%------------------------------------------------------------------------------
\begin{problem}
Classify all singular points of the following equation
\begin{equation*}
(x^{3}-x)y'' + (\cos (x) - 1)y' + (x^2 -x) y = 0.
\end{equation*}
\end{problem}





%------------------------------------------------------------------------------
\begin{problem}
Find a minimum value for the radius of convergence of a power series solution about $x_{0} =1$,
\begin{equation*}
y'' + \frac{2}{x^{2}-2x-3}y'-3\frac{3}{x^2+x+1}y =0.
\end{equation*}
\end{problem}
\begin{solution}
The coefficient in front of the $y$ terms has a singular point when $x^2+x+1=0$ and $x^{2}-2x-3=0$.

\begin{equation*}
x^2+x+1 = 0 \Rightarrow x = \dfrac{-1 \pm \sqrt{1-4}}{2} = \dfrac{-1}{2} \pm i \dfrac{\sqrt{3}}{2} 
\end{equation*}

\begin{equation*}
x^2-2x-3 = 0 \Rightarrow x = -1,3
\end{equation*}

Hence, the radius of convergence of a power series solution about $x_{0} =1$ is at least as large as the distance from $x_{0} =1$ to the closest singular point $ \tfrac{-1}{2} \pm i \tfrac{\sqrt{3}}{2} $, $-1$, or $3$.

The distance between $x_{0} =1$ to $x=\tfrac{-1}{2} \pm i \tfrac{\sqrt{3}}{2}$ is given by

\begin{eqnarray*}
d\left(1,\dfrac{-1}{2} \pm i \dfrac{\sqrt{3}}{2}\right)   & = & \sqrt{ \left(1-\dfrac{-1}{2}\right)^2 + \left(0-\dfrac{\pm\sqrt{3}}{2}\right)^2 } = \sqrt{ \left(\dfrac{3}{2}\right)^2 + \left(\dfrac{\sqrt{3}}{2}\right)^2 }\\
& = &  \sqrt{ \dfrac{9}{4} + \dfrac{3}{4} } =  \sqrt{\dfrac{12}{4} } =  \sqrt{3 }, \\
\end{eqnarray*}
while
\[d(1,-1)=2\qquad d(1,3)=2\]
Hence, $x=\tfrac{-1}{2} \pm i \tfrac{\sqrt{3}}{2}$ are the closest singular points and the radius of convergence of a power series solution about $x_{0} =1$ is at least as large as $\sqrt{3}$.
\end{solution}



% %------------------------------------------------------------------------------
% \begin{problem}
% Find a power series solution about $x=0$ to the problem
% \[4x^{2}y''+2x^{2}y'-(x+3)y=0.\]
% \end{problem}
% \begin{solution}
% Fisrt we find $p(x)$ and $q(x)$
% \[p(x)=\frac{1}{2}, \qquad q(x)=-\frac{x+3}{4x^{2}}\]
% where clearly $x_{0}=0$ is a singular point for $q(x)$. Moreover, $x_{0}=0$ is a regular--singular point since 
% \[a(x)=\frac{1}{2}x, \quad \Rightarrow \quad a_{1}=\frac{1}{2}, \quad a_{n}=0, \, n\neq 1,\]
% and
% \[b(x)=-\frac{3}{4}-\frac{1}{4}x, \quad \Rightarrow \quad b_{0}= -\frac{3}{4}, b_{1}=-\frac{1}{4}, \quad b_{n}=0, \, n\geq 2,\]
% are analytic. The indicial equation is given by
% \[f(s)=s^{2}-1s-\frac{3}{4}=0,\qquad r = \frac{4\pm \sqrt{16+48}}{8}\quad \Rightarrow \quad s_{1}=\frac{3}{2}, \,\, s_{2}=-\frac{1}{2}. \]
% Take the largest root $s_{1}=\frac{3}{2}$ so a power series solution has the form
% \[y(x)=\sum_{n=0}^{\infty} c_{n}x^{n+\frac{3}{2}}.\]
% \textit{(Recursive formula)}
% 
% Now, we can use the given recursive formula for $c_{n}$. First we find $f(s_{1}+n)$,
% \[f(s)=(s-\tfrac{3}{2})(s+\tfrac{1}{2})\quad \Rightarrow \quad f(s+n)=n(n+2),\]
% then
% \[c_{n}=\frac{-1}{n(n+2)}\sum_{k=0}^{n-1}[(k+\tfrac{3}{2})a_{n-k}+b_{n-k}]c_{k}.\]
% Note that the only $a_{n}$ term that is not zeros is $a_{1}=\tfrac{1}{2}$, so only $a_{n-k}$ for $n-k=1$ survives, i.e. only $k=n-1$ survives. For $b_{n}$, the terms $b_{0}$ and $b_{1}$ are non-zero, but $n-k=0$ implies $k=n$ which is out of the summation limits ($k$ index up to $n-1$), thus only $k=n-1$ survives. Hence
% \[c_{n}=\frac{-1}{n(n+2)}[(n+\tfrac{1}{2})\tfrac{1}{2}-\tfrac{1}{4}]c_{n-1}=\frac{-1}{2(n+2)}c_{n-1}.\]
% \textit{(Find pattern)}
% 
% Set $c_{0}=1$ to get
% \begin{equation}
% \begin{split}
% c_{1}&=\frac{-1}{2\cdot 3}c_{0}=\frac{-1}{2\cdot 3}\\
% c_{2}&=\frac{-1}{2\cdot 4}c_{1}=\frac{1}{2^{2}\cdot 4\cdot 3}\\
% c_{3}&=\frac{-1}{2\cdot 5}c_{2}=\frac{-1}{2^{3}\cdot 4\cdot 3}=\frac{-1}{2^{2}\cdot 5 !}.
% \end{split}
% \end{equation}
% After a few terms we are convinced that
% \[c_{n}=\frac{(-1)^{n}}{2^{n-1}n!},\]
% which is valid for $n=0$ too.
% Finally,
% \[\boxed{y(x)=\sum_{n=0}^{\infty} \frac{(-1)^{n}}{2^{n-1}n!}x^{n+\frac{3}{2}}}.\]
% 
% For the second solution we see that $s_{1}-s_{2}=2$ is an integer value, thus 
% \[y_{2}(x)=\alpha y_{1}(x)\ln(x)+x^{-\tfrac{1}{2}}\sum_{n=1}^{\infty}e_{n}x^{n}.\]
% First we compute $\alpha$,
% \[\alpha=-\frac{1}{2}\sum_{k=1}^{2-1}\left[ \left(k+\left( -\tfrac{1}{2} \right)\right)a_{m-k} +b_{m-k} \right]c_{k}\underbrace{=}_{m-k=1}-\frac{1}{2}\left[ \left(1 + \left( -\tfrac{1}{2} \right) \right) \tfrac{1}{2} + (-\tfrac{1}{4}) \right] \left( -\tfrac{1}{6} \right) = 0.
% \]
% When $\alpha=0$ several terms cancel (yay!). 
% 
% Then for $n<2$ (or $n=1$)
% \[e_{n}(-\tfrac{1}{2})=\frac{-1}{n(n-2)}\sum_{k=0}^{n-1}\left[ \left( k-\tfrac{1}{2} \right) a_{n-k} +b_{n-k} \right] c_{k} \underbrace{=}_{n=1, \, k=0} \frac{-1}{1(1-2)}\left[ \left( -\tfrac{1}{2}\right)\tfrac{1}{2} + \left(-\tfrac{1}{4}\right) \right] 1 = -\tfrac{1}{2}.\]
% 
% 
% And for $n>2$
% \begin{equation*}\begin{split}
% e_{n}(-\tfrac{1}{2})&=\frac{-1}{n(n-2)}\sum_{k=0}^{n-1}\left[ \left( k-\tfrac{1}{2} \right) a_{n-k} +b_{n-k} \right] c_{k} \underbrace{=}_{n-k=1} \frac{-1}{n(n-2)}\left[ \left( n-1 -\tfrac{1}{2}\right)\tfrac{1}{2} + \left(-\tfrac{1}{2}\right) \right] \frac{(-1)^{n}}{2^{n-1}(n+2)!} \\
% &= \frac{(-1)^{n+1}(n-\tfrac{5}{2})}{2^{n-1}n(n-2)(n+2)!}
% \end{split}
% \end{equation*}
% Thus,
% \[\boxed{y_{2}(x)=x^{-1/2}\left[ 1 - \tfrac{1}{2}x + \sum_{n=3}^{\infty} \frac{(-1)^{n+1}(n-\tfrac{5}{2})}{2^{n-1}n(n-2)(n+2)!}x^{n} \right]}.\]
% \end{solution}


\end{document}
