% MATH 201 Lab notes (c) by Carlos Contreras And Philippe Gaudreau
% MATH 201 Lab notes is licensed under a 
% Creative Commons Attribution 4.0 International license.
% CC BY 4.0

% You should have received a copy of the license along with this
% work. If not, see <http://creativecommons.org/licenses/by/4.0/>.

\documentclass[11pt]{article}
% MATH 201 Lab notes (c) by Carlos Contreras And Philippe Gaudreau
% MATH 201 Lab notes is licensed under a 
% Creative Commons Attribution 4.0 International license.
% CC BY 4.0

% You should have received a copy of the license along with this
% work. If not, see <http://creativecommons.org/licenses/by/4.0/>.

%% libraries
\usepackage[utf8x]{inputenc}
\usepackage{xcolor}
\usepackage[left=1.5cm,right=1.5cm,top=2.0cm,bottom=1.5cm,headheight=110pt]{geometry}
\usepackage{amsmath}
\usepackage{amssymb}
\usepackage{graphicx}
\usepackage{xifthen}
\usepackage{sverb}
\usepackage{fancyhdr}
\usepackage{mdframed}
\usepackage{textcomp}

%%%%%%%%%%%%%%%%%%%%%%%%%%%%%%%%%%%%%%%%%%%%%%%%%%%%%%%%%%%%%%%%%%%%%%%
% PDF compiling
\usepackage{ifpdf}
\ifpdf %
        \DeclareGraphicsExtensions{.pdf}%
\else %
        \DeclareGraphicsExtensions{.eps,.ps}%
\fi

%%%%%%%%%%%%%%%%%%%%%%%%%%%%%%%%%%%%%%%%%%%%%%%%%%%%%%%%%%%%%%%%%%%%%%%
% Figures path
\graphicspath{{figures/}}

%%%%%%%%%%%%%%%%%%%%%%%%%%%%%%%%%%%%%%%%%%%%%%%%%%%%%%%%%%%%%%%%%%%%%%%
% Problem counter
\newcounter{Problem}
\setcounter{Problem}{0}

%%%%%%%%%%%%%%%%%%%%%%%%%%%%%%%%%%%%%%%%%%%%%%%%%%%%%%%%%%%%%%%%%%%%%%%
% Definitions
\def\LabSolutions{\clearpage \newpage \begin{center} {\Large \it Solutions} \end{center} \setcounter{Problem}{0}}
\def\QuizSolutions{\newpage \begin{center} {\Large \it Solutions} \end{center} \setcounter{Problem}{0}}
\def\degree{\textdegree}
\def\grade#1{\begin{flushright} {\small [#1]}\\ \end{flushright} \vspace{-10pt}}
\def\codecolor{red!50!black}
\def\code#1{\textcolor{\codecolor}{\tt #1}}
\def\examname#1{%
                \ifnum\value{page}>1%
                    \newpage%
                \else%
                    \vspace*{5pt}%
                \fi%
                \large \textbf{#1} \setcounter{Problem}{0}\vspace{10pt}}
\def\topic#1{\par\needspace{2\baselineskip} \noindent \textsl{\footnotesize #1}}

 
%%%%%%%%%%%%%%%%%%%%%%%%%%%%%%%%%%%%%%%%%%%%%%%%%%%%%%%%%%%%%%%%%%%%%%%
% Environments
\newenvironment{problem}%
     {\stepcounter{Problem}%
      \begin{list}{\textbf{\arabic{Problem}}.~}{}%
      \item}%
     {\end{list}\vspace*{5pt}}

\newenvironment{solution}%
     {\indent \textit{Solution} \newline}%
     {\begin{flushright}$\blacksquare$\end{flushright}}

\newenvironment{preamble}%
     {\vspace*{1em}\begin{mdframed}[leftmargin=1cm,rightmargin=1cm]}%
     {\end{mdframed}\vspace*{1em}}

\newenvironment{multchoice}%
     {\begin{enumerate} \addtolength{\leftskip}{2em} \renewcommand{\labelenumi}{(\alph{enumi})}}
     {\end{enumerate}}

\newenvironment{formulaitem}%
     {\setlength{\leftmargini}{1.5em}\begin{itemize}%
      \setlength\itemindent{-\itemindent}%
      \renewcommand{\labelitemi}{$\rightarrow$}}%
     {\end{itemize}}


%%%%%%%%%%%%%%%%%%%%%%%%%%%%%%%%%%%%%%%%%%%%%%%%%%%%%%%%%%%%%%%%%%%%%%%
% New theorems
\newtheorem{theorem}{Theorem}


\makeatletter

%%%%%%%%%%%%%%%%%%%%%%%%%%%%%%%%%%%%%%%%%%%%%%%%%%%%%%%%%%%%%%%%%%%%%%%
%% New commands
\newcommand*{\course}[1]{\gdef\@course{#1}}
\newcommand*{\coursecode}[1]{\gdef\@coursecode{#1}}
\newcommand*{\term}[1]{\gdef\@term{#1}}
\newcommand*{\instructor}[1]{\gdef\@instructor{#1}}
\newcommand*{\lqnumber}[1]{\gdef\@lqnumber{#1}}
\newcommand*{\labtitle}[1]{\gdef\@labtitle{#1}}
\newcommand*{\quizversion}[1]{\gdef\@quizversion{#1}}
\newcommand*{\probleminfo}[1]{\noindent \textsl{\footnotesize #1}}

% Title header for labs
\newcommand\makelabtitle{%
  \begin{flushleft}%
  {\scshape \@coursecode~\@course~-- University of Alberta}\\%
  {\scshape \@term~-- Labs -- \@instructor}\\%
  {\scshape Authors: Carlos Contreras and Philippe Gaudreau}%
  \end{flushleft}%
  \begin{center}%
  {\Large \bf \@lqnumber:~\@labtitle}%
  \end{center}%
  \thispagestyle{empty}%
  \global\let\@course\@empty%
  \global\let\@labtitle\@empty%
}

% Title header for quizzes
\newcommand\makequiztitle{%
  \begin{flushleft}%
  {\scshape \@coursecode~\@course~-- University of Alberta}\\%
  {\scshape \@term~-- Labs -- \@instructor}\\%
  \end{flushleft}%
  \begin{center}%
  {\Large \bf \@lqnumber} \marginpar{\tiny\tt [\@quizversion]}%
  \end{center}%
  \thispagestyle{empty}%
  \global\let\@course\@empty%
  \global\let\@quizversion\@empty%
}

%%%%%%%%%%%%%%%%%%%%%%%%%%%%%%%%%%%%%%%%%%%%%%%%%%%%%%%%%%%%%%%%%%%%%%%
% Fancy header package
\fancyhead[L]{\small {\scshape \@coursecode~-- \@lqnumber~-- \@term~-- \@instructor}}
\pagestyle{fancy}

\makeatother


\usepackage{hyperref}
\usepackage{cancel}


\begin{document}

\course{Differential Equations}
\coursecode{MATH 201}
\term{Winter 2018}
\instructor{Carlos Contreras}
\lqnumber{Lab 10}
\labtitle{Eigenvalue problem}
\makelabtitle





\topic{Eigenvalue problems (zero boundary conditions)}
\begin{problem}
Find the eigenvalues $\lambda$ for which the given problem has a nontrivial solution. Also determine the corresponding eigenfunctions.
\begin{equation*}
y^{\prime \prime} - \lambda y =0\,; \qquad  0<x<\pi \,, \qquad y(0)=0  \,,\qquad y(\pi) =0. 
\end{equation*}
\end{problem}


\begin{problem}
Find the eigenvalues $\lambda$ for which the given problem has a nontrivial solution. Also determine the corresponding eigenfunctions.
\begin{equation*}
y^{\prime \prime} + \lambda y =0\,; \qquad  0<x<\pi \,, \qquad y(0)=0  \,,\qquad y(\pi) =0. 
\end{equation*}
\end{problem}


\topic{Eigenvalue problems (zero-derivative boundary conditions)}
\begin{problem}
Find the eigenvalues $\lambda$ for which the given problem has a nontrivial solution. Also determine the corresponding eigenfunctions.
\begin{equation*}
y^{\prime \prime} - \lambda y =0\,; \qquad  0<x<\pi \,, \qquad y'(0)=0  \,,\qquad y'(\pi) =0. 
\end{equation*}
\end{problem}


\topic{Eigenvalue problems (mixed boundary conditions)}
\begin{problem}
 Find the eigenvalues $\lambda$ for which the given problem has a nontrivial solution. Also determine the corresponding eigenfunctions.
\begin{equation*}
y^{\prime \prime} + \lambda y =0\,; \qquad  0<x<\pi \,, \qquad y'(0)=0  \,,\qquad y(\pi) =0. 
\end{equation*}
\end{problem}


\begin{problem}
 Find the eigenvalues $\lambda$ for which the given problem has a nontrivial solution. Also determine the corresponding eigenfunctions.
\begin{equation*}
y^{\prime \prime} + \lambda y =0\,; \qquad  0<x<\pi \,, \qquad y(0) - y'(0)=0  \,,\qquad y(\pi) =0. 
\end{equation*}
\end{problem}



\topic{Eigenvalue problems (general form)}
\begin{problem}
 Find the eigenvalues $\lambda$ for which the given problem has a nontrivial solution. Also determine the corresponding eigenfunctions.
\begin{equation*}
y^{\prime \prime} + 4 y' + \lambda y =0\,; \qquad  0<x<\pi/2 \,, \qquad y(0)=0  \,,\qquad y(\pi/2) =0. 
\end{equation*}
\end{problem}



%%%%%%%%%%%%%%%%%%%%%%%%%%%%%%%%%%%%%%%%%%%%%%%%%%%%%%%%%%%%%%%%%%%%%%%%%%%%%%%%%%%%%%%%%%%%%%%%%%%



\LabSolutions


Theory and problems from: Nagel, Saff \& Sneider, \textit{Fundamentals of Differential Equations}, Eighth Edition, Adisson--Wesley.

\begin{preamble}
\begin{formulaitem}
 
\item For a \textbf{general eigenvalue problem }\[ay'' +b y' + \lambda y = 0, \qquad \text{s.t. boundary conditions}\footnote{\text{s.t. stands for subject to.}},\]
the roots of the auxiliary equation are \[r=-\frac{b}{2a}\pm \frac{\sqrt{b^{2}-4a\lambda}}{2a}= -\frac{b}{2a}\pm \frac{\sqrt{\Delta}}{2a}.\]
The three possible nontrivial solutions depend on the sign of the discriminant $\Delta=b^{2}-4a\lambda$.
\begin{enumerate}
     \item $\Delta=b^{2}-4a\lambda>0.$

Then  $\lambda<\frac{b^{2}}{4a}$, $r_{1}=-\frac{b}{2}+ \frac{\sqrt{b^{2}-4a\lambda}}{2a}$,  $r_{2}=-\frac{b}{2a}- \frac{\sqrt{b^{2}-4a\lambda}}{2}$ (different real roots), and
     \[{y(x)=C_{1}e^{r_{1}x}+C_{2}e^{r_{2}x}}.\]

     \item $\Delta=b^{2}-4a\lambda=0.$

Then  $\lambda=\frac{b^{2}}{4a}$, $r_{1}=r_{2}=\frac{b}{2a}$ (repeated real roots), and
     \[{y(x)=C_{1}e^{r_{1}x}+C_{2}xe^{r_{1}x}}.\]


     \item $\Delta=b^{2}-4a\lambda<0.$

Then  $\lambda>\frac{b^{2}}{4ac}$, then $r_{1,2}=-\frac{b}{2a}\pm i \frac{\sqrt{4a\lambda- b^{2}}}{2a}=-\frac{b}{2a}\pm i\frac{\sqrt{-\Delta}}{2a}$ (complex roots), and
     \[{y(x)=C_{1}e^{-\tfrac{b}{2a}x}\cos\left(\frac{\sqrt{-\Delta}}{2a}x\right) +C_{2}e^{-\tfrac{b}{2a}x}\sin\left(\frac{\sqrt{-\Delta}}{2a}x\right)}.\]
\end{enumerate}
For each one of the cases the goal is to use the boundary conditions to determine values for $\lambda$ leading to nontrivial solutions. A solution will be \textbf{trivial} ($y(x)=0$), when the boundary conditions imply $C_{1}=C_{2}=0$.

\end{formulaitem}
\end{preamble}





\begin{problem}
Find the eigenvalues $\lambda$ for which the given problem has a nontrivial solution. Also determine the corresponding eigenfunctions.
\begin{equation*}
y^{\prime \prime} - \lambda y =0\,; \qquad  0<x<\pi \,, \qquad y(0)=0  \,,\qquad y(\pi) =0. 
\end{equation*}
\end{problem}
\begin{solution}
First we find the roots of the auxiliary equation.
\[r=\pm\sqrt{\lambda}=\pm\sqrt{\Delta}.\]
We consider the three cases.

\par \textsl{Case 1.} $\Delta = \lambda >0$. Then $r_{1}=\sqrt{\lambda}$, $r_{2}=-\sqrt{\lambda}$ are distinct real roots, and the solution to the ODE is
\[y(x)=C_{1}e^{\sqrt{\lambda}x}+C_{2}e^{-\sqrt{\lambda}x}.\]
Using the BC's,
\begin{equation*}
\left\{\begin{array}{rcl}
       C_{1} +C_{2} & = & 0\\
       e^{\sqrt{\lambda}\pi}C_{1} + e^{-\sqrt{\lambda}\pi}C_{2}& = & 0
      \end{array}\right. .
\end{equation*}
For nontrivial solutions we need 
$$\det(A)=\left|\begin{matrix}1&1\\e^{\sqrt{\lambda}\pi} & e^{-\sqrt{\lambda}\pi}\end{matrix}\right|=e^{-\sqrt{\lambda}\pi}-e^{\sqrt{\lambda}\pi}=0,$$ 
which implies $e^{\sqrt{\lambda}\pi}=e^{-\sqrt{\lambda}\pi}$, which is a contradiction since $\lambda\neq 0$. Hence, \textsl{there is no nontrivial solution}.

\par \textsl{Case 2.} $\Delta = \lambda =0$. Then $r_{1}=r_{2}=0$ are repeated real roots, and the solution to the ODE is
\[y(x)=C_{1}x+C_{2}.\]
Using the BC's,
\begin{equation*}
\left\{\begin{array}{rcl}
       C_{2} & = & 0\\
       \pi C_{1} + C_{2} & = & 0
      \end{array}\right. .
\end{equation*}
Which implies $C_{1}=C_{2}=0$. Hence, \textsl{there is no nontrivial solution}.

\par \textsl{Case 3.} $\Delta = \lambda <0 .$ Then $r_{1}=i\sqrt{-\lambda}$, $r_{2}=-i\sqrt{-\lambda}$ are complex roots, and the solution to the ODE is
\[y(x)=C_{1}\cos\sqrt{-\lambda}x+C_{2}\sin\sqrt{-\lambda}x.\]
Using the BC's,
\begin{equation*}
\left\{\begin{array}{rcl}
       C_{1}  & = & 0\\
       \cos(\sqrt{-\lambda}\pi)C_{1} + \sin(\sqrt{-\lambda}\pi)C_{2} & = & 0
      \end{array}\right. ,
\end{equation*}
which implies 
\[\sin(\sqrt{-\lambda}\pi)C_{2}=0 \,\Leftrightarrow\, C_{2} = 0 \text{ or } \sin(\sqrt{-\lambda}\pi)=0.\]
For nontrivial solutions we need 
\[\sin(\sqrt{-\lambda}\pi)=0 \,\Leftrightarrow \, \sqrt{-\lambda}\pi=n\pi\,\Leftrightarrow \, \sqrt{-\lambda}=n.\] 
Thus, the eigenvalues are
\[\boxed{\lambda_{n}=-n^{2}}, \quad n\geq 1,\]
with eigenfunctions (recall that $C_{2}$ is arbitrary)
\[\boxed{y_{n}(x)=C_{n}\sin(nx)}, \quad n\geq 1,\]
for some arbitrary constants $C_{n}$.

\textbf{Note:} the general domain eigenvalue problem
\begin{equation*}
y^{\prime \prime} - \lambda y =0\,; \qquad  0<x<L \,, \qquad y(0)=0  \,,\qquad y(L) =0,
\end{equation*}
has eigenvalues
\[\boxed{\lambda_{n}=-\left( \frac{n\pi}{L}\right)^{2}}, \quad n\geq 1,\]
and eigenfunctions
\[\boxed{y_{n}(x)=C_{n}\sin\left( \frac{n\pi x}{L} \right)}, \quad n\geq 1,\]
for some arbitrary constants $C_{n}$.

Recall the step where
\[\sin(\sqrt{-\lambda}L)=0 \,\Leftrightarrow \, \sqrt{-\lambda}L=n\pi\,\Leftrightarrow \, \sqrt{-\lambda}=\frac{n\pi}{L}.\] 
\end{solution}






\begin{problem}
Find the eigenvalues $\lambda$ for which the given problem has a nontrivial solution. Also determine the corresponding eigenfunctions.
\begin{equation*}
y^{\prime \prime} + \lambda y =0\,; \qquad  0<x<\pi \,, \qquad y(0)=0  \,,\qquad y(\pi) =0. 
\end{equation*}
\end{problem}
\begin{solution}
First we find the roots of the auxiliary equation.
\[r=\pm\sqrt{-\lambda}=\pm\sqrt{\Delta}.\]
We consider the three cases.

\par \textsl{Case 1.} $\Delta = -\lambda >0 \,\, \Rightarrow \, \lambda<0.$ Then $r_{1}=\sqrt{-\lambda}$, $r_{2}=-\sqrt{-\lambda}$ are distinct real roots, and the solution to the ODE is
\[y(x)=C_{1}e^{\sqrt{-\lambda}x}+C_{2}e^{-\sqrt{-\lambda}x}.\]
Using the BC's,
\begin{equation*}
\left\{\begin{array}{rcl}
       C_{1} +C_{2}&=&0\\
      e^{\sqrt{-\lambda}\pi} C_{1} +  e^{-\sqrt{-\lambda}\pi}C_{2}& = &0
      \end{array}\right. .
\end{equation*}
For nontrivial solutions we need 
$$\det(A)=\left|\begin{matrix}1&1\\ e^{\sqrt{-\lambda}\pi} & e^{-\sqrt{-\lambda}\pi}\end{matrix}\right|=e^{-\sqrt{-\lambda}\pi}-e^{\sqrt{-\lambda}\pi}=0,$$ 
which implies $e^{\sqrt{-\lambda}\pi}=e^{-\sqrt{-\lambda}\pi}$, which is a contradiction since $\lambda\neq 0$. Hence, \textsl{there is no nontrivial solution}.

\par \textsl{Case 2.} $\Delta = -\lambda =0 \,\, \Rightarrow \, \lambda=0.$ Then $r_{1}=r_{2}=0$ are repeated real roots, and the solution to the ODE is
\[y(x)=C_{1}x+C_{2}.\]
Using the BC's,
\begin{equation*}
\left\{\begin{array}{rcl}
       C_{2} & = & 0\\
       \pi C_{1} + C_{2} & = & 0
      \end{array}\right. .
\end{equation*}
Hence, \textsl{there is no nontrivial solution}.

\par \textsl{Case 3.} $\Delta = -\lambda <0 \,\, \Rightarrow \, \lambda>0.$ Then $r_{1}=i\sqrt{\lambda}$, $r_{2}=-i\sqrt{\lambda}$ are complex roots, and the solution to the ODE is
\[y(x)=C_{1}\cos\sqrt{\lambda}x+C_{2}\sin\sqrt{\lambda}x.\]
Using the BC's,
\begin{equation*}
\left\{\begin{array}{rcl}
       C_{1} & = & 0\\
       \cos(\sqrt{-\lambda}\pi)C_{1} + \sin(\sqrt{\lambda}\pi)C_{2}&=&0
      \end{array}\right. ,
\end{equation*}
which implies 
\[\sin(\sqrt{\lambda}\pi)C_{2}=0 \,\Leftrightarrow\, C_{2} = 0 \text{ or } \sin(\sqrt{\lambda}\pi)=0.\]
For nontrivial solutions we need 
\[\sin(\sqrt{\lambda}\pi)=0 \,\Leftrightarrow \, \sqrt{\lambda}\pi=n\pi\,\Leftrightarrow \, \sqrt{\lambda}=n.\] 
Thus, the eigenvalues are
\[\boxed{\lambda_{n}=n^{2}}, \quad n\geq 1,\]
with eigenfunctions (recall that $C_{2}$ is arbitrary)
\[\boxed{y_{n}(x)=C_{n}\sin(nx)}, \quad n\geq 1,\]
for some arbitrary constants $C_{n}$.

\textbf{Note:} the general domain eigenvalue problem
\begin{equation*}
y^{\prime \prime} + \lambda y =0\,; \qquad  0<x<L \,, \qquad y(0)=0  \,,\qquad y(L) =0,
\end{equation*}
has eigenvalues
\[\boxed{\lambda_{n}=\left( \frac{n\pi}{L}\right)^{2}}, \quad n\geq 1,\]
and eigenfunctions
\[\boxed{y_{n}(x)=C_{n}\sin\left( \frac{n\pi x}{L} \right)}, \quad n\geq 1,\]
for some arbitrary constants $C_{n}$.

\end{solution}









\begin{problem}
Find the eigenvalues $\lambda$ for which the given problem has a nontrivial solution. Also determine the corresponding eigenfunctions.
\begin{equation*}
y^{\prime \prime} - \lambda y =0\,; \qquad  0<x<\pi \,, \qquad y'(0)=0  \,,\qquad y'(\pi) =0. 
\end{equation*}
\end{problem}
\begin{solution}
First we find the roots of the auxiliary equation.
\[r=\pm\sqrt{\lambda}=\pm\sqrt{\Delta}.\]
We consider the three cases.

\par \textsl{Case 1.} $\Delta = \lambda >0$. Then $r_{1}=\sqrt{\lambda}$, $r_{2}=-\sqrt{\lambda}$ are distinct real roots, and the solution to the ODE is
\[y(x)=C_{1}e^{\sqrt{\lambda}x}+C_{2}e^{-\sqrt{\lambda}x}.\]
We need $y'(x)$ in order to use the initial conditions
\[y'(x)=\sqrt{\lambda}C_{1}e^{\sqrt{\lambda}x}-\sqrt{\lambda}C_{2}e^{-\sqrt{\lambda}x}.\]
Using the BC's,
\begin{equation*}
\left\{\begin{array}{rcl}
       \sqrt{\lambda}C_{1} -\sqrt{\lambda}C_{2} & = & 0\\
       \sqrt{\lambda}e^{\sqrt{\lambda}\pi}C_{1} - \sqrt{\lambda} e^{-\sqrt{\lambda}\pi}C_{2}& = & 0
      \end{array}\right. .
\end{equation*}
For nontrivial solutions we need 
$$\det(A)=\left|\begin{matrix}\sqrt{\lambda}&-\sqrt{\lambda}\\ \sqrt{\lambda}e^{\sqrt{\lambda}\pi} & - \sqrt{\lambda} e^{-\sqrt{\lambda}\pi}\end{matrix}\right|=-\lambda e^{-\sqrt{\lambda}\pi}+\lambda e^{\sqrt{\lambda}\pi}=0,$$ 
which implies $e^{\sqrt{\lambda}\pi}=e^{-\sqrt{\lambda}\pi}$, which is a contradiction since $\lambda\neq 0$. Hence, \textsl{there is no nontrivial solution}.


\par \textsl{Case 2.} $\Delta = \lambda =0$. Then $r_{1}=r_{2}=0$ are repeated real roots, and the solution to the ODE is
\[y(x)=C_{1}x+C_{2}.\]

We need $y'(x)$ in order to use the initial conditions
\[y'(x)=C_{1}.\]

Using the BC's,
\begin{equation*}
\left\{\begin{array}{rcl}
       C_{1} & = & 0\\
       C_{1} & = & 0
      \end{array}\right. .
\end{equation*}
Which implies $C_{2}$ is free to choose. Hence,  a non-trivial solution is the eigenvalue
\[\boxed{\lambda=0},\]
with eigenfunction
\[\boxed{y(x)=C_{0}},\]

\par \textsl{Case 3.} $\Delta = \lambda <0 .$ Then $r_{1}=i\sqrt{-\lambda}$, $r_{2}=-i\sqrt{-\lambda}$ are complex roots, and the solution to the ODE is
\[y(x)=C_{1}\cos\sqrt{-\lambda}x+C_{2}\sin\sqrt{-\lambda}x.\]

We need $y'(x)$ in order to use the initial conditions
\[y'(x)=-\sqrt{-\lambda}C_{1}\sin\sqrt{-\lambda}x+\sqrt{-\lambda}C_{2}\cos\sqrt{-\lambda}x.\]

Using the BC's,
\begin{equation*}
\left\{\begin{array}{rcl}
       \sqrt{-\lambda} C_{2}  & = & 0\\
       -\sqrt{-\lambda}\sin(\sqrt{-\lambda}\pi)C_{1} + \sqrt{-\lambda}\cos(\sqrt{-\lambda}\pi)C_{2} & = & 0
      \end{array}\right. ,
\end{equation*}
Since $\lambda\neq 0$, then $C_{2}=0$, and we need 
\[-\sqrt{-\lambda}\sin(\sqrt{-\lambda}\pi)C_{1}=0 \,\Leftrightarrow\, C_{1} = 0 \text{ or } \sin(\sqrt{-\lambda}\pi)=0.\]
For nontrivial solutions we need 
\[\sin(\sqrt{-\lambda}\pi)=0 \,\Leftrightarrow \, \sqrt{-\lambda}\pi=n\pi\,\Leftrightarrow \, \sqrt{-\lambda}=n.\] 
Thus, in this case the eigenvalues are
\[\boxed{\lambda_{n}=-n^{2}}, \quad n\geq 1,\]
with eigenfunctions (recall that $C_{1}$ is arbitrary)
\[\boxed{y_{n}(x)=C_{n}\cos(nx)}, \quad n\geq 1,\]
for some arbitrary constants $C_{n}$.

\textbf{Note:} we can combine Cases 2 and 3 to write eigenvalues
\[\boxed{\lambda_{n}=-n^{2}}, \quad n\geq 0,\]
with eigenfunctions
\[\boxed{y_{n}(x)=C_{n}\cos(nx)}, \quad n\geq 0.\]

\textbf{Note:} the general domain eigenvalue problem
\begin{equation*}
y^{\prime \prime} - \lambda y =0\,; \qquad  0<x<L \,, \qquad y'(0)=0  \,,\qquad y'(L) =0,
\end{equation*}
has eigenvalues
\[\boxed{\lambda_{n}=-\left( \frac{n\pi}{L}\right)^{2}}, \quad n\geq 0,\]
and eigenfunctions
\[\boxed{y_{n}(x)=C_{n}\cos\left( \frac{n\pi x}{L} \right)}, \quad n\geq 0,\]
for some arbitrary constants $C_{n}$.
\end{solution}




\begin{problem}
 Find the eigenvalues $\lambda$ for which the given problem has a nontrivial solution. Also determine the corresponding eigenfunctions.
\begin{equation*}
y^{\prime \prime} + \lambda y =0\,; \qquad  0<x<\pi \,, \qquad y'(0)=0  \,,\qquad y(\pi) =0. 
\end{equation*}
\end{problem}
\begin{solution}
First we find the roots of the auxiliary equation.
\[r=\pm\sqrt{-\lambda}=\pm\sqrt{\Delta}.\]
We consider the three cases.

\par \textsl{Case 1.} $\Delta = -\lambda >0 \,\, \Rightarrow \, \lambda<0.$ Then $r_{1}=\sqrt{-\lambda}$, $r_{2}=-\sqrt{-\lambda}$ are distinct real roots, and the solution to ODE is
\[y(x)=C_{1}e^{\sqrt{-\lambda}x}+C_{2}e^{-\sqrt{-\lambda}x}.\]
We need $y'(x)$ in order to use the first initial conditions
\[y'(x)=\sqrt{-\lambda}C_{1}e^{\sqrt{-\lambda}x}-\sqrt{-\lambda}C_{2}e^{-\sqrt{-\lambda}x}.\]
Using the BC's,
\begin{equation*}
\left\{\begin{array}{rcl}
        \sqrt{-\lambda}C_{1} - \sqrt{-\lambda}C_{2}&=&0\\
       e^{\sqrt{-\lambda}\pi}C_{1} +e^{-\sqrt{-\lambda}\pi}C_{2}&=&0
      \end{array}\right. .
\end{equation*}
For nontrivial solutions we need 
$$\det(A)=\left|\begin{matrix}\sqrt{-\lambda} & -\sqrt{-\lambda} \\ e^{\sqrt{-\lambda}\pi} & e^{-\sqrt{-\lambda}\pi}\end{matrix}\right|=\sqrt{-\lambda}e^{-\sqrt{-\lambda}\pi}+\sqrt{-\lambda}e^{\sqrt{-\lambda}\pi}=0,$$ 
which implies $e^{\sqrt{-\lambda}\pi}=-e^{\sqrt{-\lambda}\pi}$, which is a contradiction since the exponential function is always positive and $\lambda\neq 0$. Hence, \textsl{there is no nontrivial solution}.

\par \textsl{Case 2.} $\Delta = -\lambda =0 \,\, \Rightarrow \, \lambda=0.$ Then $r_{1}=r_{2}=0$ are repeated real roots, and the solution to ODE is
\[y(x)=C_{1}x+C_{2}.\]
We need $y'(x)$ in order to use the first initial conditions
\[y'(x)=C_{1}.\]
Using the BC's,
\begin{equation*}
\left\{\begin{array}{rcl}
       C_{1} & = &0\\
       C_{1}\pi+C_{2} & = &0
      \end{array}\right. \quad \Leftrightarrow \quad C_{1}=C_{2}=0.
\end{equation*}
Hence, \textsl{there is no nontrivial solution}.

\par \textsl{Case 3.} $\Delta = -\lambda <0 \,\, \Rightarrow \, \lambda>0.$ Then $r_{1}=i\sqrt{\lambda}$, $r_{2}=-i\sqrt{\lambda}$ are complex roots, and the solution to ODE is
\[y(x)=C_{1}\cos\sqrt{\lambda}x+C_{2}\sin\sqrt{\lambda}x.\]
We need $y'(x)$ in order to use the initial conditions
\[y'(x)=-\sqrt{\lambda}C_{1}\sin\sqrt{\lambda}x+\sqrt{\lambda}C_{2}\cos\sqrt{\lambda}x.\]
Using the BC's,
\begin{equation*}
\left\{\begin{array}{rcl}
       \sqrt{\lambda}C_{2}  & = & 0\\
       \cos(\sqrt{\lambda}\pi)C_{1} + \sin(\sqrt{\lambda}\pi)C_{2}&=&0
      \end{array}\right. ,
\end{equation*}
which implies 
\[\cos(\sqrt{\lambda}\pi)C_{1}=0 \,\Leftrightarrow\, C_{1} = 0 \text{ or } \cos(\sqrt{\lambda}\pi)=0.\]
For nontrivial solutions we need 
\[\cos(\sqrt{\lambda}\pi)=0 \,\Leftrightarrow \, \sqrt{\lambda}\pi=n\pi+\frac{\pi}{2}\,\Leftrightarrow \, \sqrt{\lambda}=n+\frac{1}{2}.\] 

Then, the eigenvalues are
\[\boxed{\lambda_{n}=(n+\tfrac{1}{2})^{2}}, \quad n\geq 0,\]
with eigenfunctions
\[\boxed{y_{n}(x)=C_{n}\cos((n+\tfrac{1}{2})x)}, \quad n\geq 0,\]
for some arbitrary constants $C_{n}$.
Note the difference with the previous problem where the BC's are slightly different.
\end{solution}




\begin{problem}
 Find the eigenvalues $\lambda$ for which the given problem has a nontrivial solution. Also determine the corresponding eigenfunctions.
\begin{equation*}
y^{\prime \prime} + \lambda y =0\,; \qquad  0<x<\pi \,, \qquad y(0) - y'(0)=0  \,,\qquad y(\pi) =0. 
\end{equation*}
\end{problem}
\begin{solution}
First we find the roots of the auxiliary equation.
\[r=\pm\sqrt{-\lambda}=\pm\sqrt{\Delta}.\]
We consider the three cases.

\par \textsl{Case 1.} $\Delta = -\lambda >0 \,\, \Rightarrow \, \lambda<0.$ Then $r_{1}=\sqrt{-\lambda}$, $r_{2}=-\sqrt{-\lambda}$ are distinct real roots, and the solution to ODE is
\[y(x)=C_{1}e^{\sqrt{-\lambda}x}+C_{2}e^{-\sqrt{-\lambda}x}.\]
We need $y'(x)$ in order to use the first initial conditions
\[y'(x)=\sqrt{-\lambda}C_{1}e^{\sqrt{-\lambda}x}-\sqrt{-\lambda}C_{2}e^{-\sqrt{-\lambda}x}.\]
Using the BC's,
\begin{equation*}
\left\{\begin{array}{rcl}
       (1-\sqrt{-\lambda})C_{1} + (1+\sqrt{-\lambda})C_{2}&=&0\\
       e^{\sqrt{-\lambda}\pi}C_{1} +e^{-\sqrt{-\lambda}\pi}C_{2}&=&0
      \end{array}\right. .
\end{equation*}
For nontrivial solutions we need 
$$\det(A)=\left|\begin{matrix}1-\sqrt{-\lambda} & 1+\sqrt{-\lambda} \\ e^{\sqrt{-\lambda}\pi} & e^{-\sqrt{-\lambda}\pi}\end{matrix}\right|=(1-\sqrt{-\lambda})e^{-\sqrt{-\lambda}\pi}-(1+\sqrt{-\lambda})e^{\sqrt{-\lambda}\pi}=0.$$ 
Or 
\[1-\sqrt{-\lambda}-(1+\sqrt{-\lambda})e^{2\sqrt{-\lambda}\pi}=0\]
Since $\lambda <0$, then $-e^{2\sqrt{-\lambda}\pi}<1$, and
\[1-\sqrt{-\lambda}-(1+\sqrt{-\lambda})e^{2\sqrt{-\lambda}\pi}<1-\sqrt{-\lambda}-1-\sqrt{-\lambda}=-2\sqrt{-\lambda}<0,\]
which is a contradiction. Thus, no non-trivial solution.

\par \textsl{Case 2.} $\Delta = -\lambda =0 \,\, \Rightarrow \, \lambda=0.$ Then $r_{1}=r_{2}=0$ are repeated real roots, and the solution to ODE is
\[y(x)=C_{1}x+C_{2}.\]
We need $y'(x)$ in order to use the first initial conditions
\[y'(x)=C_{1}.\]
Using the BC's,
\begin{equation*}
\left\{\begin{array}{rcl}
       C_{2} -C_{1}& = &0\\
       C_{1}\pi+C_{2} & = &0
      \end{array}\right. \quad \Leftrightarrow \quad C_{1}=C_{2}=0.
\end{equation*}
Hence, \textsl{there is no nontrivial solution}.

\par \textsl{Case 3.} $\Delta = -\lambda <0 \,\, \Rightarrow \, \lambda>0.$ Then $r_{1}=i\sqrt{\lambda}$, $r_{2}=-i\sqrt{\lambda}$ are complex roots, and the solution to ODE is
\[y(x)=C_{1}\cos\sqrt{\lambda}x+C_{2}\sin\sqrt{\lambda}x.\]
We need $y'(x)$ in order to use the initial conditions
\[y'(x)=-\sqrt{\lambda}C_{1}\sin\sqrt{\lambda}x+\sqrt{\lambda}C_{2}\cos\sqrt{\lambda}x.\]
Using the BC's,
\begin{equation*}
\left\{\begin{array}{rcl}
       C_{1}-\sqrt{\lambda}C_{2}  & = & 0\\
       \cos(\sqrt{\lambda}\pi)C_{1} + \sin(\sqrt{\lambda}\pi)C_{2}&=&0
      \end{array}\right. ,
\end{equation*}
which implies 
\[C_{1}=\sqrt{\lambda}C_{2}\Rightarrow \sqrt{\lambda}\cos(\sqrt{\lambda}\pi)C_{2}+\sin(\sqrt{\lambda}\pi)C_{2}=0 \,\Rightarrow\, \sqrt{\lambda} + \tan(\sqrt{\lambda}\pi)=0,\]
which has infinite solutions. Thus, the eigenvalues are given by the implicit equation
\[\boxed{\lambda_{n}+ \tan\left(\sqrt{\lambda_{n}}\pi\right)=0}, \quad n\geq 1,\]
with eigenfunctions
\[\boxed{y_{n}(x)=C_{n}\left(\sqrt{\lambda_{n}}\cos \left(\sqrt{\lambda_{n}}x\right)+ \sin \left(\sqrt{\lambda_{n}}x\right)\right)}, \quad n\geq 1,\]
for some arbitrary constants $C_{n}$.
\end{solution}



\begin{problem}
 Find the eigenvalues $\lambda$ for which the given problem has a nontrivial solution. Also determine the corresponding eigenfunctions.
\begin{equation*}
y^{\prime \prime} + 4 y' + \lambda y =0\,; \qquad  0<x<\pi/2 \,, \qquad y(0)=0  \,,\qquad y(\pi/2) =0. 
\end{equation*}
\end{problem}
\begin{solution}
First we find the roots of the auxiliary equation.
\[r=-2\pm\sqrt{4-\lambda}=2\pm\sqrt{\Delta}.\]
We consider the three cases.

\par \textsl{Case 1.} $\Delta = 4-\lambda >0 \,\, \Rightarrow \, \lambda<4.$ Then $r_{1}=-2+\sqrt{4-\lambda}$, $r_{2}=-2-\sqrt{4-\lambda}$ are distinct real roots, and the solution to ODE is
\[y(x)=C_{1}e^{r_{1}x}+C_{2}e^{r_{2}x}.\]
Using the BC's,
\begin{equation*}
\left\{\begin{array}{rcl}
       C_{1} +C_{2}&=&0\\
       e^{r_{1}\pi/2}C_{1} + e^{r_{2}\pi/2}C_{2}&=&0
      \end{array}\right. .
\end{equation*}
For nontrivial solutions we need 
$$\det(A)=\left|\begin{matrix} 1 & 1 \\ e^{r_{1}\pi/2} & e^{r_{2}\pi/2}\end{matrix}\right|=e^{r_{1}\pi/2}-e^{r_{2}\pi/2}=0,$$ 
which implies $r_{1}=r_{2}$, which is a contradiction since the roots are different. Hence, \textsl{there is no nontrivial solution}.

\par \textsl{Case 2.} $\Delta = 4-\lambda =0 \,\, \Rightarrow \, \lambda=4.$ Then $r_{1}=r_{2}=-2$ are repeated real roots, and the solution to ODE is
\[y(x)=C_{1}e^{-2x}+C_{2}xe^{-2x}.\]
Using the BC's,
\begin{equation*}
\left\{\begin{array}{rcl}
       C_{1}  &= &0\\
       e^{-\pi}C_{1} +\tfrac{\pi}{2}e^{\pi}C_{2} &= &0
      \end{array}\right. \quad \Leftrightarrow \quad C_{1}=C_{2}=0.
\end{equation*}
Hence, \textsl{there is no nontrivial solution}.

\par \textsl{Case 3.} $\Delta = 4-\lambda <0 \,\, \Rightarrow \, \lambda>4.$ Then $r_{1}=-2+i\sqrt{\lambda-4}$, $r_{2}=-2-i\sqrt{\lambda-4}$ are complex roots, and the solution to ODE is
\[y(x)=C_{1}e^{-2x}\cos\sqrt{\lambda-4}x+C_{2}e^{-2x}\sin\sqrt{\lambda-4}x.\]
Using the BC's,
\begin{equation*}
\left\{\begin{array}{rcl}
       C_{1}  & = & 0\\
       e^{-\pi}\cos(\sqrt{\lambda-4}\tfrac{\pi}{2})C_{1} + e^{-\pi}\sin(\sqrt{\lambda-4}\tfrac{\pi}{2})C_{2}&=&0
      \end{array}\right. ,
\end{equation*}
which implies 
\[\sin(\sqrt{\lambda-4}\tfrac{\pi}{2})C_{2}=0 \,\Leftrightarrow\, C_{2} = 0 \text{ or } \sin(\sqrt{\lambda-4}\tfrac{\pi}{2})=0.\]
For nontrivial solutions we need 
\[\sin(\sqrt{\lambda-4}\tfrac{\pi}{2})=0 \,\Leftrightarrow \, \sqrt{\lambda-4}\tfrac{\pi}{2}=n\pi \,\Leftrightarrow \, \sqrt{\lambda-4}=2n.\] 
Thus, the eigenvalues are
\[\boxed{\lambda_{n}=4n^{2}+4}, \quad n > 0,\]
with eigenfunctions
\[\boxed{y_{n}(x)=C_{n}e^{-2x}\sin(2nx)}, \quad n > 0,\]
for some arbitrary constants $C_{n}$.
\end{solution}






\end{document}
