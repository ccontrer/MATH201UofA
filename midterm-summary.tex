\documentclass[10pt,leqno]{article}
\usepackage[left=.25in,right=.25in,top=.5in,bottom=.75in]{geometry} 
%\geometry{paperwidth=11in,paperheight=8.5in}  
%\setlength{\paperheight}{9in}
%\setlength{\paperwidth}{6in}
%\setlength{\paperheight}{9in}
%\setlength{\paperwidth}{6in}
\usepackage[latin1]{inputenc}
\usepackage{amsmath}
\usepackage{amsfonts}
\usepackage{amssymb}
\usepackage{amsthm}
\setlength{\parindent}{0pt}
\setlength{\parskip}{1ex}
\usepackage{graphicx}
\usepackage{multicol}
\usepackage{textcomp}
\usepackage{url}
\usepackage[hang,flushmargin]{footmisc} 
\usepackage{scalefnt}
\usepackage{hyperref}
\hypersetup{
    pdftitle={Applied Differential Equations},    % title
    pdfauthor={Shapiro},     % author
    pdfsubject={Differential Equations},   % subject of the document
    pdfcreator={pdftex},   % creator of the document
    pdfproducer={Texmaker}, % producer of the document
    pdfkeywords={Differential Equations}, % list of keywords
    colorlinks=true,       % false: boxed links; true: colored links
    linkcolor=blue,          % color of internal links
    citecolor=green,        % color of links to bibliography
    filecolor=magenta,      % color of file links
    urlcolor=blue           % color of external links
}
\begin{document}
\pagestyle{empty}
 
\begin{center}
\begin{LARGE}\textbf{MATH 201\footnote{MATH 201 - Differential Equations -- University of Alberta} -- Midterm summary study guide}\footnote{Carlos Contreras {\scriptsize \textcopyleft \hspace{.5ex} 2018 \url{https://sites.ualberta.ca/~ccontrer/}} Originally: \textit{Differential Equations Study Guide} {\scriptsize \textcopyleft \hspace{.5ex} 2014 \hspace{.5ex} \url{http://integral-table.com}. This work is licensed under the Creative Commons Attribution --  Noncommercial -- No Derivative Works 3.0 United States License. To view a copy of this license, visit:  \url{http://creativecommons.org/licenses/by-nc-nd/3.0/us/}.}
 }\end{LARGE}

\section*{First Order Equations}
\begin{minipage}{4in}
\begin{align}
\textbf{General Form of ODE:\ }& \dfrac{dy}{dx}=f(x,y)\\
\textbf{Initial Value Problem:\ }& y'=f(x,y),\ y(x_0) = y_0
\end{align}
\end{minipage}

\end{center}
\vspace{1em}

\begin{multicols}{2}
\subsection*{Linear Equations}
\begin{align}
\textbf{General Form:\ }& y'+p(x)y=f(x)\\
\textbf{Integrating Factor:\ }& \mu(x) = e^{\int p(x)dx}\\
\implies & \dfrac{d}{dx}\left( \mu(x) y \right) = \mu(x) f(x)\\
\textbf{General Solution:\ }& y=\frac{1}{\mu(x)}\left( \int \mu(x) f(x) dx + C\right)
\end{align}
\subsection*{Homogeneous Equations}
\begin{align}
\textbf{General Form:\ }& y'=f(y/x)\\
\textbf{Substitution:\ }& y=zx \\ \implies & y'=z + xz'
\end{align}
The result is always separable in $z$: 
\begin{equation}
\dfrac{dz}{f(z)-z} = \dfrac{dx}{x}
\end{equation}
\subsection*{Bernoulli Equations}
\begin{align}
\textbf{General Form:\ }& y'+p(x)y=q(x)y^n\\
\textbf{Substitution:\ }& z = y^{1-n}
\end{align}
The result is always linear in $z$:
\begin{equation}
 z' +(1-n)p(x) z = (1-n)q(x)
\end{equation}
\subsection*{Exact Equations}
\begin{align}
\textbf{General Form:\ }& M(x,y)dx + N(x,y)dy = 0 \\
\textbf{Test for Exactness:\ }& \dfrac{\partial M}{\partial y}=\dfrac{\partial N}{\partial x}\\
\textbf{Solution:\ }& \phi=C\text{ where }\\
 M=\dfrac{\partial \phi}{\partial x}&\text{ and } N=\dfrac{\partial \phi}{\partial y}
\end{align}
\columnbreak

\textbf{Method for Solving Exact Equations:}

1. Let $\phi=\int M(x,y)dx + h(y)$

2. Set $\dfrac{\partial \phi}{\partial y} = N(x,y)$

3. Simplify and solve for $h(y)$. 

4. Substitute the result for $h(y)$ in the expression for $\phi$ from step 1 and then set $\phi=0$. This is the solution. 

Alternatively: 

1. Let $\phi=\int N(x,y)dy + g(x)$

2. Set $\dfrac{\partial \phi}{\partial x} = M(x,y)$

3. Simplify and solve for $g(x)$. 

4. Substitute the result for $g(x)$ in the expression for $\phi$ from step 1 and then set $\phi=0$. This is the solution. 

\textbf{Integrating Factors}

\textbf{Case 1:} If $P(x,y)$ depends only on $x$, where
\begin{equation}
P(x,y)=\dfrac{M_y-N_x}{N} \implies \mu(x) = e^{\int P(x)dx}
\end{equation}
then
\begin{equation}
\mu(x) M(x,y) dx + \mu(x) N(x,y) dy = 0
\end{equation}
is exact.

\textbf{Case 2:} If $Q(x,y)$ depends only on $y$, where
\begin{equation}
Q(x,y)=\dfrac{N_x-M_y}{M} \implies \mu(y) = e^{\int Q(y)dy}
\end{equation}
Then 
\begin{equation}
\mu(y) M(x,y) dx + \mu(y)N(x,y) dy =0
\end{equation}
is exact.

\end{multicols}

\newpage

 \begin{center}\section*{Second Order Linear Equations}
\end{center}

\begin{multicols}{2}

\subsection*{General Form of the Equation}
\begin{align} 
\text{{\textbf{General Form:} }}&
a(t)y''+b(t)y'+c(t)y=g(t) \label{eq:general}\\
\text{\textbf{{Homogeneous:} }}&
a(t)y''+b(t)y'+c(t)y=0\label{eq:homog}\\
\text{{\textbf{Standard Form:} }}&
y''+p(t)y'+q(t)y=f(t)
\label{eq:ODE}
\end{align}
%\subsection*{General Solution}
The \textbf{general solution} of \eqref{eq:general} or \eqref{eq:ODE} is 
\begin{equation}
y = C_1 y_1(t) + C_2 y_2 (t) + y_p(t)
\end{equation}
where $y_1(t)$ and $y_2(t)$ are linearly independent solutions of \eqref{eq:homog}.

\subsection*{Linear Independence and The Wronskian}
Two functions $f(x)$ and $g(x)$ are \textbf{linearly dependent} if there exist numbers $a$ and $b$, not both zero, such that $af(x)+bg(x)=0$ for all $x$. If no such numbers exist then they are \textbf{linearly independent}.

If $y_1$ and $y_2$ are two solutions of \eqref{eq:homog} then 
\begin{align}
\text{\textbf{Wronskian:} } & W(t) = y_1(t) y_2'(t) - y_1'(t) y_2(t)\\
\text{\textbf{Abel's Formula:} }&W(t) = Ce^{-\int{p(t)dt}}
\end{align}
and the following are all equivalent: 
\begin{enumerate}
\item $\{y_1,y_2\}$ are linearly independent.
\item $\{y_1,y_2\}$ are a fundamental set of solutions.
\item $W(y_1, y_2)(t_0)\neq 0$ at some point $t_0$.
\item $W(y_1,y_2)(t) \neq 0$ for all $t$.
\end{enumerate}


\subsection*{Initial Value Problem}
%The initial value problem includes two initial conditions at the same point in time, one condition on $y(t)$ and one condition on $y'(t)$. 
\begin{equation}
\left\{
\begin{array}{l}
y''+p(t)y'+q(t)y=0\\ y(t_0)=y_0 \\ y'(t_0)=y_1
\end{array} 
\right.
\end{equation}
%The initial conditions are applied to the entire solution $y=y_h+y_p$. 



\subsection*{Linear Equation: Constant Coefficients} 
%The general form of the homogeneous equation is  
\begin{align}
\text{\textbf{Homogeneous: }} &ay'' + by' + cy=0 \label{eq:linearhomog}\\
\text{\textbf{Non-homogeneous: }} &ay''+by'+cy = g(t)\label{eq:linearnonhomog}\\
\text{\textbf{Characteristic Equation: }} &ar^2 + br + c=0\label{eq:characteristiceq}\\
\text{\textbf{Quadratic Roots: }} &r=\frac{-b\pm\sqrt{b^2-4ac}}{2a}
\end{align}
The solution of \eqref{eq:linearhomog} is given by: 
\begin{align}
\text{\textbf{Real Roots$(r_1 \neq r_2)$:}}\  & y_h = C_1 e^{r_1 t} + C_2 e^{r_2t} \label{eq:LH1}\\
\text{\textbf{Repeated$(r_1 = r_2)$:}}\  & y_h = (C_1 + C_2 t)e^{r_1t}\label{eq:LH2}\\
\text{\textbf{Complex$(r=\alpha\pm i\beta)$:}}\ & y_h=e^{\alpha t}(C_1 \cos \beta t + C_2 \sin \beta t) \label{eq:LH3}
\end{align}
The solution of \eqref{eq:linearnonhomog} is $y=y_p+y_h$ where $y_h$ is given by \eqref{eq:LH1} through \eqref{eq:LH3} and $y_p$ is found by \textbf{undetermined coefficients} or \textbf{variation of parameters}.

%\columnbreak

\begin{center}
\textbf{Heuristics for Undetermined Coefficients\\ (Trial and Error)}
\begin{small}
\begin{tabular}{|l|l|}
\hline 
If $f(t)=$ & then guess that a particular solution $y_p=$ \\
\hline
$P_n(t)$ & $t^s (A_0 + A_1 t + \cdots + A_n t^n)$ \\
\hline 
$P_n(t)e^{at}$ & $t^s (A_0 + A_1 t + \cdots + A_n t^n)e^{at}$ \\
\hline
$P_n(t)e^{at}\sin bt$ & $t^s e^{at} [(A_0 + A_1 t + \cdots + A_n t^n)\cos bt$ \\
or $P_n(t)e^{at}\cos bt$ & \ \ \ \ \ $ + (B_0 + B_1 t + \cdots + B_n t^n)\sin bt]$ \\
\hline 
\end{tabular}
where $s=0,1,2$ if $r=a$ is not a root, is a single root, \\or a double root of \eqref{eq:characteristiceq}, respectively.
\end{small}
\end{center}

\subsection*{Method of Reduction of Order}
When solving \eqref{eq:homog}, given $y_1$, then $y_2$ is given by 
\begin{equation}\label{eq:ROE}
y_2 = y_1\int \dfrac{e^{-\int p(x) dx} dx}{y_1(x)^2}
\end{equation}

\subsection*{Method of Variation of Parameters}
If $y_1(t)$ and $y_2(t)$ are a fundamental set of solutions to \eqref{eq:homog} then a particular solution to \eqref{eq:ODE} is 
\begin{equation}
y_P(t) = -y_1(t) \int \dfrac{y_2(t) f(t)}{W(t)}dt
+ y_2(t) \int \dfrac{y_1(t) f(t)}{W(t)}dt 
\end{equation}

\subsection*{Cauchy-Euler Equation}
\begin{align}
\text{\textbf{ODE: }} &ax^2y''+bxy'+cy=g(x) \label{eq:CEODE}\\
\text{\textbf{Auxilliary Equation: }}& ar(r-1)+br+c=0\label{eq:CEODEAux}
\end{align}
The homogeneous solutions of \eqref{eq:CEODE} depend on the roots of \eqref{eq:CEODEAux}: 
\begin{align}
\text{\textbf{Real Roots$(r_1 \neq r_2)$:}}&\ y_{h} = C_1x^{r_1} + C_2x^{r_2}\\
\text{\textbf{Repeated$(r_1 = r_2)$:}}&\ y_{h} = C_1 x^r + C_2 x^r \ln x\\
\text{\textbf{Complex$(r_{1,2}=\alpha\pm i\beta)$:}}&\ y_{h}=x^{\alpha}[C_1\cos( \beta \ln x) \nonumber \\
                                               & \phantom{y_{h}=x^{\alpha}[} + C_2 \sin (\beta \ln x)] \label{eq:comprootCEODE}
\end{align}
The substitution $x=e^{t}$ transform \eqref{eq:CEODE} into a \eqref{eq:linearnonhomog}
\begin{equation}
ay''(t)+(b-a)y'(t)+cy(t) = g(e^t)
\end{equation}


\vspace{-1ex}\subsection*{Series Solutions}
\begin{equation}
y''+p(x)y'+q(x)y=0 \label{eq:SS}
\end{equation}
If $x_0$ is a \textbf{regular point} of \eqref{eq:SS} then 
\begin{equation}
y(x) = \sum_{k=0}^{\infty}a_k(x-x_k)^k
\end{equation}
At a \textbf{Regular Singular Point } $x_0$: 
\begin{align}
\textbf{Indicial Equation:}\ &r^2+(p(0)-1)r + q(0)=0 \label{eq:indicial}\\
\textbf{First Solution: }\ &y_1=(x-x_0)^{r_1}\sum_{k=0}^{\infty}a_k(x-x_k)^k
\end{align}
Where $r_1$ is the larger real root if both roots of \eqref{eq:indicial}  are real or either root if the solutions are complex. 
\end{multicols}
\end{document}