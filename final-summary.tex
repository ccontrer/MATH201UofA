\documentclass[10pt,leqno]{article}
\usepackage[left=.25in,right=.25in,top=.5in,bottom=.75in]{geometry} 
%\geometry{paperwidth=11in,paperheight=8.5in}  
%\setlength{\paperheight}{9in}
%\setlength{\paperwidth}{6in}
%\setlength{\paperheight}{9in}
%\setlength{\paperwidth}{6in}
\usepackage[latin1]{inputenc}
\usepackage{amsmath}
\usepackage{amsfonts}
\usepackage{amssymb}
\usepackage{amsthm}
\setlength{\parindent}{0pt}
\setlength{\parskip}{1ex}
\usepackage{graphicx}
\usepackage{multicol}
\usepackage{textcomp}
\usepackage{url}
\usepackage[hang,flushmargin]{footmisc} 
\usepackage{scalefnt}
\usepackage{hyperref}
% \usepackage{showframe}
\hypersetup{%
    pdftitle={Applied Differential Equations},    % title
    pdfauthor={Shapiro},     % author
    pdfsubject={Differential Equations},   % subject of the document
    pdfcreator={pdftex},   % creator of the document
    pdfproducer={Texmaker}, % producer of the document
    pdfkeywords={Differential Equations}, % list of keywords
    colorlinks=true,       % false: boxed links; true: colored links
    linkcolor=blue,          % color of internal links
    citecolor=green,        % color of links to bibliography
    filecolor=magenta,      % color of file links
    urlcolor=blue           % color of external links
}
\allowdisplaybreaks%

\begin{document}
\pagestyle{empty}

\begin{center}
\begin{LARGE}\textbf{MATH 201\footnote{MATH 201 - Differential Equations -- University of Alberta} -- Final exam summary study guide}\footnote{Carlos Contreras {\scriptsize \textcopyleft \hspace{.5ex} 2018 \url{https://sites.ualberta.ca/~ccontrer/} This work is licensed under the Creative Commons Attribution --  Noncommercial -- No Derivative Works 3.0 United States License. To view a copy of this license, visit:  \url{http://creativecommons.org/licenses/by-nc-nd/3.0/us/}}. Inspired by: \textit{Differential Equations Study Guide} {\scriptsize \textcopyleft \hspace{.5ex} 2014 \hspace{.5ex} \url{http://integral-table.com}. }
 }\end{LARGE}

\section*{Laplace transform}

\begin{minipage}{4in}
For $f(t)$ piecewise continuous and of exponential order 
\begin{equation}
\mathcal{L}\{f(t)\}(s)=F(s)=\int_{0}^{\infty}f(t)e^{-st} dt 
\end{equation}
\end{minipage}

\end{center}
% \vspace{1em}

\begin{multicols}{2}

\subsection*{Properties and definitions}
\begin{align}
    \textbf{Linearity:\ }& \mathcal{L}\{af+bg\}=aF(s)+bG(s) \\ 
    \textbf{Inverse:\ }& \mathcal{L}^{-1}\{\mathcal{L}\{f(t)\}(s)\}(t)=f(t) \\
    \textbf{Unit step function:\ }& u_{a}(t)=u(t-a)=\left\{
        \begin{array}{ll}
            0, & t<a \\
            1, & a<t
        \end{array} \right. \\
    \textbf{Dirac delta function:\ }& \delta(t-a) = 
    \begin{cases}
        \infty, & t = a \\ 
        0 , & t \neq a
    \end{cases} \\
    & \int_{-\infty}^{\infty} f(t) \delta(t-a) d t = f(a) \\
    \textbf{Convolution:\ }& (f*g)(t)=\int_{0}^{t}f(t-v)g(v)dv \\
    & \phantom{(f*g)(t)}=\int_{0}^{t}f(v)g(t-v)dv \nonumber
\end{align}

\subsection*{Table of Laplace Transforms}

\begin{align}
\mathcal{L}\left\{ 1 \right\} & = \frac{1}{s} \label{equ:laplace:1} \\
\mathcal{L}\left\{ e^{at} \right\} & = \frac{1}{s-a}, \quad s>a  \label{equ:laplace:exp} \\
\mathcal{L}\left\{ t^{n} \right\} & = \frac{n!}{s^{n+1}} \\
\mathcal{L}\left\{ \sin bt \right\} &= \frac{b}{s^{2}+b^{2}} \label{equ:laplace:sin} \\
\mathcal{L}\left\{ \cos bt \right\} &= \frac{s}{s^{2}+b^{2}} \label{equ:laplace:cos} \\
\mathcal{L}\left\{ \sinh bt \right\} &= \frac{b}{s^{2}-b^{2}}  \quad s>|a| \\
\mathcal{L}\left\{ \cosh bt \right\} &= \frac{s}{s^{2}-b^{2}}  \quad s>|a| \\
\mathcal{L}\left\{ e^{at}\sin bt \right\} &= \frac{b}{(s-a)^{2}+b^{2}} \quad s>a \label{equ:laplace:expsin} \\
\mathcal{L}\left\{ e^{at}\cos bt \right\} &= \frac{s-a}{(s-a)^{2}+b^{2}} \quad s>a \label{equ:laplace:expcos} \\
\mathcal{L}\left\{ te^{at} \right\} &= \dfrac{1}{(s-a)^{2}} \quad s>a \label{equ:laplace:texp} \\
\mathcal{L}\left\{ t^{n}e^{at} \right\} &= \dfrac{n!}{(s-a)^{n+1}} \quad s>a \\
\mathcal{L}\left\{ e^{at}f(t) \right\} &= F(s-a) \\
\mathcal{L}\left\{ u(t-a) \right\} &= \frac{1}{s}e^{-as} \\
\mathcal{L}\left\{ f(t-a)u(t-a) \right\} &= e^{-as}F(s) \label{equ:laplace:unit} \\
\mathcal{L}\left\{ \delta(t-a) \right\} &= e^{-as} \label{equ:laplace:delta} \\
\mathcal{L}\left\{ f(x)\delta(t-a) \right\} &= f(a)e^{-as} \\
\mathcal{L}\left\{ f'(t) \right\} &= sF(s)-f(0) \\
\mathcal{L}\left\{ f''(t) \right\} &= s^{2}F(s)-sf(0)-f'(0) \\
% \mathcal{L}\left\{ f^{n}(t) \right\} &= s^{n}F(s)-s^{n-1}f(0)-\cdots-f^{(n-1)}(0) \\
\mathcal{L}\left\{ t^{n}f(t) \right\} &= (-1)^{n}F^{(n)}(s) \\
\mathcal{L}\left\{ \int_{0}^{s}f(\tau)d\tau \right\} &= \frac{1}{s}F(s) \\
\mathcal{L}\left\{ \frac{1}{t}f(t) \right\} &= \int_{s}^{\infty}F(\sigma)d\sigma \\
\mathcal{L}\left\{ (f*g)(t) \right\} &= F(s)G(s) \label{equ:laplace:convolution} \\
\mathcal{L}\left\{ f(ct) \right\} &= \frac{1}{c}F\left(\frac{s}{c}\right)  \quad c>0 \\
\mathcal{L}\left\{ f(t)u(t-a) \right\} &= \mathcal{L}\{f(t+a)\}e^{-as} 
\end{align}
\textbf{Periodic function} $f(t)$ with period $T$
\begin{equation}
\mathcal{L}\{f\}(s)=\frac{1}{1-e^{-Ts}}\int_{0}^{T}e^{-st}f(t)dt
\end{equation}

\subsection*{Laplace transform of an IVP}
For the IVP
\begin{equation}
ay''+by' +cy = g(t),\quad y(0)=y_{0}, \quad y'(0)=y'_{0},
\end{equation}
the Laplace transform of the solution $y(t)$ is
\begin{equation}Y(s)= \underbrace{\frac{(as+b)y_{0}+ay'_{0}}{as^{2}+bs+c}}_{\text{initial conditions}}+\underbrace{\frac{G(s)}{as^{2}+bs+c}}_{\text{particular sol.}}.                          
\end{equation}
Use partial fractions and complete squares methods along with entries in the table (mainly \eqref{equ:laplace:1}--\eqref{equ:laplace:cos}, \eqref{equ:laplace:expsin}--\eqref{equ:laplace:texp}, \eqref{equ:laplace:unit}--\eqref{equ:laplace:delta}, and \eqref{equ:laplace:convolution}), to find $y(t)$.

Some important functions
\begin{align}
\textbf{Transfer function:\ }& H(s)=\frac{1}{as^{2}+bs+c} \\
\textbf{Impulse function:\ }& h(t)=\mathcal{L}^{-1}\{H(s)\}(t)
\end{align}




\end{multicols}

\newpage

\begin{center}
\section*{Heat Equation Problem}
\end{center}

\begin{multicols}{2}

\subsection*{Eigenvalue problem}

\textbf{Homogeneous Dirichlet boundary conditions}
\begin{equation}
    \left\{
    \begin{array}{ll}
        X^{\prime \prime} - \lambda X =0 & 0<x<L \\
        X(0)= X(L) =0 &
    \end{array}
    \right.
    \label{equ:eig:zero}
\end{equation}
has eigenvalues and eigenfunctions solution
\begin{align}
    \textbf{Eigenvalues:\ } & \lambda_{n} = -\left( \frac{n\pi}{L}\right)^{2}, \\
    \textbf{Eigenfunctions:\ } & X_{n}(x) = B_{n}\sin\left( \frac{n\pi x}{L} \right), \quad  n\geq 1
\end{align}

\textbf{Homogeneous Newman boundary conditions}
\begin{equation}
    \left\{
    \begin{array}{ll}
        X^{\prime \prime} - \lambda X =0 & 0<x<L \\
        X'(0)= X'(L) =0 &
    \end{array}
    \right.
    \label{equ:eig:zeroder}
\end{equation}
has eigenvalues and eigenfunctions solution
\begin{align}
    \textbf{Eigenvalues:\ } & \lambda_{n} = -\left( \frac{n\pi}{L}\right)^{2}, \\
    \textbf{Eigenfunctions:\ } & X_{n}(x) = B_{n}\cos\left( \frac{n\pi x}{L} \right), \quad  n\geq 0
\end{align}


\subsection*{Fourier series}
For $f(x)$ piecewise continuous on $[-L, L]$
\begin{align}
    F(x) &= \dfrac{a_{0}}{2} + \sum_{n=1}^{\infty} \left[ a_{n}\cos\left(\dfrac{n\pi x}{L}\right)+b_{n}\sin\left(\dfrac{n\pi x}{L}\right) \right] \\
    a_{n} &= \dfrac{1}{L} \int_{-L}^{L} f(x)\cos\left(\dfrac{n\pi x}{L}\right) dx, \qquad n\geq 0 \\
    b_{n} &= \dfrac{1}{L} \int_{-L}^{L} f(x)\sin\left(\dfrac{n\pi x}{L}\right) dx, \qquad n\geq 1
\end{align}
For all $|x|\leq L$, $F(x)$ \textbf{converges} to
\begin{equation}
    F(x) = \begin{cases} f(x) & \text{if $f$ is continuous at $x$,} \\[0.5em]
        \tfrac{f(x{-})+f(x{+})}{2} & \text{if $f$ is discontinuous at $x$,} \\[0.5em]
        \tfrac{f(-L{+})+f(L{-})}{2} & \text{if $x=L$ or $x=-L$.} \end{cases} 
\end{equation}
% \begin{equation*}
% S(x) = \sum_{n=1}^{\infty} b_{n} \sin \left(\dfrac{n\pi x}{L}\right)
% \end{equation*}
% 
% \begin{equation*}
% C(x) = \frac{a_{0}}{2}+\sum_{n=1}^{\infty} a_{n} \cos \left(\dfrac{n\pi x}{L}\right)
% \end{equation*}
\textbf{Properties of even and odd functions}
\begin{enumerate}
    \item[0.] even: $f(-x)=f(x)$, odd: $-f(-x)=f(x)$
     \item even $\pm$ even = even, \quad even $\times$ even = even
     \item odd $\pm$ odd = odd, \quad odd $\times$ odd = even
     \item odd $\pm$ even = none, \quad odd $\times$ even = odd
     \item $f(x)$ even, then $\int_{-L}^{L}f(x)dx= 2\int_{0}^{L}f(x)dx$
     \item $f(x)$ odd, then $\int_{-L}^{L}f(x)dx= 0$
     \item $f(x)$ even, then $b_{n}=0$, and $a_{n}=\frac{2}{L}\int_{0}^{L}f(x)\cos\left( \tfrac{n\pi x}{L} \right)dx$. \label{enu:fourier:even}
     \item $f(x)$ odd, then  $a_{n}=0$, and $b_{n}=\frac{2}{L}\int_{0}^{L}f(x)\sin\left( \tfrac{n\pi x}{L} \right)dx$. \label{enu:fourier:odd}
\end{enumerate}
\textbf{Fourier cosine (sine) series} of $f(x)$ defined on $[0,L]$ is an even (odd) extension of $f(x)$ to $[-L,L]$, using property \ref{enu:fourier:even} (\ref{enu:fourier:odd}).

\subsection*{Heat equation}

The Heat equation can be solved using separation of variable
\begin{align}
    \textbf{Heat equation:\ }& u_{t}=\alpha u_{xx} \\
    \textbf{Sep. of variables:\ }& u(x,t)=X(x)T(t) \label{equ:heat:separation} \\
    & \Rightarrow \frac{X''}{X} = \frac{T'}{\alpha T} = \lambda \label{equ:heat:normal}
\end{align}
For eigenvalue $\lambda_{n}$, the ODE for $T(t)$ in \eqref{equ:heat:normal} has solution
\begin{equation}
\frac{T'_{n}}{\alpha T_{n}} = \lambda_{n} \quad \Rightarrow \quad T_{n}(t)=A_{n}e^{\alpha\lambda_{n}t}
\end{equation}
By superposition principle and \eqref{equ:heat:separation}, infinitely many solutions give
\begin{equation}
u(x,t)=\sum_{n} X_{n}(x)T_{n}(t) = \sum_{n} X_{n}(x)T_{n}(t)
\end{equation}
Use Boundary Conditions and Inital Conditions to determine the asociated eigenvalue and Fourier series problems, respectively.

\textbf{Dirichlet BC:} $u(0,t) = u(L,t)= 0$ result in an eigenvalue pro\-blem of the form \eqref{equ:eig:zero} and a Fourier sine series of $f(x)$
\begin{align}
    & \left\{ \begin{array}{ll}
            u_{t} = \alpha u_{xx}, & 0<x<L, \quad t>0 \\
            u(0,t) = u(L,t)= 0, & t>0 \\
            u(x,0) = f(x), & 0<x<L
    \end{array}\right. \label{equ:heat:dirichlet} \\
    & \Rightarrow u(x,t)=\sum_{n=1}^{\infty}b_{n}e^{-\alpha \left(\frac{n \pi}{L}\right)^{2}t}\sin\left(\frac{n\pi x}{L}\right)
\end{align}

\textbf{Newmann BC:} $u_{x}(0,t) = u_{x}(L,t)= 0$ result in an eigenvalue problem of the form \eqref{equ:eig:zeroder} and a Fourier cosine series of $f(x)$
\begin{align}
    & \left\{ \begin{array}{ll}
            u_{t} = \alpha u_{xx}, & 0<x<L, \quad t>0 \\
            u_{x}(0,t) = u_{x}(L,t)= 0, & t>0 \\
            u(x,0) = f(x), & 0<x<L
    \end{array}\right. \label{equ:heat:neuman}\\
    & \Rightarrow u(x,t)= \frac{a_{0}}{2}+\sum_{n=1}^{\infty}a_{n}e^{-\alpha \left(\frac{n \pi}{L}\right)^{2}t}\cos\left(\frac{n\pi x}{L}\right)
\end{align}

\textbf{External force $g(x)$ and/or Nonhomogeneous BC:} 
\begin{align}
    &\left\{ \begin{array}{ll}
            u_{t} = \alpha u_{xx} + g(x), & 0<x<L, \quad t>0 \\
            u(0,t) = U_0, \quad u(L,t)= U_L, & t>0 \\
            u(x,0) = f(x), & 0<x<L
    \end{array}\right. \\
    &\text{use\ } u(x,t)=v(x)+w(x,t)
\end{align}
to get a second order problem on $v(x)$ and \eqref{equ:heat:dirichlet} problem on $w(x,t)$ 
\begin{equation}
    \left\{ \begin{array}{l}
        v''(x) = -\tfrac{1}{\alpha}g(x) \\
        v(0) = U_{0}, \quad v(L)= U_{L}
    \end{array}\right., \,
    \left\{ \begin{array}{l}
            w_{t} = \alpha w_{xx}, \\
            w(0,t) = w(L,t)= 0, \\
            w(x,0) = f(x) - v(x)
    \end{array}\right.
\end{equation}
First solve for $v(x)$, then use it to solve for $w(x,t)$
\begin{equation}
\text{If}\quad g(x)=0 \quad \Rightarrow \quad v(x)=(U_{2}-U_{1})\frac{x}{L}+U_{1} 
\end{equation}
$w(x, t)$ and $v(x)$ are called \textbf{transient} and \textbf{steady state} solutions

\textbf{Reaction-diffusion:} put all constants on the $t$ side of \eqref{equ:heat:normal}
\begin{align}
 u_{t}=\alpha u_{xx} + \mu u \quad \Rightarrow \quad \frac{X''}{X} = \frac{T'}{\alpha T} -\mu = \lambda
\end{align}


\end{multicols}



\end{document}
